\chapter{Skillful Thinking in
Meditation}

\epigraph{\emph{Directing thought to loving-kindness is a skillful way
of using and applying the mind that allows us to build a momentum of
wholesomeness.}}{}

In meditation, when we are trying to calm and still the mind, I think
it’s a common tendency to try to eliminate thinking and get rid of the
thinking mind. A common equation comes up: “I think, therefore I suffer.
If I didn’t think, I wouldn’t suffer. Buddhism is for the overcoming of
suffering, so I must annihilate thinking.”

Usually what happens then is either the thinking explodes or we tie
ourselves into knots trying not to think. We might be quite successful
in pushing down and suppressing thought, but it never feels very good.

Directing thought to loving-kindness is a skillful way of using and
applying thought in a way that allows the mind to build a momentum of
skillfulness and wholesomeness. The thoughts are increasingly on the
side of that which is kusala. The very nature of the kusala-citta is
that it’s peaceful, settled, and has a steady quality to it.

Another useful way of directing thought is to the aspects of what in
Pali is called \emph{saṃvega}: urgency, the sense that there is no time
to waste or to fool around. One of the reflections that the monastics
are encouraged to cultivate on a regular basis is: “The days and nights
are relentlessly passing. How well am I spending my time?” That’s a very
useful reflection because it is real: days and nights are relentlessly
passing. They don’t pass benignly. Each day that passes, we’re older and
that much closer to death. That’s the reality: I managed to make it
through another day, but I still have death as a reality that I am going
to have to face.

So that is the sense of urgency. Letting time slip by in a way that’s
frivolous, empty, or just propping up old habit patterns, recycling our
trusty companions of greed, hatred, and delusion: where does that get
us? We’ve all done it. We’ve all seen that it’s not so fruitful.

Bringing up those motivations for urgency—aging, sickness, and death;
the impermanence and uncertainty of the mind and its moods—is a skillful
antidote to complacency. It’s not something to flog ourselves with and
turn into a neurotic obsession, but it’s something that is essential in
terms of needing to prod ourselves and not waste opportunities.

Urgency is not a frantic quality in the mind. A sense of urgency is a
sense of motivation: “Yes, I want to get up and get moving, get up and
have the opportunity to use the time skillfully, to develop virtue, to
train in that which is peaceful and to establish myself in wisdom and
discernment.” These are motivations. Again, it’s not to turn urgency
into something that has a frantic or a manic edge to it; that isn’t
particularly useful. But it is helpful to have a sense of urgency.

Saṃvega is a very positive mental state and motivation. As we use
directed thought, it’s helpful in meditation or in ordinary
circumstances to be able to review the five hindrances. These particular
tendencies of the mind are the fundamental qualities that bring us to a
place of non-peacefulness and lack of clarity.

So, to review the five hindrances: sensual desire, ill will, sloth and
torpor, restlessness and worry, and doubt. In terms of directed thought,
see if they’re present, how to work with them, and how they arise.

If there is sensual desire, what are we directing thought towards? We’re
usually directing thought towards something we perceive as pleasurable,
delightful, and capable of gratification. So that directed thought is
then grabbed by the hindrances and defilements and ends up agitating the
mind.

When thought is directed towards ill will, aversion, irritation, anger,
and displeasure, then it feeds the hindrance. That feeding of the
hindrance makes it healthy and strong. We don’t want healthy and strong
hindrances; we would be better off starving them. Don’t feed them.

I remember Ajahn Chah saying one time, “You’ve got a cat that comes
around—\emph{meow, meow, meow}—and you think, ‘This poor cat!’ And you
feed it. Sure enough, it’s back again, on the porch every morning.” At
Abhayagiri, people wonder why we don’t have any dogs or cats. If you
feed them, they just keep coming around, and they usually tell their
friends as well.

If what we are doing with our hindrances is feeding them, we end up with
multiple hindrance attacks. Pay attention instead to: “If I don’t feed
it, if I don’t direct thought towards the stimulus of the hindrance,
then it fades.” That opens up a space, and we can direct attention
positively. Especially with ill will, the positive directing of thought
towards mettā is a very skillful application of directed thought.

Sloth and torpor happen when we direct thought towards fullness after
the meal. We feel a certain lassitude and disinterest. That feeds sloth
and torpor. So, just as a training, don’t let the mind rest on things
that would tend towards sloth and torpor, to dullness. If we pay
attention to it, then the mind absorbs into it, so that lassitude or
dullness starts to take over. Withdraw attention and thought from those
areas of the mind that are obscure, dull, and amorphous. Especially in
meditation, different mental states drift in and drift out of the mind.
If we let the mind dwell on the more obscure, amorphous, and drifty
states of mind, that’s where we end up, with a mind that’s very dull. So
it’s important to direct attention and feed the ability of the mind to
focus and center on something that is brightening, sharpening, and
clarifying.

How we direct our attention and thought is the same with restlessness
and doubt. So, the five hindrances are a very useful area of reflection
in terms of how best to bring up and establish attention. Because when
we do direct attention in a way that is not dissipated through the
distraction of the five hindrances, then the mind actually becomes very
steady and strong.

There is a strength there. The Buddha compares it to a mountain stream.
If the stream comes down from the mountain and is then channeled off
into different little canals and rivulets, the power of the stream is
dissipated. It’s not able to wash anything away. That is compared to the
energy and flow of the mind as it’s dissipated out into the five
hindrances. When it goes off into these different hindrances, the
strength of that current or stream is just not so strong. Of course, the
opposite is true. When a stream comes down and is not dissipated, it has
tremendous power and can be used for something beneficial.

At Abhayagiri, we’re trying to set up a micro-hydro system. If the
stream goes off in all sorts of different directions, all you get is a
piddly little amount. The turbine goes \emph{kerchunk, kerchunk}, and
you get no power or electricity. Whereas if you can get that channel
going in one direction, you can generate a lot of electricity and you
don’t have to pay the electricity company.

This is natural. These are attributes of nature. It’s just the mind. But
it’s important to be attentive and recognize how these patterns and
tendencies work.

Then, what happens when we direct thought in a particular way towards
the hindrances? What is the result when we’re able to ameliorate and set
aside the five hindrances and allow the attention to settle and focus?
Since the theme of this retreat is mettā, as the five hindrances go into
abeyance, we can bring more attention to the mettā-nimitta and allow
that to shine forth. We can then direct thought in a way that supports
the feeling of well-wishing, softness within the heart, and brightness
that holds oneself and others dear. Allow thought to be directed to the
feeling of loving-kindness.

Thoughts are there to help as reminders. Underlying them is the
particular feeling of spaciousness and warmth in the mind and heart.
Allow that to establish itself through the body, directing it toward
yourself. The mind is able to keep guarding the sustained application of
the thoughts or feelings of loving-kindness.

The Pali words are \emph{vitakka} and \emph{vicāra}. \emph{Vitakka} is
what is translated as thought or directed thought, but it’s both the
thought and the bringing of the thought, feeling, or perception up into
the mind. That’s all vitakka.The arising of the thought of
loving-kindness and directing of attention to loving-kindness, whether
on the level of thought, feeling, or perception, are all vitakka.
\emph{Vicāra} is the sustaining or continuing of that. But there is also
an evaluation that goes on: “How am I holding it? How is it sustaining
itself? What does it feel like? What’s its texture?”

There is the initial bringing up of attention into the mind. Then there
is the sustaining of that. There is an evaluation, but it’s not an
intellectual evaluation. It’s getting a feeling for the texture of that
thought and the feeling within the mind so that it’s sustained and we
are able to look at it in a different way.

Vitakka is bringing that up into view within the mind, in the same way
that I’ve just lifted up the bell-striker. Vicāra is looking at it from
different angles. There is an interest there able to sustain it. We get
distracted. Okay then, vitakka, bring it up again.

That’s what we are doing with the thoughts of loving-kindness or the
meditation object at any time. We are using that process of directed
thought and evaluation so that there is a gaining of momentum of
interest and attention towards that object. With the object of
loving-kindness, it’s that feeling.

It’s not going to be homogeneous or consistent. But that is how we use
vitakka and vicāra. We are not trying to make it absolutely consistent.
Directed thought and evaluation recognize those different textures and
maintain the basic theme of loving-kindness.

Of course, the same is true in terms of whatever meditation object we
are using, say the breath. That is what makes the breath interesting. We
can determine to make this breath interesting just by focusing. It’s
through looking at it, lifting it up, and viewing it from different
angles that we can do that. Then the mind is able to settle a bit longer
and stay steadily on it.

Then part of the evaluation is recognizing that this is starting to feel
good: “My body is feeling more comfortable and settled.” So, there is a
recognition of the effect.

With loving-kindness as a meditation object, we encourage ourselves with
saṃvega. We are attentive to the five hindrances, and then we direct
thought and evaluation to the object of loving-kindness.

One of the very powerful aspects of loving-kindness is that, as
loving-kindness gets stronger, the tendency towards ill will and
aversion drops. It can’t land in the mind so easily. There is a very
strong positive force that also then helps ameliorate the other
hindrances that are akusala. It’s a really good bridge towards the
steadiness of mind that we are cultivating in conjunction with
mindfulness of breathing and awareness of the body. Loving-kindness
allows the mind to make the bridge between the point where the
hindrances are still kicking in a bit and a place where it is more
exclusively wholesome and settled. Wholesome and bright mental states
rely on loving-kindness.

The sense of seclusion from sensual desire and unwholesome mental states
creates a very real shift through which the mind can then settle into
its object. Whether that object is the mettā-nimitta or the breath
depends on what we are comfortable with. But mettā provides a very
stable bridge into that steadiness of mind where the mind is able to
unify. Unification of the mind is disrupted by those little rivulets
going off in different directions: to sensual desire, ill will, sloth
and torpor, restlessness, or doubt.

Allow the mind to bridge that gap and develop interest in something that
is more positive. It’s quite natural. As human beings, we are hardwired
to prefer pleasure to pain. So if we can see something that is really
pleasurable, then the mind can unify with that. Something as pleasurable
as the feeling of loving-kindness is able to bridge across the habit
patterns of restlessness, doubt, or drifting, and go to that steadiness
of mind.

As we practice and work with vitakka and vicāra by bringing up and
sustaining positive thought, something pleasurable, bright, and joyful
arises in the mind. We are not doing anything wrong. It’s all right to
feel that. You may get excited and start to analyze. Instead, give
yourself permission to enjoy meditation.

As human beings, we haplessly bumble into old patterns: “Here I am, back
suffering again. At least it’s familiar and feels secure.” Rather, allow
the mind to explore the regions of well-being and happiness. Confidence
can be there, because it’s not just pleasure arising out of trying to
get something gratifying in terms of a sensual hit. It’s more deeply
satisfying and joyful when based on purity of intention and an unalloyed
quality of skillfulness. It’s wholesome.

In the story of the Buddha, when he was still the Bodhisattva and
striving for awakening, he struggled with ascetic practices. He got to
the point where he was emaciated and frustrated. He then had a memory of
when he was a young child: becoming very peaceful, settled, and
concentrated when he was left on his own during a ceremony. He was about
nine years old and his mind went into stillness. He recalled that memory
and wondered, “Why am I afraid of that happiness that is untainted by
sensuality and unskillful tendencies? Perhaps that’s the path.” There
was a realization that there is actually nothing wrong with that kind of
happiness. Then the Buddha thought that he would never be able to
cultivate that path in his emaciated shape, so he started eating again.
That is when he had the insight into the Middle Way (M 36).

So, cultivation relies on directed thought and uses the investigative
process, the ability to pay attention and allow the mind to channel its
energies. Not attending to things that are disruptive to the mind, but
being able to sustain attention on those things that are nurturing to
the mind and heart: that is how the mind is able to unify. It doesn’t
happen by chance or just by sitting long enough, as if you’re going to
get peaceful if you put in enough hours.

In my early years as a monk, I remember another monk in my monastery. He
had been ordained for several years, but he never appeared very peaceful
to me. As I got to know him, I found that from the time he first started
meditating, he had a journal and kept track of every meditation session
he’d ever had and how many hours and minutes he’d meditated. It was kind
of like, “If I build my account up enough, somehow, I’m going to get
peaceful.”

Well, no. You have to be doing it in the right way. You actually have to
be skillful. So learn those skills of directing attention: What do you
feed? What do you starve? What do you pay attention to? What do you
ignore? What do you encourage? What do you discourage? It takes some
discernment and reflection.

Then recognize that there are intrinsic effects of different mental
states, such as anger, ill will, and restlessness. They have an
intrinsic effect on the mind; they are agitating. Loving-kindness, the
quality of compassion, respect for truth: these are all things that have
a deeply settling effect within the heart.

So allow the mind to attend to that. Allow the mind to sustain that
attention. And then realize how the mind can unify, relying on that.
Sometimes there are overtones of interpretation or meaning that come
from particular words. When we talk about concentration, just the word
concentration tends to have a feeling of “me, concentrating, forcing my
mind on something and holding it there.” But that doesn’t really convey
what we actually do. Unification, on the other hand, involves directed
thought, evaluation, joy, and well-being. The mind can be unified in the
sense that it’s unified in this feeling of well-being and peace, of
being very settled and steady. It’s all working together and comes
together in a place of steadiness. So these are some reflections this
morning for meditation.
