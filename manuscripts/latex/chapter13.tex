\chapter{Spreading Mettā to the Four
Quarters}

\epigraph{\emph{Draw attention to loving-kindness and slowly build
momentum, allowing it to take hold within the heart.}}{}

As we work with the cultivation of mettā, loving-kindness, one cause for
the arising of loving-kindness is not attending to things that are
irritating or conducive to aversion. As a conscious cultivation, we pay
attention to things we perceive as good or as conducive to happiness and
well-being.

We are making a conscious choice. This doesn’t mean that everything is
wonderful and beautiful. But: “Do I really want to attend to things that
stir up the quality of aversion and ill will?” By consciously attending
to those things that we perceive as conducive to well-being and
supporting the wish of well-being, there is a strength of intention that
forms in the mind, and we are able to sustain that flow of well-wishing.

In this light, I think it is important to recognize that we don’t need
to override or dismiss our faculty of discernment, for this is what
enables us to recognize that there are things conducive to aversion and
ill will. It enables us to not give that aversion so much playing time
in the mind.

In reality, if we wanted to catalog all the things that we are offended
and feel disgruntled by, the list could start growing. But there is a
conscious choice not to give it that attention. In recognizing the
benefits of mettā, the physical and mental sense of ease that we
experience, we encourage ourselves to direct our attention to that which
is conducive to the welling up within consciousness of thoughts and
feelings of loving-kindness.

Likening this to food, there are lots of foods that draw our attention,
but if we make our diet cheesecake and junk food, we end up not very
healthy. I have a friend in Australia who sent me a picture captioned,
“Michelangelo’s David returns after two years in America.” It’s a very
funny picture of an obese David. In the same way, the heart needs to be
nourished by positive things that brighten the mind and the heart and
bring a sense of well-being.

The conscious cultivation of loving-kindness is a skillful way of
directing attention. As we do that, it acts almost as a solvent to
dissolve and wash away niggling thoughts and moods of aversion and ill
will. That has a certain power within consciousness because when we see
things in the light of aversion, ill will, and irritation, then
everything is an intrusion and source of irritation. Things get read
through the lens of irritation.

One of the images that the Buddha uses describing the hindrances is
water (A 5.163). When water is left to its own devices, it is clean,
pure, cool, satisfying, and refreshing. We can see through it clearly.
But the image he gives when he compares it to the hindrance of ill will
and aversion is boiling water. It’s boiling and bubbling, and we can’t
see our reflection in it, or see through it clearly. If it bubbles out
of the pan and splashes on us, it burns us. Aversion and ill will have
that same quality.

As we settle the mind down, bring the mind back to its natural state
without adding aversion and ill will, and cool it with the development
of mettā, then it returns to a clear, limpid state. It is refreshing to
look at, to wash with, and to drink. You can do anything you want with
it when the mind is brought to that state.

Recollecting the life of the Buddha as an example of loving-kindness,
there are incidents in which he relied on its application and it had a
powerful effect. One is the story, toward the end of his life, of his
cousin Devadatta, who was also a monk but was very covetous of the
Buddha’s position and wanted to take over. The Buddha didn’t give him
much encouragement, and Devadatta developed a grudge against the Buddha
and plotted to kill him.

One way he did that was to bribe the king’s men who were the keepers of
Nālāgiri, the state elephant but also a rogue elephant, whose job
description was that of executioner. They plied him with alcohol and let
him loose as the Buddha was approaching, so that this enraged elephant
would trample the Buddha. Ānanda, the Buddha’s trustworthy attendant,
saw the elephant coming and jumped in front of the Buddha, wanting to
protect him. The Buddha asked him to step aside and then radiated
loving-kindness. As the story goes, the elephant came barreling up to
the Buddha but then stopped in his tracks, quite close to the Buddha,
affected with that power of loving-kindness. It then bowed and used his
trunk to gather some dust from the Buddha’s feet and put it on his own
head (Cv 7.10–12). It’s a lovely story and could very well have
happened. Even if it didn’t, it’s still a meaningful story.

Rarely are we faced with raging elephants coming towards us, but the
Buddha chose to exercise his trust in the power of loving-kindness. I
think, in having contact with day-to-day circumstances, it is very
helpful to bring up such images: “If the Buddha can rely in such extreme
circumstances on loving-kindness, why don’t I give it a try in this
circumstance and see what happens?” Generally, you will find that it
works. Something changes. It is part of the natural order: all beings
respond to loving-kindness.

I remember one instance when I was still living in Thailand. I was
fairly young in my role as an abbot and was having a conflict with a few
of the monks. There was a constant feeling of agitation and wondering
why these “idiots” didn’t comply with my “rightness,” which was obvious
to me. After some time using that method and not being very successful,
I thought I’d give loving-kindness a try.

A small group of monks had banded together and were creating a lot of
division throughout the community. I decided to start going on a longer
alms round, close to an hour and a half, which was the route that they
had been going on. The whole period is in silence. Every morning, just
before dawn, you walk through the countryside and villages and return to
the monastery.

So, I would begin the day going on alms round with them, taking the
whole time to visualize, generate, and spread loving-kindness to them. I
tried to hold it as clearly as I could throughout the day, but I would
make it a particularly strong focus during alms round. It was really
interesting. About the third day, those monks started coming to me.
Whereas before, communication had completely shut down, they came, we
communicated, and it all opened up from there. We were able to discuss
and figure out where we had gone off track. It was resolved within just
a few days. I found it quite astounding and inspiring.

If you really set your mind to loving-kindness, it has a transformative
effect. Of course, the major effect was on myself, not holding to fixed
views. That helps a lot. It seems pretty mundane, not as spectacular as
stories of the Buddha. If you have experimented with loving-kindness,
you will find that you have a story or know somebody who has a story
about how it has really worked. If we take the time to cultivate and use
it, there will be some kind of effect. This is quite natural. The sense
of loving-kindness is a universal quality that we as human beings all
respond to.

Today I’d like to introduce another formulation of loving-kindness
meditation, based on the one we chanted this morning: “I will abide
pervading one quarter with a mind \ldots{}” The chanting book says “heart,”
but it is usually translated as “mind.” For me, it seems to chant a bit
more nicely with “mind.”

This is the most common formulation of loving-kindness and the
brahmavihāras in the discourses. It comes up in many places: “I will
abide pervading one quarter \ldots{}”—this is directional in the sense that
it is conceived of in terms of the directions, the four quarters—“with a
mind imbued with loving-kindness; likewise the second, likewise the
third, likewise the fourth; so above and below and around and
everywhere; and to all as to myself.” The sense is of spreading
loving-kindness in the different directions—north, south, east, west,
above and below, all around—so that we are allowing the feeling of
loving-kindness to expand.

“To all as to myself” establishes that feeling of loving-kindness within
the heart, then allows it to pervade and extend. That’s the next part of
the chant, “I will abide pervading the all-encompassing world with a
mind imbued with loving-kindness”—throughout the extension of the world
and ultimately all worlds, not just how we conceive of our little
planet, floating in space. Allow that sense of extension, that abundant,
exalted, and immeasurable quality.

“Without hostility and without ill will”: this is the encouragement.
Without any hostility, without allowing ill will, is the key to sending
or shining forth loving-kindness, as in the introduction to the chant,
“Now let us make the four boundless qualities shine forth.” We allow the
heart and mind to be established in those sublime qualities. Allow it to
shine forth, not obstructing it with aversion, ill will, worry, or fear,
all the tendencies that we can add to it.

As we do this, we allow the heart to feel abundant, exalted, and
immeasurable. We see that these qualities are truly beautiful. When we
think of physical beauty in the world, it doesn’t compare to the beauty
of a sublime quality like loving-kindness. It’s an image the Buddha
uses. In comparison with the stars, there isn’t anything that shines as
brightly as the sun. In the same way, there isn’t any quality that
shines as brightly as loving-kindness. Or at night, there isn’t anything
that shines as brightly as the full moon. In the same way, there isn’t
any quality that shines as brightly as loving-kindness.

The Buddha gives various, apt images: nourishing ourselves with those
qualities of loving-kindness, pervading above, below, around, and
everywhere, to all as to myself. There is the sense of establishing
within ourselves this abundant, exalted, and immeasurable feeling.

It wasn’t very long after we did the translation of these chants and
started chanting them that someone said, “When we do these chants, we
are supposed to feel abundant, exalted, and immeasurable, and all I feel
is abandoned, exhausted, and miserable.” But what is needed is to draw
attention to this quality and to slowly build momentum, allowing it to
take hold within the heart.

I think we also need to have a tremendous amount of kindness and
well-wishing for the habits of our minds. They are so deeply ingrained.
Do not be daunted by that. It’s all doable; it’s all workable. It’s drop
by drop and little by little. We can establish these qualities and bring
them to mind, sowing the seeds. Over time, it definitely grows.

The image that the Thai monks use is that when we are cultivating good
qualities and that which is skillful, we have to be willing to be like a
farmer sowing rice. He is just throwing seeds away, out into the fields.
We are not quite sure whether it is going to grow. What’s going to
happen? But it’s the nature of things that when conditions are ripe and
the sun, water, and soil are good, the seeds we “threw away” grow up
into plants. They go through that cycle.

Similarly, with our practice of sowing the seeds of loving-kindness in
the heart, plugging them into consciousness, we find they come back
again. What arises is the sense of spaciousness, warmth, and kindness.

We can use this particular chant as a theme as we sit and bring those
thoughts of loving-kindness into being. “To all as to myself”: bring
one’s attention to oneself. Visualize oneself sitting here. Visualize
oneself as happy, as if one were looking into a mirror at a time when
one was very happy, feeling well, comfortable, and at ease. Bring that
into mind, into consciousness. “Here I am sitting here, happy, at ease,
free from fear, ill will, and anxiety.” Just allow that to establish
itself.

Then use the phrases: “May I be well and happy; may I retain this
feeling of happiness and well-being.” Allow it to settle and pervade the
body. Breathe it in so it pervades from the top of the head down. In
just the same sense that the breath energizes the body, the feeling of
loving-kindness energizes, brightens, relaxes, and softens the head, the
top of the head, the neck, and shoulders.

Breathe out, relaxing any tension, any feeling of conflict or
difficulty. Allow it to release. Breathe in, allowing the warmth of
loving-kindness to pervade the shoulders, the arms, relaxing and
softening the whole chest, back, and abdomen. Softening but energizing
and brightening: that is the nature of loving-kindness energy. The lower
back, the legs, the knees, and down to the soles of the feet: use the
energy of loving-kindness to establish it in oneself.

Again, we use those thoughts or reminders, whether it is planting the
seed with just one word of mettā or some other formulation that may
strike a chord, “Wishing gladness and safety: May all beings be at ease.
May I be well and happy.” Whatever works. We are not trying to be
technically correct. Rather, what is actually going to work and be
helpful?

Feel your way through it. Allow your sense of loving-kindness to
establish itself within the heart and within this being. Then allow that
feeling to expand, pervade, and suffuse. It’s helpful not to be too
ambitious in the beginning. If you are using the different quarters,
pervade one quarter with the mind imbued with loving-kindness. You can
just take it to five, ten, or twenty feet around you.

Pervade one quarter: the north. What’s in front of you? What’s on the
eastern side, the right-hand side? What’s in the south, behind you? On
the western, left-hand side? “Above, below, around, and everywhere, and
to all as to myself.” Allow that to stabilize and feel comfortable.
Breathe into that space. Allow the breath energy and the energy of
loving-kindness to pervade that space, however you conceive of it in
your mind—starting small, near and around you.

Allow that feeling not to be obstructed by discursive thought or the
sense of “Am I doing this wrong? Am I doing it right?” Don’t analyze or
think about it. This is a feeling exercise. Allow that sense of
pervading space with a mind imbued with loving-kindness, whatever it is
and however you have conceived of it, five or ten feet around you.

“Abundant, exalted, immeasurable, and to all as to myself, without
hostility and without ill will.” These are reminders to keep it bright
and clear. Allow it to stabilize. Then use the phrase again: “I will
abide pervading one quarter with a mind imbued with loving-kindness.”

Extend the feeling to whatever feels comfortable. We are in a valley, so
pervade the valley with loving-kindness, sitting here. What’s in front?
All the people, all the beings, whether they are human beings, people in
our group, people that are residents here; the animals that are here;
the beings that are non-material, any earth devas, any other beings that
are here. Extend loving-kindness, including all beings.

The second quarter to the east or off to one’s right: all those beings
in that quarter; those who are in the south, who are behind one; those
who are on one’s left-hand side. Allow that to extend, becoming more
expansive and brightening.

There is the sense that all beings are worthy of our loving-kindness.
All beings appreciate it. When we meet somebody who gives us a gesture
of kindness, it’s always appreciated. It’s natural. So we sit and
generate, spread, suffuse, and pervade the space around the four
quarters, above and below, around and everywhere. Allow that to become
brighter and brighter.

Do not allow yourself to be held back by feelings of limitation. We
limit ourselves with our perception of being unable, unworthy, or not
good enough. We realize that is not necessary. With loving-kindness, it
really doesn’t apply. It’s such a universal quality; allow that to
pervade. “I will abide pervading the all-encompassing world with a mind
imbued with loving kindness; abundant, exalted, immeasurable, without
hostility, and without ill will.”

Recognize that even if it is just a niggling bit of ill will, a flash of
negativity, it’s a limitation of the heart that can be set aside and
dropped. We can allow that feeling of loving-kindness to expand. We can
continue working with it; we can experiment. Allow the feeling of
loving-kindness to extend further out, to more and more beings, not
making any distinction, not dividing the world up into us and them or me
and others. Recognize that the heart of loving-kindness doesn’t make
those distinctions. That is its fundamental quality. That is what gives
it its strength, when we’re able not to get bogged down in those
distinctions. As we make those distinctions—me and others, us and
them—that limitation and feeling of constriction is immediately felt.

If you feel the suffusion of loving-kindness collapsing, just start back
again. Direct loving-kindness to yourself: “And to all as to myself.”
Attend to those thoughts of loving-kindness directed to yourself. That
is the seed, the core, the base that we work from.

There is a teacher in Thailand whose primary teaching tool and
methodology is the cultivation of loving-kindness. Sometimes he will
have people work towards the establishing of loving-kindness to
themselves and won’t encourage them to go beyond that, for a year, two
years, three years. Just keep working on that. That’s the core. Don’t
get ambitious until you can lay that base and establish it solidly
within yourself.

As an exercise, it is interesting to see what happens to the mind as we
allow loving-kindness to expand and pervade. Again, if you feel there is
some obstruction coming up, reestablish that there is no rush to go
anywhere. The main object of the contemplation or meditation is to
establish mettā, the object of loving-kindness within the heart. And
that’s what we allow to grow, nourishing and supporting that.

If we are able to extend that feeling outward and find that it is useful
for our cultivation, all well and good. But again, pay attention to the
experience, investigating, “Is it useful? Is it working? Is it helpful?”
“I will abide, pervading the all-encompassing world with a mind imbued
with loving-kindness; abundant, exalted, immeasurable, without hostility
and without ill will.”
