\chapter{Questions and Answers}

\qaspace
Question: For me, there appears to be a fine line between paying
attention to breath and controlling breath. Is it like with quantum
physics: just being aware changes the phenomena?

\qaspace
Answer: That’s absolutely true in the sense that, as we’re attentive to
the breath, in the same way as if we’re really aware of a thought or an
emotion, a change is already set in motion. It takes a degree of skill
in crafting exactly how we are attentive. Sometimes we pay attention in
a way in which we have a very fixed idea of what we want. We then of
course set ourselves up to confirm that idea.

It’s also true that we can pay attention to the breath with an idea that
we would like to become peaceful, but there needs to be a certain
letting go, a willingness not to have the “doer” get involved and try to
make things be how we conceive of them. That sense of doing turns into
busyness, tightness, and disappointment, or it turns into gratification:
we get what we want, but then we realize that what we wanted doesn’t fit
in with what we would want if we could get it. We tie ourselves in knots
that way.

So we need to direct attention to what is useful at this time. Is it
useful to bring up a pleasing object? Is it useful to put forth a
certain determination, digging our heels in and saying, “I’ve got to
hold the line here”? Or is it recognizing, “I’ve really got to let go
into spaciousness”?

So, there is attention, and the breath is the anchor that we can use to
sustain and cultivate awareness and mindfulness. But once we get into
controlling the breath or trying to control what we are getting out of
it too much, we are going to feel tension, tightness, and frustration.
There is a fine line and that comes through experience. Of course,
suffering is a great teacher.

\qaspace
Q: What happens when someone carefully builds a fire, but it suddenly
turns into a forest fire, uncontrollable, and the person experiences
mania, delusions, and paranoia? What is the spiritual treatment for
this?

\qaspace
A: One of the first spiritual treatments is to stop meditating. When
there isn’t a very solid anchor in what is real, when one is caught up
in delusions, then it’s important not to feed the desire to meditate or
delight in those delusions, manic states, or paranoia.

That doesn’t mean there is no spiritual treatment for such delusions
because I think some of the most spiritual healing work that can be done
is around keeping precepts, associating with people who are virtuous,
and drawing close to teachings and good teachers. It’s very grounding.

Also, there are all of the aspects of generosity, giving, and service.
These are extremely positive qualities that nurture a sense of stability
and feeling good about ourselves. Often, people go into manic states,
delusions, and paranoia because they don’t feel so good about
themselves. They want to become something else and they do it through
meditation. Then they push the mind to distance themselves from how they
perceive themselves to be, trying to become something else. We so-called
normal people also do that. We need to be attentive.

As a young monk and then as a young abbot, I thought meditation was a
panacea, a remedy for everything. Experience has shown me that it’s not.
It is helpful to have pharmaceuticals around sometimes. When I was in
Thailand, there was a stretch of quite a few years, my middle years as a
monk when I first became the abbot of Wat Nanachat, when, at least once
or twice a year, there would be somebody who would flip out.

Part of this is just being in a different culture. There is an actual
type of psychosis that is generated by the disorientation of being in a
different culture. Being in a different culture, in a monastery, and
practicing \ldots{} suffice it to say that in the early days of people
getting interested in Buddhism in Asia, there were some pretty wild and
wooly characters out there. A lot of them were my friends in the
monastery.

In Thailand, a traditional society, when somebody in a monastery goes a
bit off, the teacher would usually put them to work, get them into their
bodies. They would work in the context of being surrounded by and
connecting with people with good virtue and generosity. Don’t send them
off to meditate. Get them a shovel and get them working. It’s really
helpful.

\qaspace
Q: What is the definition of wholesome? The word connotes for me the
’50s era of Ozzie and Harriet.

\qaspace
A: I think I mentioned today that the Pali word is kusala: wholesome,
skillful, beneficial. Intrinsic to a kusala state, or the
\emph{kusala-citta}, the mind that is happy and at ease, is well-being,
\emph{sukha}. There is a sense of peace and stability, and then there is
a freedom from greed, hatred, and delusion, a certain purity or a lack
of the stain of the different defilements. The opposite of this is
\emph{akusala}. With \emph{akusala}, there is a sense of dukkha:
dissatisfaction, suffering, disease in some way. With the
\emph{akusula-citta}, there is always a certain amount of restlessness
or agitation, the inability to settle and be peaceful because of the
presence of the fundamental roots of defilement. Greed, hatred, and
delusion are its different manifestations. It’s a helpful perspective to
see that these are intrinsic to such states.

\qaspace
Q: If someone were to become a monastic at Abhayagiri, how would he go
about doing so? What are the day-to-day activities of a monk? What are a
monk’s duties and responsibilities? How does this change from season to
season and year to year? What do I need to know and consider before
making such a commitment?

\qaspace
A: You have to know it’s not fun. Generally, what would happen would be
somebody would come to the monastery interested in practicing, training,
and becoming a monastic, and he would stay a few months. He comes for an
initial stay of a week, and then he comes back again and asks to stay
longer. If he fits in with the community, he could be accepted to stay
for a longer period of time.

He would be in the monastery for three or four months and then request
the \emph{anagārika} training, like Louis here, who is in white. He
formally takes the eight precepts and makes a commitment for a year. He
can leave at any time, but we encourage people to make that commitment.
After someone has been there a few months, it’s not that difficult to
say, “Yes, I’d like to stay for a year.” But we don’t have people sign
contracts or anything.

Then, as it’s coming close to a year, assuming he wants to go further
with training, he would request to take novice ordination, which is the
ten precepts and brown robes. Sāmanera Kaccāna is a novice. It’s the ten
precepts, which are a little bit different. One does not handle money.
That changes one’s life. When someone goes into brown, we try to
distance him from the kitchen more and also, no driving. Of course,
Sāmanera Kaccāna couldn’t drive anyway, for different reasons! Louis
doesn’t drive either, but again, for different reasons.

Similarly, in about a year, if somebody wants to go forward, then he
could take the full monk’s ordination and make a commitment to stay five
years with the teacher. The decisions are community based, so he could
be an anagārika or a novice longer, but it’s approximately a year for
each stage. This doesn’t mean he would stay all of the time at the
monastery because usually we’ll send monks to study somewhere else for
at least one year during that time. Like Tan Sampajāno: he is finishing
his third year here and, at the end of this Rains Retreat period, he’ll
go to study in Thailand for a year.

As for day-to-day activities, as a general rule, we begin at 5 o’clock
in the morning, with chanting and meditation. Then there is a short
period of chores and then at 7 o’clock, a light breakfast, coffee, and
tea. At 7:30 there is a community meeting during which we divide up the
work for the day. We tend to keep the work period to just the morning.
Some people could be going off to do some construction, while some
people could be going up to work in the forest—trail work, office work,
any number of the things involved in running a monastery.

Everyone comes back together at 11 o’clock for the main meal. We have
our meal together, clean up, and then people can go back to their
dwelling places, which are individual huts scattered through the forest.
People practice on their own in the afternoon. Some may come down for
tea at 5:30, some will stay up in the forest. Then we have the evening
meeting at 7:30.

Right now, in the summertime, we’re having our morning and evening
meetings on the meditation platform in the forest. It doesn’t have a
roof, so once it starts to get too cold or wet, then we have meetings
down below, where we have a small meeting hall.

Different monks will have different duties and responsibilities. The
whole monastery is run by the resident community, so you have to take
responsibility for looking after things, the day-to-day maintenance,
cleaning, building new things. You learn some skills as well as learn
how to sew your robes.

Sāmanera Kaccāna is getting ready for his monk’s ordination, so he is
deeply immersed in sewing his robes. His ordination is October 26, and
his parents were here on the retreat. Unfortunately, his father was ill
and had to leave. He recently had an operation and it wasn’t working, so
they had to leave. They were disappointed. They were the ones who
introduced Sāmanera Kaccāna to Buddhism. Anybody can come to the
monastery.

\qaspace
Q: How will I know I am experiencing loving-kindness?

\qaspace
A: Again, I think that it’s important that we are attending to the
experience of non-ill will, “not dwelling in aversion.” That is a much
more stable or steady indicator of the building of loving-kindness—ill
will, aversion, and irritation are not overwhelming the mind. We aren’t
dwelling on it; we are able to let it go; we are able to recognize its
bane.

\qaspace
Q: Please explain how to cultivate \emph{muditā}. I’ve heard it
described as “poor man’s happiness.”

\qaspace
A: Actually, the previous Upāsikā Day, the teaching day at the monastery
last month, was billed as “The Other Brahmavihāras” because mettā tends
to get all the press, and the other ones tend to be easily overlooked.
\emph{Muditā} is appreciative joy, sympathetic joy, delighting in the
well-being or success of others. Its direct opposite, of course, is
jealousy, comparing oneself to others and generating negative feelings
on account of what others have. Within the human condition, it’s fairly
strong conditioning that we compare ourselves to others and then find
ourselves coming up short.

It’s not that easy to experience a sense of joy at the well-being,
success, capabilities, and intelligence of others and not be intimidated
by or in competition with others. Our conditioning is pretty strong, but
if we are able to direct attention in that way and access that, then our
chances of happiness are increased tremendously.

Generally, our happiness is dependent on our own success: getting what
we want, experiencing something pleasurable that we like. So it’s more
focused around “me.” In contrast, muditā encompasses everybody, and if
our happiness is able to be generated by somebody else’s goodness or
delight and there are another eight billion people on the planet, our
chances go up tremendously. If you’re a betting person, I would put my
bets on that.

\qaspace
Q: What is the difference between \emph{taṇhā} and \emph{lobha}?

\qaspace
A: They are both forms of desire, Pali words for desire, but
\emph{lobha} is a bit more specific in terms of greed: covetousness, the
desire that comes from greed. \emph{Taṇhā} is a much more generic name
for desire. In the very first discourse that the Buddha gave, he pointed
to \emph{taṇhā} as the cause of suffering and then he specified
\emph{kāma-taṇhā}, sensual desire, \emph{bhava-taṇhā}, the desire for
becoming, for being, that sense of the self trying to establish itself,
and \emph{vibhava-taṇhā}, the desire for non-being, non-becoming, which
is pushing away, rejection, fear, aversion.

So \emph{taṇhā} is a much broader term for desire, and it can manifest
in aversion, ill will, fear, as well as sensual desire. \emph{Lobha} is
much more a sense of greed and is not necessarily focused on say, the
sensual, but is an underlying desire seeking an object that is
gratifying.

\qaspace
Q: “Non-contention” is a high-falutin’ word. What does it mean?

\qaspace
A: We are always trying to find language to encapsulate something.
Contention is based in aversion, a sense of competition, and is
generated through a sense of dislike. There is a certain ill will there.
When somebody is contentious, it’s usually fraught with aversion or
irritation. So non-contention is where we aren’t approaching the world,
other people, or our internal mental states from a base of aversion,
fear, or manipulation. It’s more an attitude of: “This is the way it is.
I don’t have to contend with it. I don’t have to manipulate it. I don’t
have to force it to be anything else. It can just be what it is.” That
is the underlying quality of non-contention. We can just let things be
as they are, which isn’t so easy.

\qaspace
Q: Would you talk more about working with the hindrance of doubt?

\qaspace
A: One of the things with doubt is recognizing the tendency to
uncertainty and wavering that comes when there is doubt in the mind.
There is uncertainty, wavering, not being sure. Then how do we fill that
gap? We tend either to fill it with projections of fear and aversion or
with some distraction, such as eating: “Just give me something to eat,
so I don’t have to deal with this doubt.”

This is when mindfulness of the body is really helpful. Tuning in to and
recognizing the actual feeling, that vibration of doubt and uncertainty
that comes into the mind and body and then impels us towards some way of
covering up that doubt. Doubt is a miserable state to be with, so we
want to ameliorate it in some way.

Recognizing and being able to relax around the doubt: this is when the
body is really helpful. We can start to relax around that doubt and
uncertainty, release the tightness and tension that is impelling us to
move into something just to get out of the doubt. Start to tune in to:
“How does it make my heart, mind, and body feel? What do I actually feel
when there’s doubt?”

There is that sense of wavering and uncertainty, when we hold back. To
be able to step into it and not be paralyzed or overwhelmed—not reacting
blindly—enables us to get out of that feeling of doubt and uncertainty.
So these are some skillful ways of working with doubt. What needs to be
recognized and brought out into the light of awareness is: “This is
doubt.”

One of the images that the Buddha gives for the hindrance of doubt is a
dish of muddy water that is then placed in a dark cupboard. The murky
quality of the water is not seen, not clearly understood, and not
recognized, so it is put in the cupboard and hidden away. The first
thing we need to do is to bring it out into the light of awareness and
say: “This is doubt; this is unclear to me. I’m uncertain about this. I
don’t know.”

It’s interesting to pay attention to how many times in a conversation,
when somebody asks you something and you give an answer, you respond not
because you actually know the answer, but because you don’t want anybody
else to recognize that you don’t know. You try to prop up the illusion
of certainty. So, when somebody asks you something, it’s a huge
breakthrough to have the confidence to say, “I don’t really know,”
rather than think, “I’d better say something quickly or they’re going to
think that I’m dumb.” Just say, “Well, I don’t know.” It’s actually
quite easy, but it’s amazing to watch the mind over and over again
propping up the illusion of certainty, when in truth you’re doubtful
about what something is. These are some perspectives on doubt.

\qaspace
Q: I have an ongoing problem with certain vibrations. Here the most
problematic is the recording device. The trunk of my body feels like it
is vibrating as it moves into the room, just like the hum of the
machine. When I get some concentration and I can see the fear, doubt,
and thoughts that arise, I sometimes meet them skillfully; at other
times, I lose skillful means. Any suggestions would most gratefully be
received.

\qaspace
A: One approach would be tuning in to the body itself, recognizing that
we can influence the feelings around us, particularly internally, if
something is agitating. We can pay attention to something like the
breath or the feeling somewhere in the body—picking a spot that feels
comfortable, bringing our attention to a place that feels at ease.

It’s easy for the mind to home in on something that we are irritated by
or feel uncomfortable with. That is what grabs attention. But the
conscious directing of attention to something that is more soothing and
comfortable is a necessary part of living a human life because we are
constantly surrounded by things that we like or dislike, that are
pleasant or unpleasant, that are agreeable or disagreeable.

If the mind drifts into and homes into the disagreeable, the unpleasant,
the things that we dislike, then we keep spinning around that. Certainly
the definition of loving-kindness that I’ve used on several occasions,
dwelling in non-aversion, is really a helpful tool. The mettā practice,
that dwelling in non-aversion, is a helpful base because of the nature
of the world.

One of the classic structures or aspects of the teaching is the worldly
\emph{dhammas}: praise and blame, honor and disrepute, happiness and
suffering, and gain and loss. These are basic things that are going on
all the time. It’s easy to be drawn into or be fearful of say,
criticism: to be elated when we get a bit of praise and to try to
manipulate conditions so that there is more praise and less criticism,
more gain and less loss. It’s natural, but it’s also energy intensive.
Sometimes it works, and sometimes it doesn’t.

It’s much more helpful, from a perspective of practice, to see that
there is that which is pleasant and unpleasant, success and failure,
whether it is material, worldly, or in spiritual practice. Allow that to
take place and allow it to be there without being drawn into the things
that generate the unskillful tendencies or that tend to create more
agitation and suffering in the mind.

As a young monk watching Ajahn Chah, it was tremendously inspiring. It
was as if he was this peaceful, happy presence in the center of the
universe. Things happened around him all the time. People praised him,
people criticized him, many people were drawn to him. Monks would ordain
out of inspiration, monks would leave out of frustration and
disillusionment, and Ajahn Chah was always happy, regardless.

You realize that that is really possible in the human condition. You
don’t need to have conditions that are always perfect. Ajahn Chah really
didn’t have a life of his own. He was just there for people. He wasn’t
looking to have anything special for himself. The nature of the world is
such that there are always things that jar or grate as well as things
that are pleasant. We obviously don’t go around trying to find the
things that are jarring and grating just to test ourselves or make
ourselves strong. We have enough of that in ordinary circumstances. So
it’s more of how to work with letting go, with loving-kindness. These
are the tools.

\qaspace
Q: Would you be willing to repeat the mettā phrases that you taught us
yesterday and today? The only one I can remember is, “May I be
prosperous.”

\qaspace
A: Those filters are working all the time. One is, “May I be well,
happy, peaceful, and prosperous. May no harm come to me. May no
difficulties come to me. May no problems come to me. May I always meet
with spiritual success. May I have the patience, courage, understanding,
and determination to meet and overcome inevitable difficulties,
problems, and failures in life.”

These are the phrases. It’s helpful to get these fully embedded in
memory, so you’re not thinking, “What the heck was that phrase that he
was saying?” Committing it to memory turns it into a kind of mantra that
you can use. This is what is called \emph{parikamma} in Pali, a
repetition that is useful for establishing concentration or consistent
mindfulness. These mettā phrases are skillful tools for setting a tone
within the mind.
