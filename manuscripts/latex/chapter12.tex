\chapter{Mettā and Relinquishment}

A couple of people astutely noticed that I had skipped a few lines of
the \emph{Mettā Sutta}. I consciously did that because I wanted to take
the time to address them. I had spent pretty much the full hour talking
about the \emph{Mettā Sutta} up to that point, and I thought, “I don’t
want to open up this can of worms and go on longer.”

The \emph{Mettā Sutta} lays the basis and foundation of the cultivation
of mettā in terms of conduct, speech, and how we live in the world. It
then goes into the actual cultivation of loving-kindness, giving
different parameters, guidelines, and encouragement, saying we should
sustain the recollection of mettā, which “is said to be the sublime
abiding.”

The sutta ends with, “By not holding to fixed views, the pure-hearted
one, having clarity of vision, being freed from all sense desires, is
not born again into this world.” With just a cursory glance, we might
think, “How did that get tacked on? That doesn’t have anything to do
with mettā.” But I think that we actually bring mettā to complete
fruition by having the insight and understanding not to be caught in
views, opinions, and attachment to our perspectives.

“Being freed from all sense desires” is a radical relinquishing of
attachment. For loving-kindness to manifest fully and come to fruition,
we need to undermine the fundamental roots of attachment, defilement,
and clinging.

So this takes mettā practice and ramps it up to another level, bringing
it to a liberating insight practice. As we cultivate loving-kindness,
the heart becomes more attuned to its own movement. We start to feel the
movement within the heart more clearly.

Wanting to take a fixed view or have a position involves a feeling of
contention and conflict, a feeling of butting up against something or
somebody. How do we deal with that? Do we just swamp it with
loving-kindness and make it go away? That is one way.

Or, do we say, “Well, what’s the root of that? What’s the root of that
feeling, of having to hold a view of right or wrong \ldots{} good or bad \ldots{}
this is correct \ldots{} this is incorrect \ldots{} it’s got to be this way.” What
is the root of that? The root is the sense of self, of “I am.” The mettā
practice is a very skillful means of softening and opening the heart, so
that it’s able to tune in and say, “Well, that’s suffering.” Even if
your view is correct, holding it with a sense of self is still
suffering!

We prop up views and positions and then land in places of suffering. “I
had such loving-kindness going. How did I get back to this point of
suffering again?” That is the way it works when there is still greed,
hatred, and delusion, when there are still the underlying roots of the
\emph{āsava}, the outflows. In order for mettā to come to fruition, we
need to shine a light on the deeper-rooted tendencies.

How do we work with them? Of course, one of the ways of working with
them is with loving-kindness, in the sense that if we set ourselves up
in opposition with our minds, then we are in a constant state of
conflict. So, the kindest act that we can do is to let go of defilement
and attachment. That is the kindest thing we can do for ourselves and
others. We use the cultivation of loving-kindness as a means of
highlighting where we create opposition and conflict, where we hold to a
particular view, and how significant views are in supporting negative
underlying tendencies within the mind.

The Buddha outlined the three outflows or āsavas, as sensual desire
(\emph{kāmāsava}), becoming (\emph{bhavāsava}), and ignorance
(\emph{avijjāsava}—not knowing, not having true knowledge). Very soon
after the Buddha’s passing, commentators added the outflow of views
(\emph{diṭṭhāsava}) because it seemed fundamental as well. These are a
source of suffering and also perpetuate the round of rebirth.

Views are deeply rooted in the mind. Even if they are not argumentative
views, they still involve having to take a position, having to be right
or wrong. In the Buddha’s time, there was a whole list of standard views
that are probably not so meaningful to us. But at the Buddha’s time,
they were the philosophical positions of the day. There were schools of
religious seekers and philosophers who took various positions. Is the
cosmos eternal or not eternal? What is the nature of the self? What is
the nature of the body and the mind? Do enlightened beings exist after
death? Do they not exist after death? Do they both exist and not exist
after death? Or, do they neither exist nor not exist after death?

There were great philosophical debates and conflicts about this. In a
more modern setting, well, pick your view of choice.

There is a discourse in which Ānanda goes into a wanderer’s park around
Rajagaha and goes to a hot springs. He meets a wanderer, who immediately
starts questioning him:

\begin{quotation}
“What teacher do you follow?”

“I follow the son of the Sakyans, Gotama.”

“What view do you hold? That the world is eternal, this alone is true,
anything else is wrong?”

“I don’t hold to such a view, friend.”

“The world is not eternal; this alone is true, anything else is wrong?”

“I don’t hold to such a view, friend.”
\end{quotation}

The wanderer runs through the entire litany of positions and Ānanda
responds with each one, “I don’t hold such a view, friend.”

The wanderer is perplexed with this and asks, “Could it be that you
don’t know or see?” Ānanda assures him that he does know and see (an
idiom for insight). What he knows and sees is that all these different
views are just speculative views. By understanding the foundation of
views, the obsession with views, the origination of views, and the
uprooting of views, there is a true knowing and seeing.

It’s a wonderful perspective on views, opinions, and attachments. To be
accomplished in not having views, one doesn’t have to become a kind of
space cadet, completely out there, incapable of answering a question.
But by recognizing how one starts to fix oneself into a position, and
seeing, “Oh, I’m putting myself in opposition to that person or to that
situation by holding to that view and that’s going to lead to
suffering,” then one is able to step back from that position of holding
to views. That allows one to return to a place of mindfulness and clear
comprehension or discernment. That is what one takes as one’s abiding
place.

We recognize that whatever comes up is just a mental formation within
the mind, just a thought or just a perception. We can have a perception
about something and realize that it’s impermanent, unsatisfactory, and
not-self. Sometimes we can act on views or perceptions if they have a
usefulness at that particular time, but we are not building our home or
sense of self within that.

To tie that in again with loving-kindness: it’s being very kind to
yourself and others because it doesn’t take very long to recollect the
last time you were at loggerheads with somebody because of a particular
view. You think about it afterwards and wonder, “Why did I even go
there? What was the point of that anyway?”

If we are not trapped by views, usually we can respond quite skillfully,
and that is exceedingly useful. Reflect on the sense of non-contention
as a basis for loving-kindness. There is an idiom in the scriptural
language that describes this mental state of attaching to views: “this
alone is true, anything else is wrong.” It isn’t as if we have
consciously thought this out or even articulated it within the mind, but
it is there. We can change our views, but at that particular moment it
feels like, “This is right and everything else is wrong.”

As soon as we are in that kind of position, it’s the basis for
contention and conflict. It’s the basis for feeling irritation and
aversion, whether short or protracted. Ill will is going to be attendant
on holding that particular view.

Try to make this very conscious through the cultivation of
loving-kindness so as not to allow the formation of views to be so
strong. Have a sense of loving-kindness and well-wishing towards
yourself because you are usually the first person to suffer when you are
tightly locked into a particular view. Then, of course, others suffer as
well.

The active application of loving-kindness is not just a nice emotion
that we are able to generate sometimes while we are sitting on our
cushions. It’s a very practical application of how we can interface with
the world around us and not be trapped by fixed views. It lays the basis
for peace and clarity.

One of the underlying principles of the Dhamma that the Buddha pointed
to is this aspect of letting go, relinquishing, and putting down. We can
let go: we can let go of a mood, irritation, or aversion; we can let go
of a view that’s starting to arise; we can let go of a particular
perspective of how I think it has to be; we can let go of sense desires;
and we can let go of the whole construct of “I am.” It’s that letting go
that allows us to access and experience a real peace.

There are a few places in the discourses in which the Buddha is asked
for a way to formulate these teachings in a very succinct way. In the
scriptures the Buddha says that all \emph{dhammas}—\emph{sabbe
dhammā}—are not to be clung to, all \emph{dhammas} are not worthy of
holding on to, all \emph{dhammas} are not to be attached to. All things,
all \emph{dhammas} with a small “d,” are not worthy of clinging. That’s
a very simple phrase that’s helpful for practice, especially in terms of
mettā-bhāvanā. If we get entangled in the obstructions that come up in
the mind or try to analyze and figure them out too much, they hook us
in.

The answer is to be able to come from that place of “all things are not
to be clung to.” Maybe there is a mood of distraction, irritation, or
confusion, but all \emph{dhammas} are not to be clung to. Then we need
to bring into being the cultivation of loving-kindness, which is an
expansive, wholesome quality that we ultimately have to let go of. But
again, it’s a useful bridge to putting our effort and attention into
things that help establish and sustain a sense of peace and clarity. So,
the act of letting go is a very useful tool.

One of the places that this teaching comes up is in a very interesting
discourse, a well-known teaching that most people who have been on a
meditation retreat have probably heard referred to in one way or
another. It’s the discourse given to Moggallāna when he was meditating
and experiencing drowsiness (A 7:61). It was shortly after Moggallāna
was ordained. He is practicing diligently but is being overcome by
drowsiness, sloth, and torpor. The Buddha comes and gives him
instruction.

That set of instructions is, first, to make your object of meditation
clear. If that doesn’t work, then use an image of light to bring
brightness to the mind. Use a recitation if that image of brightness
doesn’t work. If that doesn’t work, pull your ears, rub your skin, and
get some energy going somehow. If that doesn’t work, go out and look at
the sky, look at the stars. If that doesn’t work, do some walking
meditation. In the end, if none of this works, then lie down and have a
rest, mindfully establishing a time for getting up and continuing your
practice.

The Buddha continues on and gives Moggallāna instruction in being
careful not to get too involved with or to draw too close to laypeople,
because laypeople have busy lives, and if you’re trying to be a part of
their lives and they ignore you, then you wonder, “Oh, what did I do
wrong?” There are endless complications. You’re a monk, do what monks
do. Let laypeople do what laypeople do. Don’t get too entangled in other
people’s lives.

Then the Buddha gives Moggallāna instruction in maintaining a sense of
circumspection in speech because inevitably, a lot of conversation and
talking ends up in a lot of conflict. There are always
misunderstandings. Restrain yourself in speech, paying attention to
maintaining humility and not being proud wherever you go or whatever you
do: good instructions for a new monk.

Then Moggallāna asks the Buddha, “What is the core of the teachings?
What is their basis?” That’s when the Buddha says that the fundamental
teaching is that all \emph{dhammas} are not to be clung to. When you
don’t cling to things, you are not building the momentum of habits and
attachments. When you are not building that momentum of habits and
attachments, you are not creating a sense of being sustained by them.
When you are not being sustained by all those habit patterns, then you
can dwell free and unsustained within the world and the body-mind
complex. It’s a wonderful teaching. It’s very, very direct.

As we reflect and investigate, we see how views—attachments to a sense
of self, our sense of what we think is right and wrong, how things
should and shouldn’t be—are endlessly frustrating, agitating, and a
source of dukkha. The Buddha’s perspective is not that we are trying to
wipe it all out and annihilate it, but rather that we are trying to
understand the reason why we take a position or view. We are seeing the
dynamic of that movement of mind that has to have a position of: “I am
\ldots{} I am right \ldots{} I am wrong \ldots{} I am good \ldots{} I am not good \ldots{} I’m
correct \ldots{} I’m wrong again”—all the positions that we take.

The word “conceit” in Buddhism has a whole level of meaning different
from the way we normally use it in the English language. According to
the teachings, there are nine different bases of conceit:

\begin{enumerate}
\def\labelenumi{\arabic{enumi}.}
\item
  being inferior and assuming oneself to be inferior;
\item
  being inferior and assuming oneself to be equal;
\item
  being inferior and assuming oneself to be superior;
\item
  being equal and assuming oneself to be inferior;
\item
  being equal and assuming oneself to be equal;
\item
  being equal and assuming oneself to be superior;
\item
  being superior and assuming oneself to be inferior;
\item
  being superior and assuming oneself to be equal;
\item
  being superior and assuming oneself to be superior.
\end{enumerate}

All of that is the basis of conceit. This is the act of trying to take a
position all the time.

You realize all of that is suffering: trying to figure out “Who am I?
Where am I?” in this realm of relating to myself in the world around me
and in comparison to others. As soon as you go there, you’re trapped.

So again, bring it back to kindness, mettā: “May I abide in well-being;
may all beings abide in well-being.” That is a much safer position,
being able to drop and set aside all the “I”-making and “my”-making.
This “I”-making and “my”-making is constantly being constructed; it
isn’t as if the “I, me, and mine” is related to an actual fixed entity.

This is something we are doing all of the time. That is made really
clear in the Pali words \emph{ahaṅkāra} and \emph{mamaṅkāra}.
\emph{Kāra} is the verb “to do”: the “I-doing” and “mine-doing.” We are
doing it and creating it, although it’s like the sorcerer’s apprentice
in \emph{Fantasia}. The apprentice ends up with a lot of complications
and far more than he bargained for.

The teachings of the Buddha give us the tools to be able to allow that
to be let go of, to be put down, to be dropped. Of course that is a
“doing” as well, but part of the practice is the craft of placing
attention. We need to be doing something. On a certain level, we are
doing something all the time. It’s just the nature of the body and mind.
But we can put attention on the act of relinquishment, the act of
kindness. Those are very conscious acts that reap the fruits of
well-being and peace.

We need to remind ourselves to do this: “One should sustain this
recollection.” It’s reminding ourselves and recollecting mindful
application. As we relinquish, “not holding to fixed views, the pure
hearted one, having clarity of vision, being freed from all sense
desire, is not born again into this world.” Somebody who is freed from
all sense desires and is not born again into this world is an
\emph{anāgāmi} or \emph{arahant}, a non-returner or somebody who is a
fully enlightened being. That’s a real act of kindness, to be somebody
who is fully able to penetrate the teachings and to be completely free
of the entanglements of \emph{saṃsāra}.

Of course, the example closest to me, a contemporary example, is Ajahn
Chah. The Buddha’s example is still resonating after 2,500 years. The
act of relinquishment, being able to be freed from all sense desires,
turning away from the cycle of \emph{saṃsāra}, has had an
extraordinarily powerful effect on the world.

Ajahn Chah set an example of loving-kindness and had the ability to
inspire people to practice. I don’t know exactly how many monasteries in
Thailand are affiliated with Ajahn Chah—several hundred. And, of course,
there is the Western Sangha: America, Canada, England, France,
Switzerland, Germany, Italy, New Zealand, Australia, and Malaysia.
Monasteries and so many other places have grown up just from this small
bullfrog of a man. It’s like he sat on his lily pad in the middle of
this monastery, and everybody came to him.

Ajahn Chah’s loving-kindness was one that came to fruition through
commitment to relinquishment. When he was asked, “What seems to have
made the difference to you?” he said, “Well, I always wanted to find an
end to things.” He wasn’t satisfied until he really found an end.

Ajahn Chah had a lot of difficulty in practice. It wasn’t easy for him.
On a certain level, that probably made him a lot more kind and
compassionate, since he could definitely take in the foibles of the rest
of us. He had the ability to see through. He cultivated loving-kindness,
but it isn’t as if he didn’t have to deal with his own anger, aversion,
or ill will.

There is a nice story about Ajahn Chah. He was sitting and teaching, as
often happened, underneath his dwelling place. A palm reader (an
astrologer) was visiting and was captivated by his humor and warmth. He
finally screwed up the courage to ask Ajahn Chah if he could look at the
lines on his hands. Of course, Ajahn Chah teased him and said, “Well, am
I going to win the lottery?” The palm reader looked at his hands and
said, “Luang Por, the lines in your hands—you’ve got a lot of anger!”
Ajahn Chah smiled and said, “Yes, but I don’t use it.” It’s those simple
words: you don’t negate personality or temperament, but you have the
wisdom and the clarity to say: “I really don’t want to repeat that. Been
there, done that. Time to let it go.”

Another story about Ajahn Chah took place when he had already been abbot
of the monastery for some years. He was a real taskmaster. If you talk
to the really senior monks, you would find out it was not much fun; he
was very strict. One day, after the meal, everyone else had finished and
left. As Ajahn Chah was walking past a novice, he saw the novice
standing, picking up a kettle, and, rather than going to find a cup,
drinking directly from it. Ajahn Chah lost it. In the first place, it is
impolite for a monk or novice to stand and drink. And then to do it by
drinking straight from the kettle was too much.

Ajahn Chah took one look and just bolted after the lad. He was going to
beat this little novice. The novice saw him and was out of there. Ajahn
Chah chased after him, trying to catch the novice to thrash him. The
novice was quicker and got away. Then it hit Ajahn Chah, what he was
doing. He went back to his dwelling place and didn’t come out for three
days. He sat and worked with that anger. He didn’t come out, go on
almsround, or eat. The monks said that after that occasion, nobody ever
saw him display anger again. It was very, very interesting. He just
realized, “This is painful, this is suffering, and I’ve really got to
deal with it.”

Teachers who are wise or compassionate aren’t born like that—that’s not
how it works. You have to work at it; you have to deal with it. There
needs to be the realization: this is suffering and there is a way out of
this. It’s seeing that taking rebirth again and again; even if it’s
pleasant, is not worth it. This is a tremendous gift of kindness to the
world. That’s something that we can be doing, each in our own way. We
can set that intention, that resolution. Mettā-bhāvanā is not an
indulgent way of making yourself feel good for a little bit. It’s
cultivating the roots of your liberation and recognizing what is of
benefit to you and all beings. So I offer that for reflection this
morning.
