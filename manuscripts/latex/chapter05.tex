\chapter{Questions and Answers}

Let’s begin the question-and-answer session. It looks like there is a
veritable blizzard of questions. That’s always encouraging.

\qaspace
Question: Can you speak a little bit about \emph{samatha-vipassanā} and
explain the difference between serenity and equanimity?

\qaspace
Answer: \emph{Samatha-vipassanā} is actually a commentarial division.
\emph{Samatha} is about tranquility, \emph{vipassanā} about insight.
This has a scriptural basis to a certain extent, but as the commentaries
tried to systematize the teachings and methodologies, there was a
division into distinct realms and separate practices. Samatha includes
concentration and tranquility practices. Vipassanā practices are insight
techniques.

I tend to lean much more towards the earliest tradition. The Buddha
himself didn’t draw huge distinctions between the two. Also, my teacher,
Ajahn Chah, didn’t teach in a way that drew large distinctions between
them. Oftentimes he said, “Samatha-vipassanā is kind of like a green
mango and a ripe mango. Same mango, but it’s a part of the process of
ripening.”

The aspects of concentration—settling the mind, unification, creating
tranquility—are part of the same process of ripening. The reflective,
investigative, contemplative aspects of the practice are part of the
same spectrum of practice and training. They depend on each other. A
proficiency in both of these aspects of training is required for the
mind, the heart, to ripen in relinquishment and liberation.

There is a difference between serenity and equanimity. Serenity is often
used as a translation for samatha (tranquility or peaceful abiding of
the mind). This is more the result of a concentration and focusing
practice.

Equanimity is a bit more nuanced in the sense that it is, first, a
brahmavihāra, part of the spectrum of loving-kindness, compassion,
sympathetic joy, and equanimity. These are sublime, lofty states of mind
that result in a very refined state of consciousness. Equanimity is also
an enlightenment factor. The seven factors of enlightenment taught by
the Buddha are mindfulness, reflection on dhamma, effort, joy,
tranquility, concentration, and equanimity. Equanimity is also a factor
of wisdom in the sense of the mind becoming very stable, steady, and
unshakeable because of an abiding involving both emotion and wisdom. So
equanimity is a bit more nuanced. Samatha is just the settling of the
mind, the peacefulness of the mind, a peaceful abiding.

\qaspace
Q: Could you please explain the death process? What happens to us as we
die or as the body dies? How quickly does rebirth occur? Is there
anything in those last moments of that life that affects how we are
reborn: our thoughts, our environment, how we are being cared for, our
mature or not-so mature emotions, and so forth?

\qaspace
A: One of the things that is good to take into account is that when we
think about the death process, we often think, “There I am, in my bed at
home, surrounded by my family and loved ones, and then I mindfully pay
attention to the death process.” In reality, more often what happens is
unpredictable, such as driving down the road at night, all of a sudden
there are headlights coming straight at you, and it’s, “Oh, shit!”
That’s the death process. Or it’s completely out of your control: you
are carted off to the hospital for some reason or other, they fill you
full of tubes, and then you are wondering, “Well, this isn’t how I
wanted it.”

We idealize how we think we are going to die. Death can come quickly,
come slowly, it can come on our terms, it can come not on our terms at
all—more often than not, that is what happens. So, the preparation for
death is reflection and investigation on a day-to-day basis. One of the
chants in the chanting book is called \emph{abhiṇha-paccavekkhana}:
things that are to be reflected upon again and again. The Buddha said we
should be reflecting on these on a daily basis.

“I am of the nature to age; I have not gone beyond aging. I am of the
nature to sicken; I have not gone beyond sickness. I am of the nature to
die; I have not gone beyond dying. All that is mine, beloved, and
pleasing, will become otherwise, will become separated from me. I am the
owner of my \emph{kamma}, heir to my \emph{kamma}, born of my
\emph{kamma}, related to my \emph{kamma}, abide supported by my
\emph{kamma}. Whatever \emph{kamma} I shall do, for good or for ill, of
that I will be the heir.”

The reflections on aging, sickness, death, impermanence, separation, and
being the owner of our \emph{kamma} are preparation for the death
process. It is exceedingly important to challenge our perceptions,
illusions of control, and the assumptions about how long we are going to
live. If we’ve been practicing for a bit, we usually get it that we have
to die at some point in time, but it’s usually “off in the future.”
There are so many different ways of dying.

What happens to us as we die? As we die, there is a breakdown of the
physical body, of the processes that sustain life, and on a certain
level, of the mind itself. Not in the sense of “we all fall apart and go
bonkers,” but there is the breakdown of what we conceive of as “my mind
and my thoughts.” We prop that up through our whole life and identify
with the thoughts, memories, and perceptions. That starts to unravel.

If what you have done is spend your life identifying with thoughts,
feelings, perceptions, and memories, then it’s going to be confusing
because it’s just natural that these unravel. Reflections in the morning
chanting include, “The body is impermanent, feelings are impermanent,
perceptions are impermanent, mental formations are impermanent,
consciousness is impermanent; the body is not self, feelings are not
self, perceptions are not self, mental formations are not self,
consciousness is not self.” These are extraordinarily important
reflections to investigate and to take on board, so that you are not
identifying with and building your home in these impermanent, not-self
phenomena, as is the habit of human beings.

Rebirth generally takes place fairly quickly. The Theravadin doctrine is
that it’s an instantaneous rebirth. Doctrinal positions were generated
historically after the Buddha passed away. The various schools of
Buddhism, such as the Theravadins, had particular positions that they
took, and one of them was this instantaneous rebirth. There is some
scriptural basis for that, but I think one of the wonderful things about
Theravadin scriptures is that they kept everything. As the name itself,
“the way of the elders,” indicates, they tried to keep everything. They
were the conservative element.

So within the scriptures, you can also have things that contradict that.
The \emph{antarābhava}, the in-between becoming, is a state the Buddha
talked about. There is also a discourse in which the Buddha compared the
state to a fire being blown by the wind. The wind picks up the flame and
pushes it somewhere else, which could be near or far away. The Buddha
compared that to the desire, the degree of \emph{taṇhā} and
identification, that impels the mind to rebirth.

In general, I don’t think it’s as neat and tidy as the ancient
Theravadins put it, instantaneous rebirth, or as neat and tidy as the
Tibetans put it, at forty-nine days, or that it’s like this, this, or
this. I don’t think anything is neat and tidy. But there are these
processes by which rebirth is generated. Rebirth is generated by
identification, attachment, state of mind, and the degree of clarity and
calmness within the mind.

That is why within Thai tradition, there is emphasis on people attending
to a person who is dying: supporting them, making sure they’re
comfortable and feeling a sense of ease, physically, emotionally, and
spiritually. Trying to remind them by chanting, speaking to them,
meditating around them, spreading loving-kindness around the sick
bed—these are very supportive ways of creating a wholesome mental state.
It does have an effect when the mind is in a wholesome state and is
clear.

Those transitions are also a time of opportunity as well, of very deep
letting go. “Right, I’ve heard this. ‘I am of the nature to die.’ I’ve
heard this before.” The important thing is then being able to turn to
relinquishment from a place of calm. It’s possible. That is why the
support is there and is culturally encouraged.

In all cultures, there tends to be a washing of the body, a ritual both
for the people who are alive and for the person who has died so that
they can make the transition with peace and relinquishment. This is
important.

There are also those who have built up unskillful habits over a
lifetime. One of Ajahn Chah’s former monks asked him about working with
death and dying, and working with people like that to give them the
opportunity for a wholesome rebirth. Ajahn Chah had his cane beside him
and, poked the former monk in the chest, knocked him over, pushed him on
the ground and said, “\emph{buddho, buddho}.” Ajahn Chah was saying that
the time of death can be painful and very difficult, and if somebody
hasn’t done the work beforehand, it’s not that easy. So it’s important
to prepare throughout our lives. His Holiness the Dalai Lama was once
asked what the basis of his spiritual practice was, and he said, “I
practice dying.”

\qaspace
Q: As a guilt-ridden American, I can’t quite comprehend how to respond
without guilt when I have caused harm, even inadvertently. Could you
walk through the non-guilt-oriented mental process of a person who has
harmed or caused inconvenience to someone else? I’m thinking of the monk
who falsely accused Sāriputta. If that were me, I would practically have
to leave the sangha out of embarrassment and shame.

\qaspace
A: It’s essentially answered in the way the Buddha responded when the
monk made his confession—“I falsely accused Sāriputta”—and then asked
for forgiveness. The Buddha said, “It’s recognizing one’s faults as
faults, one’s unskillful action as unskillful action and wishing to make
amends: that is growth in the Noble One’s Discipline.” That in itself is
the difference. Recognizing that we all make mistakes, whether it’s
intentionally or unintentionally, we have the opportunity to reestablish
ourselves in that which is skillful, appropriate, in accordance with
truth.

Nobody is exempt. Remorse, from a Buddhist perspective, says, “I really
missed my shot on that and that was unskillful,” and there is an impetus
to change. In Pali there is the word \emph{hiri-ottappa}: \emph{hiri} is
a sense of conscience or embarrassment toward an action done in the
past; \emph{ottappa} is a concern for future wrong-doing. It’s
interesting when you say it in English because early translators used
shame and fear of wrongdoing, and what comes up in our minds is that
these are heavy, negative states, whereas in the Pali language, these
are actually wholesome mental states in the way that mental states are
categorized.

In Pali it’s called \emph{kusala}. Intrinsic to \emph{kusala} or a
wholesome mental state is a sense of well-being, clarity, and
discernment. It’s having the clarity to recognize, “That is really
unbecoming and not in accord with how I really want to be, so it would
be better for me to step back from that,” rather than, “Oh, my god, I
did this and what are people going to think of me? I can’t show my face
around here anymore.” That spins around “me”: me and the problems I’ve
created; me and my awfulness. That in itself is the problem.

The recognition of that which is unskillful has a very wholesome,
protective quality. Also, in the Pali language, when that quality of
conscience or concern toward wrongdoing is described, the underlying
state of the mind regarding hiri is one of respect for ourselves, in the
sense that we do not want to do something that would be inappropriate or
not in accord with our values and deepest intentions. Ottappa arises out
of respect for others because we recognize that as human beings, we’re
in the same boat of being subject to birth-aging-sickness-death. We have
a mixture of wholesome and unwholesome qualities. There is a fundamental
respect for other beings and therefore, we don’t want to harm anybody
else either. Both qualities are underpinned by an underlying respect.

The quality of guilt does not give oneself due respect. It undermines
our own wholesome qualities. It’s really interesting to reflect on the
Buddha’s statement that the recognition of our unskillful actions is a
sign of growth in the Noble One’s teaching and training. It’s the
actions that speak louder than the perception of individuality.

That’s what tends to govern our sense of who we are: the perception of
individuality rather than a recollection of the intentions and actions
that we rely on. What do we do on a habitual basis? How do we make our
decisions? There are very few Buddhist meditators whose lives are
committed to stealing, dishonesty, harming, and violence; it’s pretty
rare. But the perception is not in accord with the real actions; rather,
the perception is often driven by guilt.

So it’s important to be able to separate that out, to say, “Well
actually, I’m committed to integrity, to that which is good. I’m going
to shrink back from things that are unskillful.” Also, anybody can make
a mistake and miss their shot. A person has the opportunity to recognize
that and reestablish him or herself in what is skillful.

There are several occasions in the scriptures where we might think that
the Buddha really messed up. It doesn’t come across that the Buddha was
agonized with guilt and struggling. It’s more like, “That didn’t work!
I’ll try it this way instead.” One of the things that is very helpful to
do when recognizing that we have missed our shot is asking for
forgiveness and seeking reconciliation. It’s not that difficult to do,
and it’s very healing.

\qaspace
Q: How would you describe the \emph{jhāna} states, and do you teach this
kind of meditation?

\qaspace
A: That’s pretty extensive. The \emph{jhāna} states are internal states
of stability and composure that are a basis for very bright states of
consciousness. These are states the Buddha encourages because they lead
to an energizing and a clarity of the mind that allows seeing things in
their true nature. They are a very skillful tool.

I tend to follow Ajahn Chah’s way of teaching. Ajahn Chah would describe
the jhāna states as a process of letting go. To the extent that you are
willing to let go, the mind settles very stably and steadily. I think
that is very helpful because oftentimes there can be a tremendous amount
of desire generated for attaining the jhānas and comparing yourself to
those who do attain the jhānas. Maybe you do attain the jhānas, but then
further comparing occurs; it goes on and on. Recognize that the refined
states of consciousness are most easily approached and accessed when we
are able to keep letting go of those aspects of the mind that are
obstructive. As we keep doing that, the mind settles.

\qaspace
Q: When there is a lot of pain in the body, it is difficult to maintain
right effort. Yet sometimes, through patient endurance, the pain
lessens. Can you speak about right effort and the connection between
right effort and \emph{samādhi}?

\qaspace
A: It’s good to refresh our memory as to what right effort actually is.
When we conceive of right effort, we use the word “effort,” and of
course effort is about doing, struggling: getting out there and
energetically being assiduous and diligent, which is sometimes good.

But the way the Buddha defines right effort is effort for decreasing
unwholesome mental states that have arisen. Effort for the preventing of
unwholesome mental states that have yet to arise is also right effort.
Effort for the bringing into being of wholesome states that have not yet
arisen is right effort. And effort for the maintaining of wholesome
states that have arisen is right effort.

It might sound a bit convoluted, but it’s a very clear description, once
you get your mind around it. When the Buddha defines it, it’s all about
actual mental states. right effort is not about how diligent we appear
externally or how much we flog ourselves. It’s about attending to the
quality of the mind—wholesome or unwholesome—and how we relate to the
unwholesome or unskillful states: whether we support them, buy into
them, or exacerbate them, at the expense of undermining the wholesome.

Similarly with wholesome states: how do we encourage those wholesome
mental states, those bright states of mind? How do we maintain them,
without clamping down, shutting down, or worrying? That worrying is in
itself an unwholesome mental state.

In dealing with pain, it’s helpful to pay attention and see the
difference between the physical sensation and the mental elaborations
around the pain. The physical sensation of pain is oftentimes telling us
something like, “I’m tired. I want out of here.” Something is stressed
and pain is a warning, trying to get our attention. There are many
levels to the physical side: some of it is bearable; some of it may not
be necessary to bear and needs to be adjusted, dealt with, or
ameliorated in some way.

On the mental level, what is the reaction, what is being added, what is
the level of fear or aversion? Pain can be very threatening. It can also
be very useful. It definitely gets our attention. There is an immediacy
there, so we can even use pain to challenge the mind and the assumptions
that we have, bringing a certain urgency to practice.

There’s no single response. We need to be balanced. We need to see what
is a genuine call of distress that needs to be heeded and what we can
just be patient with, bearing with the pain and directing attention to
the mind itself. Sometimes with pain we can simply direct attention to
the physical sensation of the pain. Sometimes we can absorb into the
pain and the mind can be very, very peaceful. We can be inside the
sensation and find that the mind becomes very cool and bright.

There are many ways of responding. It’s a very interesting aspect of the
practice. Having physical bodies, we are always going to experience some
sort of pain. We can learn a lot about how much we expect, how much we
push, or how much we worry or fear, through seeing our reactions to
pain. It’s something that is very natural. Having pain is a natural part
of the human condition, and it’s important to learn how to deal with it
skillfully.

I think one of the main things is gaining confidence that pain can be
worked with and that there is a point at which the appropriate way of
working with it is to back off, rest, or do something about it. At other
times, we can just go into the pain. For many years of my early monastic
life, pain, illness, and injury were fairly constant features of my
meditation. I had the opportunity and had to use it. I had to become
familiar with pain, and I find that that has held me in good stead.

\qaspace
Q: What is loving-kindness? What is the body and mind’s experience when
I feel mettā for myself and others?

\qaspace
A: I think everybody is going to experience it in a slightly different
way. There is not a formulaic way that you should experience
loving-kindness. There is a general parameter at the base of
loving-kindness, which is dwelling in non-aversion, to put it in
negative terms.

Oftentimes Western culture casts things in idealistic ways. We are very
good at ideals: ideally, what would be the perfect expression of
loving-kindness? That’s what we try to get, and we can drive ourselves
nuts doing this.

I think it’s helpful to begin with a less idealistic basis: dwelling in
non-aversion. It’s the absence of aversion, of ill will. Recognizing
that, we have something fairly tangible to work with. There is an
experience and a recognition because we don’t dwell in aversion all the
time, do we? We don’t spend our whole lives in ill will and grumpiness
towards the whole world.

We then build on that in a more positive sense. When it’s translated
into the Thai language, it is expressed as a well-wishing, wishing
oneself and others well. There is a recognition of what we would
appreciate, of what others would appreciate: that sense of well-being. I
think that is a very useful perception of loving-kindness.

Sometimes the English language is loaded with our cultural baggage: “How
can I be loving towards myself and all living beings? I don't love
everybody.” Well, sometimes you don’t even like them, but you can wish
them well. You can establish that wish for your own well-being and for
others’ well-being. You don’t even have to like people to do that. I
think that is casting it into a light that is practical and realistic.

Sometimes it’s good to use words that are unfamiliar to us. Bhante
Guṇaratana plays with language: loving friendliness or an appreciative
friendliness. It’s not a word that we use all the time, but it’s trying
to convey a particular feeling. That kind of well-wishing or
appreciative friendliness is the mental state. Internally, within the
mind, there is a warmth, spaciousness, non-contention, a concern for
others’ well-being.

Then, of course, there will be a physical relaxation. Aversion, ill
will, contention—these have their attendant feelings of agitation and
tightening within the body. Physically, with loving-kindness, we will
feel a sense of ease, spaciousness, and settledness. Within the heart
there will be a similar sense of openness. There will be space for
yourself and for others.

So these are some reflections. I enjoy the questions. I never quite know
what is going to come out when people present a question. It’s
interesting to me as well.
