\chapter{Questions and Answers}

\qaspace
Question: Realized beings abound these days. Care to comment? When the
conditions are conducive and the inquiry is in earnest, is it possible
to “wake up” quickly?

\qaspace
Answer: I remember when Ajahn Chah came to America to teach. Because he
had done some traveling and had taught in a few different places, he was
asked, “Now that you have been in America for a while, how do you feel
about Buddhism coming to America?” Ajahn Chah had this blank look on his
face and said that Buddhism hadn’t come to America yet: “I haven’t seen
any Buddhism yet.” Ajahn Chah could be naughty sometimes.

Certainly, when the conditions are conducive and the inquiry is in
earnest, it is possible to wake up quickly, but finding conducive
conditions and having that earnest inquiry is not that easy. You really
have to sustain inquiry and be focused, which is not that easy.

Much of Ajahn Chah’s teaching is around endurance because that is what
is required. We all tend to set the intention to wake up quickly when we
first start practicing. I ordained with the idea that I would take a
temporary ordination, learn about meditation (I had three or four spare
months) and achieve stream-entry at least. Then I could go do what I
wanted. I wouldn’t have to be reborn in the lower realms again. It’s not
quite that easy, but certainly, attention to conducive conditions and
the earnestness of inquiry are really important.

\qaspace
Q: Which suttas describe path and fruit?

\qaspace
A: There are two different modalities. One is the path as I described it
the other night. There is also another kind of modality, a seven-factor
description of path and realization. That is in \emph{Majjhima Nikāya}
70, the \emph{Kīṭāgiri Sutta}. Path and fruit are fairly ubiquitous, but
I don’t have a good reference off the top of my head.

\qaspace
Q: Would you say a bit about the benefits of dwelling in loving-kindness
during the dying process? Both for the caregiver and the person dying.

\qaspace
A: The immediate benefit for the person dying is that the mind is not
drawn to negative mental states. That eases the whole process of dying.
There is physical ease and the diminution of fear. One is able to pass
away and let the body go with clarity because it is held within the
sphere of loving-kindness. That’s a tremendous benefit.

For the caregiver, there is the ability to retain and dwell in the
perspectives of loving-kindness. It is very tiring looking after a dying
body, so there are endless amounts of frustration and uncertainty.
Dwelling in loving-kindness is a way to ameliorate some of the
frustrations, worries, fears, and projections that come. Sometimes we
overextend ourselves, trying to be the ideal caregiver. To be able to
say with loving-kindness, “I’m really fried. You need to ask somebody
else to help out,” is itself an act of loving-kindness.

These bodies are high maintenance when they don’t work. In the last nine
years of Ajahn Chah’s life, he was unable to look after himself. He was
bedridden and unable to speak or use his limbs. It was a wonderful
opportunity for the monks to be able to look after him. They set things
up extraordinarily well. There would be four monks and a novice in the
daytime and four monks and a novice at night, fully looking after all of
his needs. There would be a very experienced monk as a head caregiver
overseeing everything.

Every two weeks there would be a change. People would come to Ajahn
Chah’s monastery and wait in line to care for him. They wouldn’t do
anything for Ajahn Chah that they hadn’t become proficient at. As it got
to the later stages, with nasal feeding tubes, monks had to train and
practice on themselves before they were allowed to get close to Ajahn
Chah. He was bedridden for nine years and never had a bedsore, which is
quite extraordinary. That level of care was done out of gratitude and
loving-kindness. It was a real honor to be able to look after him.

But most people do not have a huge sangha of monks. It’s hard work to
look after bodies that are breaking down. One of the uses of
loving-kindness is being realistic as to how we need to be kind to
ourselves and draw other people in as well. Often we feel that we should
be doing everything. It’s like the horse in \emph{Animal Farm}, “I will
work harder. I will work harder.” And of course the pigs keep piling on
the work. The horse collapses, dies, and is shipped off to the glue
factory. It is really important to have loving-kindness about our limits
and to find other resources.

\qaspace
Q: If everything about me, my body, thoughts, feelings, and
consciousness is impermanent, and even if my “self” is illusionary, whom
or what am I addressing when I wish myself well-being?

\qaspace
A: It isn’t as if there is no one home. This conventional self is a
focal point, has taken rebirth, has a mind, and has results ripening
from the past. These aren’t illusory in the sense that they have no
meaning or effect. We experience and feel those results. Part of the
means of disentangling ourselves and ceasing to suffer from the
complications that we make of that self is loving-kindness. It is one of
the skillful means for amelioration.

The negation of self is just as much a wrong view as the full-blown
belief in self. There is a narrow balance that we need to make in order
to understand the teachings. This is not a nihilistic teaching. That is
one of the fundamental wrong views that the Buddha points to as a source
of suffering: being caught up in the self and what we perceive as self,
but also the negation and the tendency to annihilation of self. That is
also a wrong view that leads to suffering.

There is a very real cause-and-effect process going on. We feel and
experience the results of actions and the skillful application of that
which is positive, wholesome, and leads to peace, clarity, and
well-being. These are exceedingly useful.

\qaspace
Q: For most of the time on this retreat I have had a lot of thoughts and
worries about a problem that I have been dealing with in daily life and
which I will have to continue to deal with after the retreat. I’ve tried
to work skillfully with it, extending mettā to myself and the others
involved and doing mindfulness practice, working with anxiety, anger,
aversion, playing out all the possible scenarios, and returning to the
body. This only seems to intensify things and feels obsessional. I
recognize this pattern from numerous prior retreats, although the
thematic content changes. I’m feeling discouraged. Any suggestions?

\qaspace
A: I think one of the things is recognizing that this is the way the
mind tends to go. It grabs a particular scenario or difficulty and then
obsesses on it. There is anxiety and fear around it. It’s good to be
able to say, “Okay, that’s a reasonably well-established pattern,” and
be able to recognize that. It’s better to approach it by asking, “Can I
have loving-kindness for that tendency to follow and be in that
pattern?” rather than trying to think, “I need loving-kindness for
myself; I need loving-kindness for that person,” and parcel it all out.

The ideal is to bring a sense of kindness and well-wishing to that
pattern and see how the mind loops into those reactions. It helps to
give it some space, seeing the particular habit or pattern, and then
stepping back from the identification with it and recognizing it is an
object worthy of well-wishing.

When we relate to ourselves, our habit patterns, our temperament and
personality, we’re often not extraordinarily kind. We tend to have
higher expectations and a lower level of patience. If you were your
close friend, would you respond in the same way? Take the same scenario,
but place a good friend in the spot where you perceive yourself. With
that same scenario and circumstances, what would be your response?

You would find that the response would be almost invariably much kinder
and more patient, spacious, forgiving, and generous. That’s a very
helpful exercise in imagination or visualization: just to replace
yourself with somebody else. Recognize that you would probably respond
in a completely different way.

\qaspace
Q: Was Ajahn Chah an \emph{anāgāmi} (non-returner) when he became angry
with the young monk? I thought the root of anger is uprooted at that
stage. Please explain.

\qaspace
A: Absolutely not, with that kind of anger. An anāgāmi has absolutely
eradicated anger and sensual desire. That was the sign that he really
missed his shot and still had some major defilements to work with, and
that’s what he put his energy into working with.

\qaspace
Q: Would you explain, in English, the Pali blessings you chant after the
meal offering? I can’t remember the whole translation but remember that
it is very beautiful and complements the concept of mettā.

\qaspace
A: I can’t remember it either, truth be told. One of the reasons that we
don’t do it in English is that we haven’t got a really good translation
of it. We have tried on various occasions. It is either hokey or clunky,
so we haven’t been able to get it into English. But essentially, the
chant is a recognition of the goodness that has been done. One of the
images is that as water wells up and overflows, goodness spreads forth.
Recognize the goodness that is done, and, by the power of that goodness,
may you be free from difficulty and illness. May you receive the
blessings of goodness. Through the force of the goodness that has been
done, may you experience good health, long life, happiness, and
well-being.

The last section of the chant is: “By the power of the Buddha, by the
power of the Dhamma, by the power of the Sangha, may the devas protect
you by the forces of goodness, and may you live happily.”

So it is a very beautiful chant, and it is unfortunate that we still
haven’t got it together yet in English. They are called anumodana
chants, and there are a few different ones. “Anumodana” literally means
delighting in the goodness that has been done. So that’s the gist of it.

\qaspace
Q: Could you please talk about the difference between awareness and
consciousness? Is awareness a conditioned state, the way consciousness
is? How can we be aware if not through consciousness?

\qaspace
A: Exactly. You can’t really separate out the khandhas, the five
aggregates of being. You can by using language and concepts, but, in
reality, they’re tied together, rely on each other, and depend on each
other. The difference between awareness and consciousness is that
consciousness is the basic function of being able to be aware through
the sense doors: eye consciousness, ear consciousness, nose
consciousness, tongue consciousness, body consciousness, and mind
consciousness. The ability to be conscious of sensory contact—that’s a
basic function. If you live, exist, and have a body and a mind, then you
have a basic function of consciousness.

Awareness is a function of \emph{saṅkhārā}, mental formations. The
distinguishing characteristic of saṅkhārā, is \emph{cetanā} (volition or
intention), so that we are directing the awareness. The ability to
direct awareness and bring awareness into being is a function of the
application of volition and attention. So we can become skilled in the
use of awareness, or we can be forgetful in the use of awareness. This
needs to be nurtured, looked after, and developed. Consciousness,
receiving sensory contact, is just a basic function.

That’s the basic difference. They’re both conditioned. We direct
awareness in consciousness as well. By becoming increasingly aware of
what we are conscious of, we can make clearer decisions and more clearly
discern what is actually happening to us.

\qaspace
Q: Would you be willing to share some of your personal journey,
including some description of your life before you became a monk? Why
did you choose to become a monk? Could you shed light on how the holy
life can help people grow and change?

\qaspace
A: I have a very short life history. I grew up in a podunk little town
in Canada and got out of there as soon as I could. I became interested
in Buddhism in university in the 1960s. I grew up in the province of
Manitoba. The town I grew up in is six-hundred miles north of the US
border. There wasn’t much Buddhism happening up there then, and there
still isn’t. Then I went to university in Winnipeg, which is a nice
city, but small, in midwestern Canada. So I didn’t come into contact
with any teachings or teachers.

But I had an interest. I tracked down as many books as I could on
Buddhism. And then after finishing university and working long enough to
get some money, I headed off traveling. Many people did in those days. I
traveled through Europe, the Middle East, and India. I had a vague idea
that I would go to Japan to study Buddhism. In those days, the majority
of the books you could get on Buddhism were about Zen.

But then I arrived in Thailand. There were all these monasteries around.
It looked like it was a Buddhist culture. I was really fascinated, so I
stayed on. I went to a few different monasteries and practiced here and
there. My introduction to meditation was a solitary, one-month Mahasi
Sayadaw retreat. That was how I started to learn to meditate, and I
loved it. I wondered, “How do I do more of this?”

It was a bit difficult because they tell you to meditate twenty hours a
day and sleep four. And given the instructions, I felt four hours
couldn’t be real, so I just disregarded that. It just didn’t sink in; it
wasn’t a part of my reality.

I was meditating diligently, as much as I could, and having some very
interesting states. Then there was a bit of a plateau after about ten
days or two weeks. Then after a couple of days, the teacher asked, “How
much are you sleeping?” I said, “Maybe six or seven hours a night.” I
thought that was pretty good because I could see my sleep dropping. And
he chewed me out: “You slacker! You’ve been cheating. You’ve built up
all this sleep. Two hours a night! That’s all you can have. Go back and
meditate.” So I had lots of interesting experience with meditation.

I went to another monastery and took temporary ordination. It was a city
monastery on the outskirts of Bangkok. That’s where I started hearing
about the forest tradition and Ajahn Chah, and that sounded really
attractive to me. After I’d been a monk for only a month, I asked my
teacher if I could go up to visit and pay my respects to Ajahn Chah. He
said that he had heard about him and that he was a very good monk.

So I left and went to Ajahn Chah’s monastery in the early morning. You
see monks coming back from almsrounds, winter mists in the fields. It’s
very ethereal. I came in as everything was quiet and composed and got
the opportunity to pay respects to Ajahn Chah. So I bowed and finished
my bowing, and he looked at me and said, “If you want to stay here, you
have to stay at least five years.” So I felt, “Oh, I guess it’s not for
me.”

I stayed for a while and then went off to another monastery that was
very quiet. I was putting a lot of effort into practice. Meditation was
going very well. I was very diligent in practice. The months were going
by, but I just kept thinking of Ajahn Chah all the time. I thought,
“Well, five years is five years. I’ll go back and give myself to Ajahn
Chah.” That’s thirty-five years ago.

In terms of how the holy life can help people grow and change,
especially having been in robes for such a long time and been in
monasteries where people come and go, one thing that is clear to me is
there is no such thing as the ideal monastic or ideal practitioner. I’ve
seen so many types of people come. Some people stay and some people go.
Some people are absolutely inspired and seem so composed, knowledgeable,
and clear, and then they disappear in the middle of the night. Other
people come, and, despite problem after problem and difficulty after
difficulty, they stick with it and they change. That is one of the
things that is consistent: if somebody stays with it, change takes
place. People really do grow and change.

\qaspace
Q: It has been so helpful to hear stories from your own experience.
Could you talk about some of the more challenging moments in your
practice and how you worked with them?

\qaspace
A: I’ve shared a few things, especially in terms of conflicts with
others. One of the things that might be useful is dealing with doubt.
That is something that we all experience: doubt in ourselves, doubt in
the practice. One thing that I have found interesting is that it isn’t
actually me resolving doubt. It is allowing the practice or Dhamma to
work. It’s like getting on the bus: it will take us to where its
destination is. We don’t really need to figure everything out or have it
all absolutely clear, but there is a basic foundation of trust in the
teachings and practice. We give ourselves to that, the vehicle of the
Buddha, Dhamma, and Sangha, and see what the results are.

Those qualities of generosity, virtue, mindfulness, training, and
reflection on basic themes are a trustworthy vehicle. Just plug into
that, getting into the vehicle and allowing it to carry you. After some
time these qualities start to mature and, as you stick with the
practice, there is a realization. It isn’t as if you had an insight into
the doubt and it dissolved, but at some point you say, “I used to have
doubt about that. I used to worry about that. That’s not there anymore.
That’s interesting.” On reflection, you see that the doubt has unraveled
because of the power of the practice and path itself. Confidence in the
efficacy of the path and training grows and continues to grow.

\qaspace
Q: What is the Pali word for letting go or relinquishment? Please spell
that as well.

\qaspace
A: Well there are a few different words that sometimes get tied into the
phrase “letting go.” \emph{Abhinivesāya}: all \emph{dhammas} are not to
be clung to. Not clinging is, of course, letting go. \emph{Paṭinissagga}
directly means relinquishment or letting go. \emph{Vossagga} is
relinquishment, particularly around the sense of self—letting go of the
sense of self or “I.”

This comes up in many suttas in which the Buddha outlines a course of
practice. Each step of the practice is accompanied by reflection on
impermanence or solitude, and then dispassion and cessation, culminating
in relinquishment. The cultivation of \emph{ānāpānasati}, mindfulness of
breathing, culminates in reflections on impermanence, cessation, and
relinquishment, which are important terms for reflection.

\qaspace
Q: Is this relinquishment, letting go, the opposite of \emph{upādāna}?

\qaspace
A: Yes, it would be the opposite of that clinging.

\qaspace
Q: Is \emph{attavādupādāna} clinging to \emph{sakkāya-diṭṭhi}?

\qaspace
A: There are four different types of upādāna. The last one is
\emph{attavādupādāna}, which is the clinging to the belief in self.
\emph{Sakkāya-diṭṭhi} is one of the fetters; it is the identification
with the body-mind complex. Attavādupādāna is much broader. It would
still play itself out in the more refined types of clinging to self and
self-conceit. It forms the basis of sakkāya-diṭṭhi but is broader. The
difference between attavādupādāna and \emph{diṭṭhupādāna} is that
\emph{diṭṭhupādāna} is clinging to views, while attavādupādāna is the
clinging to belief in self. It is interesting that the structure of the
language is not “clinging to self” because the self is non-existent.
Rather, it is the clinging to the belief in self.

\qaspace
Q: Do you think being interested in Buddhism guided you to the
contemplative life or would you have become a monk in any tradition?
Would you share a bit of your story?

\qaspace
A: Certainly there was a draw to some kind of contemplative life. When I
was young, religion didn’t draw me, although that is probably a
manifestation of stubbornness and contrariness. But something resonated
immediately with Buddhism and something drew me to meditation and
training the mind. I was looking for ways to do that.

I remember when I was still in university. I was out east and visiting
an uncle and aunt of mine. I was already interested in Buddhism, and we
were talking and something came up about religion, Buddhism, and
meditation. My aunt said that she remembered looking after me when I was
a small child—she had asked me what I wanted to be when I grew up, and I
said I wanted to be a priest. I remember being very embarrassed. I had
another friend traveling with me, and I remember thinking, “Oh, dear.”
But I can see that something resonated, even if it wasn’t conscious.
Something was definitely drawing me to some kind of spiritual tradition
and training.

\qaspace
Q: Could you say a few more words on posture? For example, I noticed
that my body was leaning towards the left, and if I weighted my right
hand with intention, this seemed to stop. Is this a correct tactic? In
my marital arts training the goal is to relax and make the breathing
easier. Is that true of \emph{vipassanā} as well?

\qaspace
A: I haven’t addressed posture very much on this retreat; I somehow
skipped over it. Posture is a very helpful anchor for awareness: being
able to attend to posture, paying attention to a balance of posture and
the energy we bring into the posture. It is very helpful in terms of the
practice: recognizing where our bodies either slump, lean, or drift. It
is helpful for balancing energy within the body-mind complex, but you
also start to get an inkling of what’s going on in your body-mind.

What is important is relaxing into the posture, and then allowing the
breathing to flow quite easily. If you’re sitting slumped over, it’s
very difficult to breathe. The mind tends to get dull; the body will
tend to ache. If there is a nice, balanced posture, the body can sit
still for longer periods of time.

You can’t force the body into sitting still. Sitting straight up, you
just get tense, but if you continue to relax, allow the breathing to
soften and balance, and let the posture balance the energy, then you
find that there is a lightness within the body and mind.

One of the things I try to draw attention to is that whenever you go
into a shrine room where there is a Buddha image, you almost always see
that it has a very nice posture. It is the archetype of balance and
composure.

I remember the very first time that Ajahn Sumedho came back from
England. He had gone to England to establish a monastery and had been
there for several years. He came back to visit and pay his respects to
Ajahn Chah. One of the gifts he brought was a Buddha image that had been
sculpted by an English student there. The image was very upright. Ajahn
Chah saw it, was quiet for a while, and looked at it from time to time.
In the Thai language, the word for “westerner” is \emph{farang}. Finally
he said, “It looks like a farang Buddha. It looks very tense.”

\qaspace
Q: Could you please explain devas to us? I had an experience yesterday
after the Dhamma talk in which I may have seen a group of beings above
us, particularly above you. This is not the first experience I’ve had
with perceiving other-worldly beings, but this is the first time new
ones communicated to me. They seemed to indicate that they particularly
wanted me to be free from anxiety. Today, they also led me to radiate
compassion using the phrases from the second stanza of the
sublime-abiding chant that we did this afternoon. Once I focused on this
a while, they seemed calmer, and I no longer felt their presence. I
don’t think I’m crazy, but I’m very interested in what the Dhamma says
about other-worldly, non-material beings.

\qaspace
A: Unfortunately, I’m one of the last people to talk about any direct
experience with this. I don’t see other realms or other beings. When I
used to live at Wat Nanachat, there was a cremation ground, so there
were many ghosts as well as many devas. On more than one occasion,
either laypeople or a monks good at this would come up to me and say,
“Such and such deva in that place and that spirit in that place would
like you to dedicate merit and offer blessings.” They always had to use
an intermediary. I was thick, thick, thick. I don’t get it.

My own experience, having lived in Thailand a long time, a place that
has a totally different world view, is that other people whom I trusted
had these experiences. It certainly seems that they are just another
part of reality.

That being said, it can also be a sign of psychosis, so it can be a
delicate balance. Once you meet someone who is communicating with other
beings in other realms, you might need to call in a psychiatrist or a
psychotherapist. So it does call for circumspection, but I have met many
people who are very stable and sane, who say there are realms of devas,
particularly in a place like this, which is a natural environment. In a
place that’s protected, there will be many such beings, tree devas,
forest devas, and earth devas, and they would particularly be drawn to
loving-kindness. It draws a crowd; it feels good. This is a natural
phenomenon. I see it as a part of nature.

One of the monks we had visiting us last year, Luang Por Plien, has a
very tangible connection with other realms, which came in very useful.
Casa Serena at Abhayagiri, the women’s residence and place of practice,
was offered by someone who had passed away in the house itself. Many of
the Thai women who stay there are frightened that there might be ghosts
there, so when he came to Abhayagiri he was asked, “Luang Por Plien,
please come over to Casa Serena, sit in meditation, and give me the
straight scoop. Are there any ghosts here?” We went over and did
blessing chants, and he sat and later said, “There are no ghosts here.
They were reborn immediately by the force of their goodness. There is
nothing but good spirits here.” Not having direct experience, being able
to rely on Ajahn Plien was very helpful.

\qaspace
Q: What is the difference between meditating on and contemplating or
thinking about something? Could you give some examples about someone who
would be skillfully meditating on something versus unskillfully doing
so? And what does \emph{saṅkhāra} mean?

\qaspace
A: “Meditating” is a generic term. “Contemplating” definitely has the
sense of reflection, investigation, and thinking. Meditation can use the
thought process, but also, the thought process can be still. We use
contemplation and reflection to help the meditation process, either in
the sense of bringing up a sense of urgency or bringing up positive
states of mind, but in the end, meditation, ideally, turns into a
stillness, a silence, a freedom from the thought process.

This doesn’t mean that we aren’t able to discern anything, but that
restless thinking, that compulsion to ruminate, settles, and the mind
becomes still and clear. There’s a stability and clarity that allows us
to see things in a different light, the light of relinquishment, the
light of passing away or ceasing, recognizing things coming to
cessation, settling.

So, we aren’t thinking about it. It’s a very direct experience because
the mind is still. That would be an example of skillfully meditating. It
would be unskillful to obsess on a particular thought process or
experience. Even if an experience is one of peace and stillness, one can
still obsess on it. There can be assumptions of selfhood or the
extrapolation of one’s worth and value, like one has a good meditation,
so one feels one is a good person. Or one has had a bad meditation, and
therefore, one’s a bad meditator or bad person. That’s unskillful.
Recognize that that is just an impermanent experience, arising and
ceasing, so there is space around it.

\emph{Saṅkhāra} probably has one of the longest definitions in the
Pali-English dictionary. It is a complicated word and concept, and it’s
used in different contexts. In a literal sense, it is something that is
built, compounded, or created. So in the chant that we do, \emph{sabbe
saṅkhārā aniccā} means, “all things of a compounded nature are
impermanent.” But the phrase is also used in \emph{rūpaṃ aniccaṃ},
\emph{vedanā aniccā}, and \emph{saññā aniccā}, as well as \emph{saṅkhārā
aniccā}. It’s used in the context of the khandhas, the aggregates of
being. And there is also a specific meaning of \emph{saṅkhāra} that
refers to mental states, mental formations, which are the compounded and
compounding, volitional aspect of mental activities.

\qaspace
Q: When I think of people with spontaneous, open, and generous hearts,
they are full of mettā, \emph{karuṇā}, and muditā but do not seem
equanimous. Can upekkhā come naturally, as well as be cultivated?

\qaspace
A: I think it really is more a result of cultivation because it isn’t
immediately apparent in human experience that equanimity has value. We
can see the value in loving-kindness, compassion, and sympathetic joy
(the freedom from jealousy), but it isn’t immediately apparent what is
good about equanimity.

The others are more bound with becoming and all the positive
connotations we have associated with a good human being. We don’t
appreciate the sense of upekkhā, the equanimity, equipoise, and balance,
until we start to reflect and investigate and see how difficult it
actually is to sustain that: to be present with something, not moved by
circumstance, not moved internally with the moods or impressions within
the mind.

When we say not moved, it doesn’t mean dull, shut down, or closed off,
but being completely in tune, very clear, and not being shaken by
anything. It takes some time to appreciate that and then recognize that
this is going to take some cultivation and work. Of course, when we
recognize the benefits of not being dragged into our own internal or
external reactivity or the expectations of the world around us, there is
tremendous freedom in that quality of equanimity.

As we cultivate and attend to the qualities of mettā, karuṇā, and
muditā, as we become more attuned to the sublime nature of these
wholesome qualities, then we also start to recognize the nature of the
mind, in the sense that our desires are always outstripping our
experience. So, you get your latest gadget and it has more memory and
it’s faster: “I’d like one of those.”

We do the same things in spiritual practice. It can torment us, but it
is also quite natural. When it’s channeled, we can try to put it to
skillful use. As we cultivate loving-kindness and compassion, we can
think, “Well that’s pretty good, but I bet there is something more.”
Well, in fact, there is. Equanimity is the refinement that allows us to
be drawn towards true peace.

\qaspace
Q: What is the difference between \emph{pīti} and sukha?

\qaspace
A: These are qualities that usually arise in meditation as we develop a
continuity of awareness and the mind becomes more and more settled,
composed, and unified. The description of the first \emph{jhāna}, or
stage of absorption, includes \emph{vitakka-vicāra}, or directed thought
and evaluation, \emph{pīti}, sukha, and \emph{ekaggatā}, which is
unification of mind.

\emph{Pīti} is usually translated as joy and sukha as happiness. The
classic description from the \emph{Visuddhimagga} is a very good
illustration of the difference between the two. The image is of a person
walking through an area of wilderness with no source of water or shade,
so that there is an experience of heat, discomfort, and suffering. As
you continue on your journey, you see some people coming. It looks like
they are freshly washed. They look refreshed, and their clothes are
clean. They say, “Well, just over there, you can see on the horizon an
area of forest where there are trees. There is water there. You will be
able to bathe and drink and rest as much as you like.” The feeling that
comes up is a sense of joy.

You continue the journey and arrive at that oasis or forest area. It is
cool. It is shaded. There is water there. You are able to bathe, refresh
yourself, and drink your fill. The feeling that arises then is sukha or
happiness. So the joy is in the expectation and the happiness is in the
experience.

So, similarly in practice, as we begin to have a continuity of awareness
and attention, the mind becomes more settled and we start to have a
joyful energy coming through the body. The feeling of joyful energy that
we experience is pīti, which can be experienced in different ways.
Sometimes when you are sitting and meditating and the mind is settling,
there might be some tingling and a rushing feeling in the body. The
breathing is deep and the body is experiencing a sense of joy, but it’s
not so settled.

With sukha, we continue with that attentiveness and mindfulness, and as
the mind is not shaken or distracted by the joy, the happiness becomes
much more pervasive throughout the body. The mind and body become
relaxed and settled. Feelings of pain and discomfort are completely
gone, and there is a feeling of well-being.

So those are the differences between pīti and sukha. Pīti can be a bit
exciting. Sometimes you are meditating, reflecting on some aspect of
Dhamma, and tears of joy come. It’s hard to get peaceful and settled
with tears running down your face: you need to blow your nose.

Keep being mindful and it settles into a steadier sukha, happiness,
which is very pervasive. Do not try to control or force things at this
point. Allow the mind to move into the object of attention. There is a
unification of mind.

Pīti and sukha are very helpful tools in practice, but they are also a
natural result. People experience them in different ways. If you create
an idea or concept of it, you will almost certainly block it off. The
most important aspect of practice is mindfulness; that’s really the
bedrock or bottom line. Keep coming back to the quality of mindful
attention and awareness.
