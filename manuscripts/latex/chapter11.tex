\chapter{Questions and Answers}

So, we’ll begin this evening’s question period. Nature abhors a vacuum
and we managed to get through all the submitted questions last night.
The basket is full again, which is a good sign.

\qaspace
Question: One of my lay insight meditation teachers said, “The Western
lay practitioner is an experiment in Buddhism.” What do you think? To me
it seems our lay teachers are also an experiment.

\qaspace
Answer: There seems to be some bias there. Certainly, in the West so
far, the predominant mode is lay practice and lay teachers. Although
there’s an increasing amount of interest in people wanting to study with
monastic teachers, and there’s certainly a strong interest in monastic
training, the monastic presence is still pretty tiny, really.

At Abhayagiri, we almost always have a waiting list for people who want
to come and train.

The classic model is for the monastic sangha to be laying the foundation
for teaching, leadership and being an example. I think oftentimes there
is a false dichotomy perceived, from the West, because the monastic
presence is so strong in Asia. Oftentimes, people overlook the strength
of the tradition of lay practice and lay teachers in Asia. Certainly in
Thailand, that is very much the case. There are laymen and women who are
excellent scholars, teachers. It’s not as monochromatic as people
perceive.

Look at the model that the Buddha himself gave. There’s a very
interesting discourse where Māra comes, right after the Buddha is
enlightened, and says, “You’ve done your work; it's time to pack it in.
You can die now.” And the Buddha says, “It is my function and purpose to
teach and I won’t pass away until the monks, nuns, laymen, and laywomen
are well established in the path and the fruits, able to give leadership
and solve problems, able to stand on their own, so this Dhamma and
Discipline can last for a long time.” That was the Buddha’s standard of
what was a balanced, strong dispensation: when monks, nuns, laymen, and
laywomen were all knowledgeable and well-practiced.

\qaspace
Q: This is a common scenario: I’m caught in a story of praise and blame.
I notice a voice says, “That was very quick. You’re getting good at
this.” I wake up again. “Ah, I know you, Māra.” A voice says,
“Excellent, what a skillful yogi you’re becoming.” Māra seems to co-opt
every moment of awakening to feed the ego. Do I patiently endure, and
this will dissipate with time and practice, or is there something you
can suggest?

\qaspace
A: Well, it’s both, in the sense that it’s really hard to overestimate
how important patient endurance is to the practice – the willingness to
bear with the pathetic nature of the mind. But then there are also those
experiments, those reflections we use in undermining the perceptions of
self and ego – the recollection of impermanence, uncertainty, and
change. “These are mental states arising and ceasing.” That’s tuning
into it as \emph{dhammas} arising and ceasing, as opposed to me
struggling with Māra, me being the champion over Māra, me being defeated
by Māra again: that “me and Māra” story.

So, seeing \emph{dhammas} arising and ceasing, that’s all that is
needed. There isn’t a “me” in that. It’s a really important aspect of
insight to be turning attention towards. Also, we can develop the
deconstructing of the process of how thoughts arise and cease: getting
to know the sense of feeling, perception, mental formation,
consciousness, desire, attachment, and clinging, the process of
Dependent Origination –“It’s just these five khandhas that are arising
and ceasing”– and being able to rest in the awareness of those five
khandhas, as opposed to buying into the story of “me and my khandhas.”

\qaspace
Q: The loving-kindness chant includes “May I abide in freedom from
affliction.” Why is affliction not included in the wish for all beings,
while hostility, ill-will and anxiety are in both?

\qaspace
A: The reason is because in that chant it’s actually just a little bit
further down. I can’t remember how it’s translated, but that particular
chant is the chant for all the brahmavihāras: loving-kindness,
compassion, sympathetic joy, and equanimity. The wish “May all beings be
free from affliction” is an expression of compassion and is actually
worded, “May all beings be released from all suffering.” That is how it
is expressed, because that is the fundamental expression of compassion.
The loving-kindness aspiration is the wish for happiness and well-being
and the expression of compassion is the wish for freedom from suffering.
For some reason, in that chant, it is folded in with loving-kindness.
Sometimes it’s difficult to find a hard and fast distinction between the
two. In order to experience well-being, then of course one needs to be
free from suffering.

\qaspace
Q: This morning you spoke about bringing to mind the mettā-nimitta, the
mettā-object. I have some understanding of the feeling of mettā, but no
understanding of the mettā-nimitta. Could you explain more fully?

\qaspace
A: It’s exactly the same. \emph{Nimitta} is more a generic term for the
object. We often associate \emph{nimitta} with some type of light or
image. With mindfulness of breathing and concentration, there would be
encouragement to develop a sign of light or imagery. But the actual word
\emph{nimitta} just means sign. So the sign of metta is the feeling, the
actual feeling, emotion within the heart, within experience. That is the
nimitta.

It is important to attend to that. If one is looking for something else,
or just repeating the phrases, thinking, “This is the mettā-nimitta,” or
trying to conjure up an image of oneself being suffused with
loving-kindness somehow, one can get off track. The feeling itself is
the sign, the tangible object that is the basis for unification, which
one is able to absorb into. As one experiences and nurtures that, one
can allow the mind to unify with the feeling, and that’s how it expands
and stabilizes.

\qaspace
Q: Can you speak about working with fear and loss of ego identity, fear
and death?

\qaspace
A: That’s one of the places where loving-kindness is a very skillful
meditation and exercise, because that sense of fear easily comes up with
the loss of the familiar, with the uncertainty of where to place one’s
attention. What can one trust as one starts to see? Body: can’t rely on
that. Feelings, perceptions, thoughts: completely untrustworthy.
Consciousness: not a good deal. You look at the world around and \ldots{}

So it’s easy to be shaken by the instability and uncertainty of
everything, and there can be a fear there, uncertainty, a certain
confusion and discombobulation. It’s good to be able to recognize that,
even if our experience is completely uncertain and unstable, “This
particular feeling of loving-kindness is trustworthy. I’ve experienced
that. I’ve seen that. That’s a true feeling within the heart.”

There is also the confidence that arises just in virtue, all the things
that the Buddha encourages. There isn’t anything that the Buddha
encourages that is something to be looked on askance, as if it were
untrustworthy.

On a certain level when one first approaches the teachings, like in the
morning chanting: Birth is dukkha, ageing is dukkha, sickness is dukkha,
death is dukkha, separation is dukkha, association is dukkha:
everything’s dukkha. One can think, “Wow, this is a miserable teaching.”

But then when one investigates: “Well, what does the Buddha actually
encourage you to do? What does he lay out as a path of practice and
training?” The Noble Eightfold Path. Right view, right intention:
association with wisdom. Right speech, right action, right livelihood:
virtue. Right effort, right mindfulness, right concentration: peaceful,
tranquil states of mind. Well, it’s all good. The path is completely
associated with the wholesome, the skillful, the uplifting.

Then there are other spiritual attributes we can trust, say, the
qualities that are conducive to passing over, transcending, crossing
over suffering: the \emph{pāramīs}. Generosity, virtue, renunciation,
patience, effort, resolution, truthfulness, loving-kindness, equanimity,
wisdom: everything that one could conceive of that is really good.
That’s what the Buddha encourages and that’s what forms the basis for
the ability to cross over suffering. The heart is replete with
well-being and stability. Recollecting and reminding oneself of what the
Buddha encourages us to cultivate and develop: those are all exceedingly
beneficial, bright states of mind.

When aspects of fear come up, it’s helpful to be able to recognize where
they come from. Of course, they come from identification with self,
identification with ego, identification with body, identification with
mind, identification with the world and all the things that we have
pretty much no control over. So when one looks at it quite clearly, one
realizes, “Well, I really have very little control.” It’s a very
practical teaching to allow us to approach this. It’s quite natural in
the human condition to be dismayed at the loss of, the threat to self
and ego.

There’s a wonderful quote by Ajahn Mahā Boowa, speaking about Ajahn Mun.
Ajahn Mun once taught, “Normally, as human beings, we respond to
teachings on liberation and nibbāna with a curious sense of fear and
trepidation. I don’t want to go there. I don’t want to do that. There’s
nothing there. Our parents and grandparents have all taught us quite
well to attach to family, attach to possessions, and attach to
position.” That’s our conditioning. As the conditioning changes,
inevitably there is a shift that takes place, and I think it is
important to recollect what the Buddha actually encourages us to do.
This is good stuff.

\qaspace
Q: I’m not sure within the context of this retreat when to be resting
simply with the four foundations of mindfulness, and when to be reciting
mettā phrases. Can you please advise as to how and when to move
skillfully from one practice to the other?

\qaspace
A: Within the context of how I’ve been presenting what is “billed” as a
mettā retreat, there is a recognition of how mettā is a part of a
spectrum of practices that are the Buddha’s teachings. Certainly the
four foundations of mindfulness are an overarching framework to attend
to.

There’s a discourse (M 19) where the Buddha talks about establishing
mindfulness on particular skillful, wholesome thoughts and being able to
hold that exclusively for a day, or a night, or a day and a night:
establishing the mind in that, discerning what’s skillful and
unskillful, and then committing to the wholesome thoughts. But then the
mind gets tired, so then it’s necessary to return to a place of
stillness or stability.

One has to recognize clearly and engage skillfully in the recitation of
the mettā phrases. It’s using a thought process, and, by its very
nature, it’s easy for it to become repetitious and feel tiring to the
mind. Again, it’s important to recognize that the recitation of those
phrases is to be used as a recollection and a reminder of the actual
feeling. As one establishes that feeling, which is non-discursive, the
mind can stay with that. But the recitation is a helpful tool to keep
the mind within the bounds of mettā, the loving-kindness perception.

But if one is continuing to do that and the feeling is not strong enough
to enable one to rest in it in a non-discursive way, then it’s important
to rest the mind on an object within the four foundations of mindfulness
– body, feeling, mind, and objects of mind – in some way. Of course, the
body is an immediate non-discursive object that the mind can rest in,
relax in, bringing a settling quality throughout the body and mind. When
the mind is refreshed, then it can go back to the recitation to keep
developing and stabilizing the perception of the feeling. So, one can
work with them back and forth. There’s no hard and fast rule. It’s more:
What is the result? What is the state of the mind? What is going to be
useful for settling and stilling?

\qaspace
Q: It seems that mettā would be much easier without a self to protect.
How does one realize anattā?

\qaspace
A: That’s an excellent question, in the sense that, as an insight, one
starts by realizing that the obstruction to mettā is actually not anger
and ill-will, but the sense of self. That’s the big obstruction, and as
we reflect, investigate and start having a clear insight into the nature
of Dhamma, the nature of the teachings and what the Buddha is giving us
as an opportunity for liberation, then loving-kindness is a natural
expression of that not-self aspect of the Dhamma. Yes, the cultivation
of mettā would be a lot easier without the perception of self.

\qaspace
Q: What is \emph{upekkhā}, equanimity, and how does one practice it?

\qaspace
A: Upekkhā is equanimity and evenness of the mind, the mind being in
perfect balance, a sense of recognition of the nature of \emph{kamma}.
The practice and basis of equanimity is traditionally described as
understanding the nature of \emph{kamma}. The chants we do say: “All
beings are the owners of their \emph{kamma}, heir to their \emph{kamma},
born of their \emph{kamma}, related to their \emph{kamma}, abide
supported by their \emph{kamma}. Whatever \emph{kamma} they shall do,
for good or for ill, of that they will be the heir.” The sense of
equanimity is being able to recognize that whatever anybody is
experiencing is because of causes and conditions that have been laid
down.

The stain or “near enemy” of equanimity is indifference. Equanimity is
the ability to be very steady and clear, and be in a place of equipoise
and balance. One recognizes that the way things work is through causes
and conditions, and there’s not a “self” involved in it, there’s not an
individual, it’s merely collections of causes and conditions.

Our habit of mind is to go to this person, that individual, that
personality, our preferences, our likes and our dislikes. But stepping
back to a place of equanimity is being able to see these causes and
conditions, so one is not shaken by anything. But it is important to
recognize that indifference is not equanimity.

Also, the brahmavihāras are not separate, individual things, like you
only have equanimity because you’ve given up on loving-kindness,
compassion, and sympathetic joy. Because loving-kindness and compassion
are exceedingly strong and stable, one has increased understanding and
insight, so that one is able to rest in a place of complete stability
and not be shaken by anything, because one sees that the nature of all
things is to arise through causes and conditions.

The reflection on \emph{kamma} is that all beings inherit the results of
their \emph{kamma}. There’s not a personality, a “me” or “mine.” It’s
not fixed. It isn’t as if something happens and then that inevitable,
immutable state of being is going to exist forever. Everything is in a
state of change. Patterns and circumstances can all change. As one
recognizes that and has a much clearer insight into the nature of what
causes and conditions are functioning, then one has an opportunity to
put attention onto the place where it is going to make the most
difference.

\qaspace
Q: What is the most fun part about being a monk?

\qaspace
A: I can assure you it’s not fun. It’s like eating and sleeping. It’s
very ordinary. Probably in terms of the most delight or joy that comes
is the opportunity to meet really accomplished and pure beings. Fun for
a monk is the opportunity to draw close to people and to serve them, or
to be in situations where one has contact. You get to meet some
exceptional beings.

\qaspace
Q: What does “the four pairs, the eight kinds of noble beings,” in the
Recollection of the Sangha, mean?

\qaspace
A: It’s a way of classification in terms of entry into realization.
There are four stages of realization: steam-entry, once-returner,
non-returner, and fully enlightened being, or arahant. \emph{Sotāpanna},
\emph{sakadāgāmi}, \emph{anāgāmi}, arahant. So there’s four.

Then there’s path and fruition. So there are those who are on the path
to stream-entry and those who have reached its fruition, those who are
on the path to once-returning and those who have reached its fruition,
those who are on the path to non-returning and its fruition, and those
who are on the path to arahantship and its fruition. Those are the four
pairs and eight kinds of noble beings.

In the commentarial tradition, the way they line it up is that the path
and fruition are tied to each other. There’s a realization of path, and,
within mind-moments, there’s a realization of fruition. There are a few
suttas in the Canon, however, that make it clear that, for someone who
is on the path, it could be a long period of time, years even, before
there is a fruition. To me, that makes sense.

The sutta tradition has a clear description that there are those who
have got an insight into what the path is, what the practice is, what
the actual goal is, but who are still to have the full realization of
that. It could be a very quick insight and realization, or it could take
time for that to come to fruition. Those are the four pairs, the eight
kinds of noble beings. It \emph{is} a bit mysterious when we chant it.

\qaspace
Q: When doubt is mentioned under the hindrances, is it mainly referring
to doubt about the Buddha’s teachings? Are there other implications?

\qaspace
A: Oftentimes, when one reads the texts and commentaries, inevitably
they talk about doubt about the Buddha’s teachings. I and most of the
teachers that I know don’t narrow it down that much.

In terms of a hindrance, a \emph{nīvaraṇa}, a quality of mind that
obstructs the fundamental goodness of the mind, then it isn’t just doubt
about the Buddha’s teachings. It’s the tendency to doubt, to waver. Of
course, when you doubt, waver, are uncertain, then the focus will be on
the teachings, but you’ll tend to doubt everything else as well.

So it’s good to get a handle on doubt, uncertainty, wavering: how it
plays itself out in day-to-day life. It’s really interesting to watch on
a day-to-day basis. How does it play out? “Should I have two of those
fruits or should I just have one? It can turn into a moment where
everything stops: should I do this or should I do that; go forward or go
back?” We start to doubt everything, doubt can creep into everything and
then it manifests in the teachings as well: doubting the teachings,
doubting the practice, doubting oneself, can one do it, can one not do
it, is there any point even to doing this, “I’m not sure that I can do
this.” All of it is doubt. So, being able to recognize that it’s doubt,
and that’s all it is, one can then go forward.

This is also a situation where mindfulness of the body is really
helpful, in the sense that you’re never going to talk yourself into a
resolution of doubt through the mind. You’ll never get everything
figured out enough not to doubt, if doubt is the underlying, fallback
position. Recognize how doubt manifests as a feeling of uncertainty, a
feeling of hesitation, a feeling of discomfort around a particular
decision or action. Then, by employing mindfulness of the body, you can
cut through that and move forward. Sometimes it’s helpful to make a
wrong decision and do something, at least not get caught in wavering and
uncertainty, just as a part of training. That sense of wanting to be
absolutely sure that one is doing something right before doing it
paralyzes you and ties you up in knots. So, be willing to cut through
that and move forward. One can feel in the body that it is helpful just
to move forward, to take action, and not get caught in doubt.
