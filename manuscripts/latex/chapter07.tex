\chapter{The Karaṇīya-Mettā
Sutta}

Yesterday, I introduced a method for the cultivation of loving-kindness
using the phrases: “May I be well, happy, peaceful. May no harm come to
me. May no difficulties come to me. May no problems come to me. May I
always meet with spiritual success. May I have the patience, courage,
understanding, and determination to meet and overcome inevitable
difficulties, problems, and failures in life.”

Then we used those same phrases to generate the recollection of
loving-kindness towards parents, teachers, family, friends, those who
are unfriendly, and then all living beings. I think the use of phrases
is a helpful reminder to try to direct attention to a particular
feeling. The feeling of loving-kindness is the object of meditation in
this practice: the actual feeling, an emotional tone within the heart.
That is what we are trying to generate, support, and sustain. As we
direct attention, it’s like trying to enter into and abide in that
feeling, allowing it to establish itself.

I think it’s also quite important in terms of the cultivation of
loving-kindness that we recall this \emph{mettā-nimitta}, the sign of
loving-kindness. Find a place where that feeling of mettā begins to
establish itself and then protect it. Then use the different phrases to
try to support, nurture, and hold loving-kindness in a skillful way so
that it can grow and expand.

Sometimes there is a problem with language: directing attention to
parents and teachers isn’t like taking loving-kindness and beaming it
off this way and that way, shooting it out and radiating it. It’s much
more the sense of beginning with what might be a little spark and
allowing the heart to be a vessel for loving-kindness. What we do is
then expand that vessel, expand that sphere of loving-kindness, so that
it includes parents, family, teachers, friends, and ultimately, all
living beings.

Alternatively, it’s establishing that base of loving-kindness, that
mettā-nimitta, stabilizing it, nurturing it, and then inviting parents,
teachers, and friends into it. Allow them to come into that sphere,
include them so that the base of loving-kindness is right here, within
this body and mind, in the present moment. This is where the base of
loving-kindness is—then allow that to include other beings, inviting and
expanding it so that it’s not excluding anything.

This is thus a different approach in terms of stabilizing the
concentration and using it as a meditation that provides a solid anchor
within the heart. It’s quite important, this shift of perspective to
having a stable vessel of loving-kindness that then expands, allowing it
to include all of these other beings we associate with. It’s never
divorced or separated from the loving-kindness that is established
toward yourself. That is the base that you are always returning to and
depending on: “May I be well, happy, peaceful. May no harm come to me.
May no difficulties come to me. May no problems come to me. May I always
meet with spiritual success. May I have the patience, courage,
understanding, and determination to meet and overcome inevitable
difficulties, problems, and failures in life.”

It’s always grounded in respect, kindness, and well-wishing toward
yourself. That’s not selfish. It’s the most practical reality, to be
able to look after yourself so that you have the resources to be able to
support others.

Over these days, I’ll introduce a few different methodologies, options,
and kinds of phraseology. What rings true for you? What feels
meaningful?

Today we will address the Buddha’s words on loving-kindness, which is
one of the chants that we do. It’s probably the most famous expression
of loving-kindness that the Buddha taught. The formula of directing
loving-kindness towards yourself, parents, teachers, and friends is
probably the most common structure. However, this particular chant is
the most well-known teaching on loving-kindness in the suttas.

I’ve heard a couple of different Pali scholars who are fluent in Pali,
Bhikkhu Bodhi and Bhante Guṇaratana, say this is the most beautiful
discourse in terms of the scriptural Pali language. There is not a spare
word anywhere; it’s all perfect. Of course, in trying to get things into
English, you try your best, and there are many different translations.
The beauty of the Internet is that you can find a half-dozen different
translations of the \emph{Karaṇīya-Mettā Sutta}.

One of the striking things, given that this is the Buddha’s best-known
discourse on loving-kindness, is that it is about a third of the way
into the sutta before the Buddha even mentions loving-kindness. I think
this is great, in the sense that loving-kindness is to be cultivated but
that there is also that which should be done to get to that point. We
need to be skilled in goodness and know the path of peace. “Let them be
able and upright, straightforward and gentle in speech.”

This is not just a mental phenomenon. We can’t just grab a feeling of
loving-kindness out of the ether of our mental states and hold on to it.
No. It arises out of our actions and speech. This is the ground that we
need to be attending to: the sense of our actions being able and
upright, our speech being straightforward and direct.

When we chant the recollection of the Sangha the same word occurs:
\emph{uju}, direct, straight, upright, practicing directly. Be
straightforward in speech, direct, but also gentle. There is a vivid
idiom in the scriptural language: “Attacking each other with verbal
daggers.” Here’s another one: “Someone born with an axe in their mouth.”
Speech can be just as harsh, cutting, and devastating as action, so it’s
important to be gentle as well as straightforward in speech.

The quality of being “humble and not conceited”: there is a softness and
a harmonious quality that arises in the human condition when somebody
has humility and is not carrying around conceits of their importance and
worth. “Contented and easily satisfied”: similarly, in terms of being
able to harmonize with each other in the human community, the sense of
contentment and being easily satisfied is one that allows us to blend
with each other. If we’re discontented and difficult to satisfy, with
constant demands, whether they are material or emotional, we’re always
walking on eggshells. It’s difficult to get along. Again, there is an
idiom from the scriptural language: “blending like milk and water.”

One discourse comes to mind immediately (M 128). The scenario is that
the Buddha has left the community because he has become fed up. The
monks were arguing and refusing to listen to his advice. He went to a
forest where a few other monks were living. One was his cousin,
Anuruddha. The Buddha arrives and asks them, “Are you living
harmoniously? Are you living blending like milk and water?” That’s an
image that is used to indicate that the sense of being contented and
easily satisfied is a source of inner well-being as well as outer
well-being.

“Unburdened with duties and frugal in their ways”: not having too many
things on our plate, not taking on more than we can handle. Living in
simple ways: these are ways that allow us to have ease within the heart
that can then be translated into a cultivation of a mature emotion like
loving-kindness. “Peaceful and calm, wise and skillful”: these are all
qualities to be reminding ourselves of, in the sense of bringing them
into our daily lives, qualities of calm and discernment. Making
decisions, not according to our preferences or our ideals, but by
asking: “What is truly skillful, wholesome, and beneficial? What is
neither beneficial nor skillful, what is being driven either by your own
biases or by the biases of the society around you?”

The Buddha defines that which is skillful as that which inherently leads
to happiness, peace, and a freedom from the defilements and
obscurations. What is unskillful, its opposite, is that which is
associated with suffering, with \emph{dukkha}. It’s agitated, turbulent,
and associated with greed, hatred, and delusion.

Asking, “What is really beneficial and skillful?” is helpful in making
decisions, rather than, “Do I like it? Do I not like it? What is that
person going to think? What’s the popular notion within this particular
group of people?” The Buddha has laid it out quite clearly. There is an
intrinsic well-being that comes from aligning ourselves with the
skillful.

“Not proud and demanding in nature. Let them not do the slightest thing
that the wise would later reprove.” If we wish to take into account and
worry about what somebody is going to think, let’s take the standards of
somebody who is really wise and use that as a sounding board.

“Wishing in gladness and in safety, may all beings be at ease.” The
scriptural Pali is, \emph{sukhino vā khemino hontu, sabbe sattā bhavantu
sukhitattā.} \emph{Sukhino} is happy, with gladness and well-being: may
they be happy; may they experience well-being. \emph{Khemino} is safety,
security, stability, freedom from any kind of threat. \emph{Sabbe sattā
bhavantu sukhitattā}: may all beings bring this quality of happiness and
well-being into being; may they be at ease.

Establish that wish, that thought, that aspiration. The Buddha then
encourages us to expand that. “Whatever living beings there may be;
whether they are weak or strong, omitting none; the great or the mighty,
medium, short, or small.” Weak or strong is not just a physical
attribute. It could also be a social attribute, whether they are
powerful or disadvantaged. It’s not just whether they are
ninety-eight-pound weaklings or Mr. Universe. It’s much more regarding
any physical, social, or emotional manifestation.

Whether weak or strong, every being is worthy of the wish, “May they be
at ease,” because whether they’re weak or strong, they’re suffering.
Ajahn Chah said, “Poor people suffer like poor people, rich people
suffer like rich people; intelligent people suffer like intelligent
people; and not-so-intelligent people suffer like not-so-intelligent
people.” That’s the universal leveler: we all suffer.

I remember Ajahn Chah teasing us Westerners because we generally had far
more education than most of the Thai monks who went to study, train, and
ordain with him. As he became better-known, there were more Bangkok
Thais coming who were well educated also. It was always a source of
amusement to him. He would say, “You know, people go and study and they
get bachelor’s degrees, but their defilements get bachelor’s degrees as
well. Then they get master’s degrees, and their defilements also get
master’s degrees. They say, well, this isn’t satisfying, so they decide
to get doctorates, but then their defilements get Ph.D.s as well.”

It’s the nature of beings immersed in greed, hatred, and delusion
without liberating insight. We all suffer. Generating loving-kindness
cuts across the board.

Another thing Ajahn Chah said—I can’t resist—teasing us as Westerners
because we came from an affluent society and had education: “Yeah, well,
you Westerners have affluence and education. Look at the vultures—they
fly really high but look at what they come down to eat.” It was always a
source of amusement to him.

The chant conveys the message of spreading loving-kindness to all living
beings, “the seen and the unseen.” Again, we realize that even if we
don’t see or know them on a personal level, we know that all beings are
worthy and would respond to that wish to live in safety and happiness.
“Living near or far away, those born and to be born.” In the Pali, it’s
\emph{sambhavesī vā}, those beings that are seeking birth: they don’t
know what they are in for. They think, “Boy, I can’t wait to get born.”
They get bored with something, whatever realm they’re in, and they
think, “I want to be born, I want to be reborn.” Of course, that’s the
cycle—it goes around and around.

So, having that deep sense of kindness and compassion towards all
beings—\emph{sabbe sattā sukhitā hontu}, “May all beings be at ease.”
\emph{Sabbe sattā bhavantu sukhitattā}, “May all beings bring that
happiness into being”—it then comes back to how we live with each other,
living with loving-kindness.

“Let none deceive another”: the commitment to real honesty and
straightforwardness in our dealings with each other and, again,
recognizing that how beings can have a sense of safety and happiness is
by having a sense of trust with each other. “Let none deceive another.”
Be solidly committed to that which is true and try to be as upfront and
clear with each other as possible.

It’s also good to qualify that, bringing it back to the sense of being
gentle in speech as well, because sometimes we can take truth as an
ideal and forget the effect it’s going to have on the ground.
Skillfulness is very important in how we deal with each other.

“Or despise any being in any state”: there is a common human tendency to
lift oneself up by putting somebody else down. “Despise” is a very
strong word, but it has the sense of looking down on somebody, blowing
them off, or disregarding them in some way. As human beings we easily do
that because there is a sense that criticizing or being able to put
others down lifts ourselves up a bit. It’s just not a very beautiful
mental state.

When I went to Thailand, I had been traveling around the world for about
a year and then was ordained as a Buddhist monk. After having been a
monk for about sixteen years, I went back to Canada for the first time
for a visit. After being in the rural culture of monasteries in the
northeast of Thailand, which is quite a gentle culture, I found it quite
painful to see how people spoke and talked, putting each other down.
Being exposed to media and the sarcasm, critical put-downs, and snide
comments: I found it very painful.

It’s ordinary. I don’t think my family was more egregious than anybody
else. In fact, they were pretty good. It’s just a norm within the
culture. I think it’s important to become very conscious of that, so
that we aren’t drawn into putting anybody down in any way, making fun of
others, criticizing, or being sarcastic.

“Let none, through anger or ill will, wish harm upon another”: this
means protecting and sustaining your thoughts of loving-kindness and
well-wishing, or at least neutrality. And then there is the phrase,
“Even as a mother protects with her life her child, her only child.” The
Buddha very consciously uses the archetypal image of the mother who
identifies with the child. It’s her only child and therefore, there is a
ceaseless well-wishing on her part. Even if a mother is not right there
with the child, her antennae are out. The slightest sound pulls the
mother back to check and make sure that the child is okay, so the image
is very apt.

Then the Buddha says, “With a boundless heart should one cherish all
living beings.” The sense is that, in the way that the mother doesn’t
think of herself and her first thought is for the child, we, too, have
that boundless quality through which we’re not thinking of ourselves
anymore, we’re setting ourselves aside. We establish the quality of
cherishing all living beings and radiating kindness, allowing those
thoughts and feelings of kindness, of mettā, to permeate: “Spreading
upwards to the skies and downward to the depths, outwards and unbounded,
freed from hatred and ill will.”

There is a sense of unobstructed radiating and of not choosing. Not
saying, “I’ll radiate upwards, downwards, out in all directions
everywhere, except for that person.” We don’t have that limitation.

This emphasizes the quality of being freed from hatred and ill will and
that the basis of loving-kindness is abiding in non-aversion. It’s not
being trapped by negativity, aversion, ill will, or irritation. It’s
important to recognize that though thoughts of irritation may come up in
the mind, as long as we don’t feed, nurture, or support them, they’re
just thoughts. Or if there is a memory or perception, as long as we
don’t invest in, believe in, or support it, then it doesn’t have a
landing place. It doesn’t have a place to establish itself. So as long
as we don’t identify with it, it ceases.

The nature of the human mind is such that we have a whole bank of
experiences, both internally, within ourselves, and externally, from
society, that impinges on us. All of that can pop up in the mind as
thoughts because that is what a mind is for, to provide thoughts. But as
meditators, we have to use discernment. What are we going to invest in?
What are we going to put our energy into?

That is where \emph{kamma} is created, whether it be the mental
\emph{kamma} of investing in thoughts of ill will, aversion, and greed,
or the mental \emph{kamma} of investing in renunciation,
loving-kindness, and compassion. We are making that choice. That’s the
\emph{kamma} that we’re creating.

Sometimes people think, perhaps if they’re upset or devastated, “I had
this terrible thought of aversion \ldots{}” Well, did you act on it? No?
Well, don’t make a problem out of it.

Thoughts pop up in the same way that there are all kinds of things
happening in the world around us. What do we pay attention to? Where do
we allow our attention to rest and gather momentum? Abiding in
loving-kindness is a conscious choice that we make, both for ourselves
and on account of how it affects others.

“Whether standing or walking, seated or lying down.” These are the four
postures. If we’re not doing one, we’re doing another, 24/7. This is our
opportunity for practice: whether sitting, walking, lying down, or
standing.

“Free from drowsiness”: of course, it helps not to let the mind get
clouded. It isn’t just any sort of drowsiness or sleepiness, however,
but the kind of drowsiness or dopiness that can come from going on
automatic pilot. We can drowse through all sorts of interactions and
activities.

“One should sustain this recollection.” The word that’s used in the
sutta itself is \emph{sati}: to sustain this mindfulness, this
awareness, this recollection. “This is a sublime abiding.”

This is a chant that many of the people associated with monasteries know
by heart. It’s helpful to have these chants available so that when you
chant, you can pick out phrases that resonate. Things will resonate
differently at different times. It’s good to have a phrase or a whole
series of phrases that help remind you.

During the time that I was on sabbatical in Thailand, a couple of years
ago now, I was living on my own, and I used those phrases that I
introduced yesterday a lot. Something that also struck a chord with me
is the phrase, “Wishing in gladness and in safety, may all beings be at
ease,” in both English and Pali. Because I chant it in Pali all the
time, the Pali is meaningful for me as well. So I would alternate back
and forth. I would walk almsround and that would be the mantra. Not just
repeating it as quickly as I could, but planting that seed and carrying
it through, planting that seed and carrying it through, generating
loving-kindness. I found it very fruitful.

So take some of these phrases and recollections and experiment with
them. Work with them and see what happens.
