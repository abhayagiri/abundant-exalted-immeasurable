\chapter{Sutta Readings}

This afternoon, I thought I’d take the opportunity to do some sutta
readings about loving-kindness and give people the real scoop—I’ve
rambled on long enough. There are some sutta readings that are
interesting in that they are not so familiar to people or take something
familiar and cast it in a new light. Then there are some that are
daily-life applications of loving-kindness.

The first one I thought I would read is one I have mentioned. It’s an
occasion when the Buddha goes to visit three monks, one of whom is the
Venerable Anuruddha, who was a very accomplished monk and also the
Buddha’s cousin.

The Buddha was departing from a situation not mentioned in the sutta. A
particular monastic community was in disharmony. Actually, there was a
schism happening in the sangha. The Buddha tried to intervene, and they
encouraged the Blessed One to live in peace: “We’ll sort this out,”
which meant, “We are going to go at it tooth and nail. Please don’t get
in our way.” He gave up on them, went off on his own, and came across
three monks who were living in a woodland area. The Buddha sat down, and
the monks paid their respects to him.

\begin{quotation}
“I hope, Anuruddha, that you are all living in concord, with mutual
appreciation, without disputing, blending like milk and water, viewing
each other with kindly eyes.” “Surely, venerable sir, we are living in
concord, with mutual appreciation, without disputing, blending like milk
and water, viewing each other with kindly eyes.” “But, Anuruddha, how do
you live thus?”~“Venerable sir, as to that, I think thus: ‘It is a gain
for me, it is a great gain for me, that I am living with such companions
in the holy life.’ I maintain bodily acts of loving-kindness towards
those venerable ones both openly and privately; I maintain verbal acts
of loving-kindness towards them both openly and privately; I maintain
mental acts of loving-kindness towards them both openly and privately. I
consider: ‘Why should I not set aside what I wish to do and do what
these venerable ones wish to do?’ Then I set aside what I wish to do and
do what these venerable ones wish to do. We are different in body,
venerable sir, but one in mind.” The venerable Nandiya and the venerable
Kimbila each spoke likewise, adding: “That is how, venerable sir, we are
living in concord, with mutual appreciation, without disputing, blending
like milk and water, viewing each other with kindly eyes.”

(M 128)
\end{quotation}

That’s a delightful description of what is possible in the human realm.
There is another discourse in which the Buddha is trying to convince the
monks in Kosambi, the monks referred to in the previous discourse, who
had been disputing and creating conflict. He admonishes them and then he
outlines a more appropriate, skillful way to live together.

\begin{quotation}
“So, bhikkhus, when you take to quarreling and brawling and are deep in
disputes, stabbing each other with verbal daggers, on that occasion you
do not maintain acts of loving-kindness by body, speech, and mind in
public and in private to your companions in the holy life.

“Misguided men, what can you possibly know, what can you see, that you
take to quarreling and brawling and are deep in disputes, stabbing each
other with verbal daggers? That you can neither convince each other nor
be convinced by others, that you can neither persuade each other nor be
persuaded by others? Misguided men, that will lead to your harm and
suffering for a long time.”~Then the Blessed One addressed the bhikkhus
thus: “Bhikkhus, there are these six principles of cordiality that
create love and respect and conduce to cohesion, to non-dispute, to
concord, and to unity. What are the six?

{[}1{]} “Here a bhikkhu maintains bodily acts of loving-kindness both in
public and in private towards his companions in the holy life. This is a
principle of cordiality that creates love and respect, and conduces to
cohesion, to non-dispute, to concord, and to unity.

{[}2{]} “Again, a bhikkhu maintains verbal acts of loving-kindness both
in public and in private towards his companions in the holy life. This
too is a principle of cordiality that creates love and respect, and
conduces to \ldots{} unity.

{[}3{]} “Again, a bhikkhu maintains mental acts of loving-kindness both
in public and in private towards his companions in the holy life. This
too is a principle of cordiality that creates love and respect, and
conduces to \ldots{} unity.

{[}4{]} “Again, a bhikkhu uses things in common with his virtuous
companions in the holy life; without making reservations, he shares with
them any gain of a kind that accords with the Dhamma and has been
obtained in a way that accords with the Dhamma, including even the mere
contents of his alms-bowl. This too is a principle of cordiality that
creates love and respect, and conduces to \ldots{} unity.

{[}5{]} “Again, a bhikkhu dwells both in public and in private
possessing in common with his companions in the holy life those virtues
that are unbroken, untorn, unblotched, unmottled, liberating, commended
by the wise, not misapprehended, and conducive to concentration. This
too is a principle of cordiality that creates love and respect, and
conduces to \ldots{} unity.

{[}6{]} “Again, a bhikkhu dwells both in public and in private
possessing in common with his companions in the holy life that view that
is noble and emancipating, and leads one who practices in accordance
with it to the complete destruction of suffering. This too is a
principle of cordiality that creates love and respect, and conduces to
cohesion, to non-dispute, to concord, and to unity.”

(M 48)
\end{quotation}

The six qualities of creating concord and living together come up in
different places in the suttas: loving-kindness in acts, speech, and
thought; sharing with generosity; common virtue and common view. There
is also the stock phrase “virtues that are unbroken, untorn, unblotched,
unmottled, liberating, commended by the wise, not misapprehended, and
conducive to concentration.”

These virtues are also talked about as a distinguishing characteristic
of one who has entered the stream of Dhamma: a stream-enterer. It is
interesting how the Buddha most often describes the stream-enterer not
by their dazzling meditations or astute wisdom, but by their qualities
of faith in Buddha, Dhamma, and Sangha, and well-established virtue.
Such virtue in not misapprehended, not clung to, not attached to. It is
commended by the wise and conducive to concentration.

Virtue is settling. There is composure involved. Virtue is not “keeping
you in line,” but allowing you to be established in a quality of
non-remorse and well-being, so that you can concentrate easily. It is a
factor of our development of meditation.

This next sutta reading also comes from \emph{The Middle-Length
Discourses}. It’s a very famous sutta, \emph{The Simile of the Saw}.

There is a buildup to the simile. The incident that gave rise to the
teaching is that one of the bhikkhus was in conflict with the other
bhikkhus and was not taking admonishment very well. The circumstance
came to the attention of the Buddha, so he called the monk to him and
gave a teaching. He took the occasion to teach all of the monks who were
present, a situation that often happened, not just in the Buddha’s time.
I do it all the time: a circumstance comes up and it is a great occasion
to be used as an example of application of Dhamma within the community.

\begin{quotation}
“So too, bhikkhus, some bhikkhu is extremely gentle, extremely kind,
extremely peaceful, so long as disagreeable courses of speech do not
touch him. But it is when disagreeable courses of speech touch him that
it can be understood whether that bhikkhu is really kind, gentle, and
peaceful. I do not call a bhikkhu easy to admonish who is easy to
admonish and makes himself easy to admonish only for the sake of getting
robes, almsfood, a resting place, and medicinal requisites. Why is that?
Because that bhikkhu is not easy to admonish nor makes himself easy to
admonish when he gets no robes, almsfood, resting place, and medicinal
requisites. But when a bhikkhu is easy to admonish and makes himself
easy to admonish because he honors, respects, and reveres the Dhamma,
him I call easy to admonish. Therefore, bhikkhus, you should train thus:
‘We shall be easy to admonish and make ourselves easy to admonish
because we honor, respect, and revere the Dhamma.’ That is how you
should train, bhikkhus.

“Bhikkhus, there are these five courses of speech that others use when
they address you: their speech may be timely or untimely, true or
untrue, gentle or harsh, connected with good or with harm, spoken with a
mind of loving-kindness or with inner hate \ldots{} Herein, bhikkhus,
{[}regardless of how someone speaks{]} you should train thus: ‘Our minds
will remain unaffected, and we shall utter no evil words; we shall abide
compassionate for their welfare, with a mind of loving-kindness, without
inner hate. We shall abide pervading that person with a mind imbued with
loving-kindness, and starting with him, we shall abide pervading the
all-encompassing world with a mind imbued with loving-kindness,
abundant, exalted, immeasurable, without hostility and without ill
will.’ That is how you should train, bhikkhus.

“Bhikkhus, suppose a man came with a hoe and a basket and said: ‘I shall
make this great earth to be without earth.’ He would dig here and there,
strew the soil here and there, spit here and there, and urinate here and
there, saying: ‘Be without earth, be without earth!’ What do you think,
bhikkhus? Could that man make this great earth to be without
earth?”—“No, venerable sir.” “Why is that? Because this great earth is
deep and immeasurable; it is not easy to make it be without earth.
Eventually the man would reap only weariness and disappointment.

“So too, bhikkhus, there are these five courses of speech \ldots{} Herein,
bhikkhus, you should train thus: ‘Our minds will remain unaffected \ldots{}
and starting with him, we shall abide pervading the all-encompassing
world with a mind similar to the earth, abundant, exalted, immeasurable,
without hostility and without ill will.’ That is how you should train,
bhikkhus.

“Bhikkhus, suppose a man came with crimson, turmeric, indigo, or carmine
and said: ‘I shall draw pictures and make pictures appear on empty
space.’ What do you think, bhikkhus? Could that man draw pictures and
make pictures appear on empty space?”—“No, venerable sir.” “Why is that?
Because empty space is formless and non-manifestive; it is not easy to
draw pictures there or make pictures appear there. Eventually the man
would reap only weariness and disappointment.

“So too, bhikkhus, there are these five courses of speech \ldots{} Herein,
bhikkhus, you should train thus: ‘Our minds will remain unaffected \ldots{}
and starting with him, we shall abide pervading the all-encompassing
world with a mind similar to empty space, abundant, exalted,
immeasurable, without hostility and without ill will.’ That is how you
should train, bhikkhus.

“Bhikkhus, suppose a man came with a blazing grass-torch and said: ‘I
shall heat up and burn away the river Ganges with this blazing
grass-torch.’ What do you think, bhikkhus? Could that man heat up and
burn away the river Ganges with that blazing grass-torch?”—“No,
venerable sir.” “Why is that? Because the river Ganges is deep and
immense; it is not easy to heat it up or burn it away with a blazing
grass-torch. Eventually the man would reap only weariness and
disappointment.

“So too, bhikkhus, there are these five courses of speech \ldots{} Herein,
bhikkhus, you should train thus: ‘Our minds will remain unaffected \ldots{}
and starting with him, we shall abide pervading the all-encompassing
world with a mind similar to the river Ganges, abundant, exalted,
immeasurable, without hostility and without ill will.’ That is how you
should train, bhikkhus.

“Bhikkhus, suppose there were a catskin bag that was rubbed, well
rubbed, thoroughly well rubbed, soft, silky, rid of rustling, rid of
crackling, and a man came with a stick or a potsherd and said: ‘There is
this catskin bag that is rubbed \ldots{} rid of rustling, rid of crackling. I
shall make it rustle and crackle.’ What do you think, bhikkhus? Could
that man make it rustle or crackle with the stick or the potsherd?”—“No,
venerable sir.” “Why is that? Because that catskin bag being rubbed \ldots{}
rid of rustling, rid of crackling, it is not easy to make it rustle or
crackle with the stick or the potsherd. Eventually the man would reap
only weariness and disappointment.

“So too, bhikkhus, there are these five courses of speech that others
may use when they address you: their speech may be timely or untimely,
true or untrue, gentle or harsh, connected with good or with harm,
spoken with a mind of loving-kindness or with inner hate \ldots{} Herein,
bhikkhus, {[}regardless of how someone speaks{]} you should train thus:
‘Our minds will remain unaffected, and we will utter no evil words; we
shall abide compassionate for their welfare, with a mind of
loving-kindness, without inner hate. We shall abide pervading that
person with a mind imbued with loving-kindness; and starting with him,
we shall abide pervading the all-encompassing world with a mind similar
to a catskin bag, abundant, exalted, immeasurable, without hostility and
without ill will.’ That is how you should train, bhikkhus.

“Bhikkhus, even if bandits were to sever you savagely limb by limb with
a two-handled saw, he who gave rise to a mind of hate towards them would
not be carrying out my teaching. Herein, bhikkhus, you should train
thus: ‘Our minds will remain unaffected, and we shall utter no evil
words; we shall abide compassionate for their welfare, with a mind of
loving-kindness, without inner hate. We shall abide pervading them with
a mind imbued with loving-kindness; and starting with them, we shall
abide pervading the all-encompassing world with a mind imbued with
loving-kindness, abundant, exalted, immeasurable, without hostility and
without ill will.’ That is how you should train, bhikkhus.

“Bhikkhus, if you keep this advice on the simile of the saw constantly
in mind, do you see any course of speech, trivial or gross, that you
could not endure?”—“No, venerable sir.”—“Therefore, bhikkhus, you should
keep this advice on the simile of the saw constantly in mind. That will
lead to your welfare and happiness for a long time.”

That is what the Blessed One said. The bhikkhus were satisfied and
delighted in the Blessed One’s words.

(M 21)
\end{quotation}

This sets the bar pretty high, doesn’t it? You think, “How is it
possible when someone is attacking you not to give rise to any anger or
ill will?” It’s possible if you step back and look at it from a
perspective of an abundant heart of loving-kindness and with a
conviction in the Dhamma, knowing that because all beings are the owners
of their actions, heirs of the actions, born of their actions, and
related to their actions, those causes will result in rebirth in a
particular way. The result of anger, ill will, or any kind of violence
is an unfortunate rebirth. Being well-established in loving-kindness is
a cause that produces a result in exceedingly fortunate circumstances.
In the long run, it is worth training oneself so the option isn’t just
aversion, ill will, or irritation.

So those were some suttas in the application of loving-kindness in
day-to-day life. There is also an interesting discourse on the
application of loving-kindness as a basis for insight. It’s a discourse
by Ānanda. Ānanda actually defines a number of bases for insight in the
jhānas, the four brahmavihāras, and the three formless attainments, but
I’ll just mention how he describes loving-kindness.

\begin{quotation}
“Again, a bhikkhu abides pervading one quarter with a mind imbued with
loving-kindness, likewise the second, likewise the third, likewise the
fourth; so above and below, around, and everywhere, and to all as to
himself, he abides pervading the all-encompassing world with a mind
imbued with loving-kindness, abundant, exalted, immeasurable, without
hostility and without ill will. He considers this and understands it
thus: ‘This deliverance of mind through loving-kindness is conditioned
and volitionally produced. But whatever is conditioned and volitionally
produced is impermanent, subject to cessation.’ If he is steady in that,
he attains the destruction of the taints. But if he does not attain the
destruction of the taints, then because of that desire for the Dhamma,
that delight in the Dhamma, with the destruction of the five lower
fetters he becomes one due to appear spontaneously {[}in the Pure
Abodes{]} and there attain final nibbāna without ever returning from
that world.

“This too is one thing proclaimed by the Blessed One \ldots{} wherein if a
bhikkhu abides diligent, ardent, and resolute, his unliberated mind
comes to be liberated, his undestroyed taints come to be destroyed, and
he attains the supreme security from bondage that he had not attained
before.”

(M 52)
\end{quotation}

Even something as sublime as loving-kindness is impermanent. Seeing its
changing nature can be a basis for investigation and penetration.

I’ll do a few readings from \emph{The Numerical Discourses}, the
\emph{Aṅguttara Nikāya}. This is addressed to monks but applies to
everybody.

\begin{quotation}
“Monks, if for just the time of a finger-snap, a monk produces a thought
of loving-kindness, develops it, gives attention to it, such a one is
rightly called a monk. Not in vain does he meditate. He acts in
accordance with the Master's teaching, he follows his advice, and eats
deservingly the country’s alms-food. How much more so if he cultivates
it.”

(A 1:53)
\end{quotation}

So what we have been doing is not to be sniffed at. The next sutta is
one that everyone knows, the \emph{Kālāma Sutta,} but almost never when
the \emph{Kālāma Sutta} is quoted does loving-kindness get any mention.
So I thought I’d read the complete discourse.

\begin{quotation}
I have heard that on one occasion the Blessed One, on a wandering tour
among the Kosalans with a large community of monks, arrived at
Kesaputta, a town of the Kālāmas. The Kālāmas of Kesaputta heard it
said, “Gotama the contemplative—the son of the Sakyans, having gone
forth from the Sakyan clan—has arrived at Kesaputta. And of that Master
Gotama this fine reputation has spread: ‘He is indeed a Blessed One,
worthy, and rightly self-awakened, consummate in knowledge and conduct,
well-gone, a knower of the cosmos, an unexcelled trainer of those
persons ready to be tamed, teacher of human and divine beings, awakened,
blessed. He has made known—having realized it through direct
knowledge—this world with its devas, maras, and brahmas, its generations
with their contemplatives and brahmans, their rulers and common people;
has explained the Dhamma admirable in the beginning, admirable in the
middle, admirable in the end; has expounded the holy life both in its
particulars and in its essence, entirely perfect, surpassingly pure. It
is good to see such a worthy one.’”

So the Kālāmas of Kesaputta went to the Blessed One. On arrival, some of
them bowed down to him and sat to one side. Some of them exchanged
courteous greetings with him and, after an exchange of friendly
greetings and courtesies, sat to one side. Some of them sat to one side
having saluted him with their hands palm-to-palm over their hearts. Some
of them sat to one side having announced their name and clan. Some of
them sat to one side in silence.

As they sat there, the Kālāmas of Kesaputta said to the Blessed One,
“Lord, there are some brahmans and contemplatives who come to Kesaputta.
They expound and glorify their own doctrines, but as for the doctrines
of others, they deprecate them, revile them, show contempt for them, and
disparage them. And then other brahmans and contemplatives come to
Kesaputta. They expound and glorify their own doctrines, but as for the
doctrines of others, they deprecate them, revile them, show contempt for
them, and disparage them. They leave us absolutely uncertain and in
doubt: Which of these venerable brahmans and contemplatives are speaking
the truth, and which ones are lying?”

“Of course you are uncertain, Kālāmas. Of course you are in doubt. When
there are reasons for doubt, uncertainty is born. So in this case,
Kālāmas, don’t go by reports, by legends, by traditions, by scripture,
by logical conjecture, by inference, by analogies, by agreement through
pondering views, by probability, or by the thought, ‘This contemplative
is our teacher.’ When you know for yourselves that, ‘These qualities are
unskillful; these qualities are blameworthy; these qualities are
criticized by the wise; these qualities, when adopted and carried out,
lead to harm and to suffering’—then you should abandon them.

“What do you think, Kālāmas? When greed arises in a person, does it
arise for welfare or for harm?”

“For harm, lord.”

“And this greedy person, overcome by greed, his mind possessed by greed,
kills living beings, takes what is not given, goes after another
person’s wife, tells lies, and induces others to do likewise, all of
which is for long-term harm and suffering.”

“Yes, lord.”

“Now, what do you think, Kālāmas? When aversion arises in a person, does
it arise for welfare or for harm?”

“For harm, lord.”

“And this aversive person, overcome by aversion, his mind possessed by
aversion, kills living beings, takes what is not given, goes after
another person’s wife, tells lies, and induces others to do likewise,
all of which is for long-term harm and suffering.”

“Yes, lord.”

“Now, what do you think, Kālāmas? When delusion arises in a person, does
it arise for welfare or for harm?”

“For harm, lord.”

“And this deluded person, overcome by delusion, his mind possessed by
delusion, kills living beings, takes what is not given, goes after
another person’s wife, tells lies, and induces others to do likewise,
all of which is for long-term harm and suffering.”

“Yes, lord.”

“So what do you think, Kālāmas: Are these qualities skillful or
unskillful?”

“Unskillful, lord.”

“Blameworthy or blameless?”

“Blameworthy, lord.”

“Criticized by the wise or praised by the wise?”

“Criticized by the wise, lord.”

“When adopted and carried out, do they lead to harm and to suffering, or
not?”

“When adopted and carried out, they lead to harm and to suffering. That
is how it appears to us.”

“So, as I said, Kālāmas: ‘Don’t go by reports, by legends, by
traditions, by scripture, by logical conjecture, by inference, by
analogies, by agreement through pondering views, by probability, or by
the thought, “This contemplative is our teacher.” When you know for
yourselves that, “These qualities are unskillful; these qualities are
blameworthy; these qualities are criticized by the wise; these
qualities, when adopted and carried out, lead to harm and to
suffering”—then you should abandon them.’ Thus was it said. And in
reference to this was it said.

“Now, Kālāmas, don’t go by reports, by legends, by traditions, by
scripture, by logical conjecture, by inference, by analogies, by
agreement through pondering views, by probability, or by the thought,
‘This contemplative is our teacher.’ When you know for yourselves that,
‘These qualities are skillful; these qualities are blameless; these
qualities are praised by the wise; these qualities, when adopted and
carried out, lead to welfare and to happiness’—then you should enter and
remain in them.

“What do you think, Kālāmas? When lack of greed arises in a person, does
it arise for welfare or for harm?”

“For welfare, lord.”

“And this ungreedy person, not overcome by greed, his mind not possessed
by greed, doesn’t kill living beings, take what is not given, go after
another person’s wife, tell lies, or induce others to do likewise, all
of which is for long-term welfare and happiness.”

“Yes, lord.”

“What do you think, Kālāmas? When lack of aversion arises in a person,
does it arise for welfare or for harm?”

“For welfare, lord.”

“And this unaversive person, not overcome by aversion, his mind not
possessed by aversion, doesn’t kill living beings, take what is not
given, go after another person’s wife, tell lies, or induce others to do
likewise, all of which is for long-term welfare and happiness.”

“Yes, lord.”

“What do you think, Kālāmas? When lack of delusion arises in a person,
does it arise for welfare or for harm?”

“For welfare, lord.”

“And this undeluded person, not overcome by delusion, his mind not
possessed by delusion, doesn’t kill living beings, take what is not
given, go after another person’s wife, tell lies, or induce others to do
likewise, all of which is for long-term welfare and happiness.”

“Yes, lord.”

“So what do you think, Kālāmas: Are these qualities skillful or
unskillful?”

“Skillful, lord.”

“Blameworthy or blameless?”

“Blameless, lord.”

“Criticized by the wise or praised by the wise?”

“Praised by the wise, lord.”

“When adopted and carried out, do they lead to welfare and to happiness,
or not?”

“When adopted and carried out, they lead to welfare and to happiness.
That is how it appears to us.”

“So, as I said, Kālāmas: ‘Don’t go by reports, by legends, by
traditions, by scripture, by logical conjecture, by inference, by
analogies, by agreement through pondering views, by probability, or by
the thought, “This contemplative is our teacher.” When you know for
yourselves that, “These qualities are skillful; these qualities are
blameless; these qualities are praised by the wise; these qualities,
when adopted and carried out, lead to welfare and to happiness”—then you
should enter and remain in them.’ Thus was it said. And in reference to
this was it said.

“Now, Kālāmas, one who is a disciple of the noble ones—thus devoid of
greed, devoid of ill will, undeluded, alert, and resolute—keeps
pervading the first direction {[}the east{]}—as well as the second
direction, the third, and the fourth—with an awareness imbued with good
will. Thus he keeps pervading above, below, and all around, everywhere
and in every respect the all-encompassing cosmos with an awareness
imbued with good will: abundant, expansive, immeasurable, free from
hostility, free from ill will. {[}And the same for compassion,
appreciative joy, and equanimity.{]}

(A 3.65)
\end{quotation}

It is interesting how the Buddha’s teaching that is often quoted as a
charter of how to make up one’s mind is then taken a step further and
used to lay out a strong foundation for practice.

This final sutta is a short one that might sound a bit obscure as I read
it. Basically what it concerns is the result of \emph{kamma}, based on
scriptural teachings. One of the traditional ways of dealing with the
fruits of bad \emph{kamma}—dealing with some kammic obstacle or
difficulty—is the commitment to skillful \emph{kamma}, building up the
bank account of good \emph{kamma}. This is a sutta that gives a
perspective on and certainly encourages this.

\begin{quotation}
“Bhikkhus, I do not say that there is a termination of volitional kamma
that has been done and accumulated so long as one has not experienced
{[}its results{]}, and that may be in this very life, or in the
{[}next{]} rebirth, or on some subsequent occasion. But I do not say
that there is making an end of suffering so long as one has not
experienced {[}the results of{]} volitional kamma that has been done and
accumulated.

“This noble disciple, bhikkhus, who is thus devoid of longing, devoid of
ill will, unconfused, clearly comprehending, ever mindful, dwells
pervading one quarter with a mind imbued with loving-kindness, likewise
the second quarter, the third quarter, and the fourth quarter. Thus
above, below, across, and everywhere, and to all as to himself, he
dwells pervading the entire world with a mind imbued with
loving-kindness, vast, exalted, measureless, without enmity, without ill
will. He understands thus: ‘Previously, my mind was limited and
undeveloped, but now it is measureless, and well developed. No
measurable kamma remains or persists there.’”

(A 10.219)
\end{quotation}

What is being referred to is that no measurable \emph{kamma} will come
of this. None will persist. With the liberation of the mind with
loving-kindness, the kammic potential of that attainment will take
precedence over any other \emph{kamma} and generate fortunate results
and rebirth.

This is actually called “heavy \emph{kamma},” but it’s \emph{kamma} that
takes precedence over unskillful \emph{kamma}. Taking the life of
parents and killing an arahant: these are called “heavy \emph{kamma}” on
the unskillful side.

However, on the skillful side, heavy \emph{kamma} is such as the
attainment of jhāna, and the attainment of the liberation of the mind by
the brahmavihāras. They take precedence. Stream-entry, paths and fruits,
are heavy \emph{kamma}s because they take precedence over other forms of
\emph{kamma}.

\begin{quotation}
“What do you think, monks. If a man, from his boyhood onwards, were to
develop the liberation of the mind by loving-kindness, would he then do
an evil deed?”

“He would not, Lord.”

“And not doing any evil deed, will suffering afflict him?”

“It will not, Lord. How could suffering afflict one who does no evil
deeds?”

“Indeed, monks, the liberation of the mind by loving-kindness should be
developed by a man or a woman. A man or a woman cannot take their body
with them and depart; mortals have consciousness as the connecting link.

“But the noble disciple knows: ‘Whatever evil deeds I did before with
this physical body, their results will be experienced here and they will
follow me along.’

“Loving-kindness, if developed in such a way, will lead to the state of
non-returning, in the case of a monk who is established in the wisdom
found here in this teaching, but who has not penetrated to a higher
liberation.”

(A 10.219)
\end{quotation}

So there is the recognition of the strength of loving-kindness as the
ability to undermine other tendencies within the kammic potential.

\begin{quotation}
“If, monks, liberation of the mind is developed and cultivated,
frequently practiced, made one’s vehicle and foundation, firmly
established, consolidated, and properly undertaken, eleven benefits may
be expected. What eleven?

\begin{enumerate}
\def\labelenumi{(\arabic{enumi})}
\tightlist
\item
  “One sleeps well; (2) one awakens happily; (3) one does not have bad
  dreams; (4) one is pleasing to human beings; (5) one is pleasing to
  spirits; (6) deities protect one; (7) fire, poison, and weapons do not
  injure one; (8) one’s mind quickly becomes concentrated; (9) one’s
  facial complexion is serene; (10) one dies unconfused; and (11) if one
  does not penetrate further, one fares on to the brahmā world.”
\end{enumerate}

(A 11.15)
\end{quotation}

Even if one does not realize some state of liberation, those are all
good things. I don’t see anything that is a drawback here at all.

I’d like to plant some seeds to encourage you to go back to the sources.
The Buddha is the root. Sometimes the language is a bit daunting and the
repetition is boring, but there is this extraordinary treasure trove of
teachings and liberating insight, and the Buddha’s tremendous ability to
see things clearly. It is rare in the world and certainly rare in the
world of religious teachings. There is a great blessing in being able to
tap into the teachings that we have inherited through our good fortune.

We can take this next period to sit, bringing those thoughts of
loving-kindness to mind, reminding ourselves of the various
formulations—whatever resonates. One of the words used to describe the
brahmavihāras when one is accomplished in these meditations is
“immeasurable.” The heart is not bound by constricted, contracted
small-mindedness. One sets that aside and allows the heart to shine
forth in that which is beautiful.

In the teachings, there are what are called \emph{vimokkhas}. One of the
\emph{vimokkhas} (liberations or emancipations) is liberation through
the beautiful. This can be a meditative attainment in terms of color,
the \emph{kasiṇas}, which are exceedingly beautiful. But it can also be
a state of mind like loving-kindness, which is a beautiful mind-state
that allows the mind to drop its self-concern, worries, fears,
anxieties, and comparisons and dwell in that which is truly beautiful.
It is not really about me being anything or getting anything. It’s the
beauty of being able to put it all aside. Dwell in that wish of
loving-kindness, to be expansive, abundant, and immeasurable. We can
take this time to cultivate some of these qualities.
