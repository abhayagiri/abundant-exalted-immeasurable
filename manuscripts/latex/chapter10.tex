\chapter{Guided Mettā Meditation}

As we cultivate this meditation on loving-kindness, a variety of
formulations can be used. Some of them can be quite complicated and some
quite simple. What is most important is to recognize how it works, how
it affects us. What is it that resonates and helps bring into being a
feeling and perception of loving-kindness, an interest in the wish for
well-being in oneself and others?

As a meditation, the use of words and phrases are helpful ways of
establishing mindfulness or recollections. We can use a phrase to plant
that seed within the heart and mind. Because the nature of the mind is
to come up with different thoughts and, usually, with stories (not just
short, little sentences here or there), we use those phrases to help
keep the mind on track: “Oh right, I’m doing loving-kindness
meditation.”

So using those phrases, perhaps from the Buddha’s discourse on
loving-kindness—“Wishing in gladness and in safety, may all beings be at
ease” or “May I be well, happy, and peaceful”—is a way of keeping the
mind on track and reminding it what it’s intending to do. So that is
actually the meditation; that is how we train the mind.

In Thailand, one of the most common meditation methods or techniques is
using the word “\emph{buddho},” the name of the Buddha. “\emph{bud-}” on
the in-breath and “\emph{-dho}” on the out breath: just that simple
word. If you want, you can use the word “mettā”: “met” on the in-breath
and “tā” on the out-breath. Use that word to keep the mind on track.
Tune in to the feeling, the connotations of mettā you have reflected on.
This helps the mind to channel its attention and energy to the pleasing
emotional tone of well-wishing.

In meditation practice, mettā has the very useful function of providing
emotional moisture. If you think, “\emph{Anicca, dukkha, anattā, anicca,
dukkha, anattā, anicca, dukkha, anattā},” it gets pretty dry after a
while. “Impermanence, unsatisfactoriness, not-self, impermanence,
unsatisfactoriness, not-self” are very true. However, the mind has to be
prepared and balanced. Mettā-bhāvanā can be a very useful basis for
providing a sense of moisture, warmth, and brightness to the mind.

The opposite is true as well. If you really see impermanence,
unsatisfactoriness, and not-self, one of the sure indications that the
insight is a true one is the sense of universal kindness and
well-wishing that arises. You think, “Wow, everyone is in the same
boat.” Then there is a sense of caring, the heart’s natural response to
a deep insight.

Usually, we don’t start with deep insights. It’s good to sprinkle in a
bit of mettā to lubricate the insights and heart so that it’s ready,
ripe, and able to respond. Also, as we continue in this particular
spiritual practice, it is quite natural to become inured or accustomed
to the particular pattern of reflection, investigation, and practice we
are doing. So having supportive practices helps create a broad base for
practice and insight to ripen. It gives us the energy for practice.

Today I’d like to introduce as a meditation, the chant, “Reflections on
Universal Well-Being,” which are all four brahmavihāras. Almost half of
the chant is specifically about mettā, loving-kindness. I think that is
quite appropriate in that mettā is really the doorway into the other
brahma vihāras. As we are able to establish the wish for well-being,
this loving-kindness lays the foundation for the qualities of
compassion, sympathetic joy, and equanimity. To try to establish
equanimity or sympathetic joy on its own is not that easy, but
loving-kindness is an appropriate doorway into these sublime abidings.

The chant begins with a very direct statement: in Pali, \emph{ahaṃ
sukhito homi}, which is “May I be happy.” Of course, if you’re chanting
“May I be happy,” it falls a bit flat. Whereas, if you begin a chant,
“May I abide in well-being,” it’s kind of nice to chant.

“May I be happy” is a very precise translation of \emph{ahaṁ sukhito
homi}, but I think it is appropriate to take a bit of poetic license, to
make it a bit more chantable: “May I abide in well-being.” For me,
having chanted this for decades now, that is what happens when I think
of the words. When I use it as a meditation, it comes out as a chant.
It’s kind of hard for me just to say it.

It’s a very beautiful chant. Establish that wish: “May I abide in
well-being, may I abide in happiness, may I abide in the basis of all
that is pleasurable and delightful.” The wish directs attention to
oneself. That is very much the classical mode of the cultivation of
loving-kindness: to direct attention inward and establish that very
strong feeling of well-wishing towards oneself.

In the scriptures, there is a very lovely exchange between King Pasenadi
and Queen Mallikā. He was the king of Kosala, in which is Sāvatthi,
where the Buddha spent the majority of his Rains residences in Jeta's
Grove and elsewhere in the region. So he was very familiar with King
Pasenadi. Queen Mallikā was a devout Buddhist and had a solid practice.
King Pasenadi had tremendous faith in the Buddha, but there is never any
mention of him having attained any levels of liberation.

In any event, in the palace, King Pasenadi asks Queen Mallikā, “Who is
it that you cherish the most?” Queen Mallikā replies, “Well, I cherish
myself the most.” “Hmmm,” he thinks. He always takes her lead. He has
tremendous love and respect for her, so there must be something to this.

Pasenadi goes to the Buddha and recounts the conversation. The Buddha
says, “That’s exactly what one’s attitude should be and that’s how it is
in reality. It’s when we cherish and honor ourselves fully that we can
really look after and cherish others. We need to have that foundation.”
(S 3.8) It isn’t a selfish obsession with “me.” In reality, we do need
to cherish ourselves.

If I eat something, nobody else gets full. We relate to ourselves first
on a cellular level. So, on an emotional level, we need to attend to
ourselves in appropriate and skillful ways. When we look after
ourselves, attend to our well-being, that is when we can be a refuge and
be there for others. As human beings, we’re not all that different,
despite all of our assumptions about our fascinating uniqueness. We’re
not very different, so when we tune in and understand ourselves, we can
be really present for others. It’s our unskillful obsession with
ourselves that blocks us off from others.

So, establish that sense of “May I abide in well-being.” Take that in
and open the heart to the sincere wish, “May I abide in well-being and
freedom from affliction.” In Pali the phrase is \emph{niddukkho homi},
without dukkha, without suffering. \emph{Avero homi} is “In freedom from
hostility.” \emph{Vera} is conflict and also grudges—the grudges and
conflicts that ironically bind us together. In Thai, there is a phrase
that you say when something goes wrong. In English, we say, “Oh, shoot.”
In Thai, it’s “\emph{wain gam},” “Oh, this is conflict and kamma.”
Establish the wish, “May I abide in freedom from hostility,” in freedom
from any kind of conflict.

“\emph{Abyāpajjho homi}”: freedom from ill will. \emph{Byāpāda} is ill
will, anger, irritation, aversion. “May I abide in freedom from ill
will.” \emph{Anīgho homi}: in freedom from anxiety, worry, or fear.
Establish that wish, “May I abide in freedom from anxiety.”

\emph{Sukhī attānaṃ pariharāmi}: \emph{attā} is me, self. “May I be
happy, and may I maintain well-being in myself.” Draw attention to that
wish for freedom from affliction, hostility, ill will, and anxiety, and
that wish to maintain well-being. Establish yourself as grounded in
well-being. As you develop this as a meditation, going through the
phrases, you can repeat it many times until it feels solid and you feel
grounded in well-being.

Then allow it to be a bit more non-directional: not directed toward
yourself, but a bit more pervasive: “May everyone abide in well-being.”
In the Pali, \emph{sabbe sattā sukhitā hontu}: “May all beings abide in
well-being; may all beings be happy.” Then proceed to similar
recollections or aspirations. \emph{Sabbe sattā averā hontu}: “May all
beings abide in freedom from hostility, ill will, and anxiety.” Those
feelings of fear, worry, and anxiety are part of not just the human
condition, but the universal condition of having a life force, whether
it’s a human being, a mammal, or an insect. As soon as you have a life
force, you’re protecting it. Every being wants to get away from pain or
any kind of threat.

Of course, there is going to be some kind of threat. Living in the
forest, animals are constantly attuned to danger. Abhayagiri is in the
forest, and there are lots of animals there. Deer, particularly the
does, don’t have any clear physical means of aggression, but they have
big ears and they are listening. They have big eyes and they’re on the
lookout for anything that might be a bit threatening. There is anxiety
there. That is the nature of being born into a physical body: we are
vulnerable. We can obsess on this and keep spinning ourselves out with
anxiety.

There is a book by Robert Sapolsky, \emph{Why Zebras Don’t Get Ulcers}.
If there is danger and a lion attacks, zebras are out of there as
quickly as they can. They might even be mauled or see a friend mauled,
but as soon as it’s over, they’re out there grazing again. Human beings,
on the other hand, are constantly planning about the next perceived
threat, which may or may not happen. And we end up with ulcers. It’s a
response that serves us well in certain circumstances, but when it’s
applied across the board, it doesn’t really serve us that well. Dwelling
in anxiety is exceedingly uncomfortable.

On the sheet of paper I passed out on the sharing of loving-kindness,
there is a very similar formulation, with a bit of a twist that is very
nice. But it uses those same structures and is very classical. It is out
of a very early commentarial treatise, the \emph{Paṭisambhidāmagga},
which is so old it’s in the Pali Canon. These are the formulations for
one of the ways for the cultivating of loving-kindness: developing
freedom from hostility, freedom from ill will, freedom from anxiety, and
then the wish for happiness. So that as we cultivate this recitation and
recollection, we cultivate the feeling, “May I be well, happy, and
peaceful. And just as I wish to be well, happy, and peaceful, may all
beings be well, happy, and peaceful.”

This is recognizing that that wish is not just isolated or focused
solely on oneself. “In the same way that I wish to be well, happy, and
peaceful, may all beings experience that. May I be free from animosity.
As I wish to be free from animosity, may all beings be free from
animosity. May I be free from any kind of suffering. And as I wish to be
free from any kind of suffering, may all beings be free from any kind of
suffering. May I live in peace and happiness. And as I wish to live in
peace and happiness, may all beings live in peace and happiness.”

These are helpful phrases or structures. It’s helpful to become
knowledgeable about Pali, the scriptural language, because sometimes our
own English language has a backlog of associations and habit patterns
that affect our thoughts and condition our reactions. As we delve into
Buddhism, the Pali language, and not just these phrases, can be very
helpful.

The only function the Pali language has performed over the last 2,300
years or so is to maintain the discourses of the Buddha. It’s a
technical language, and the Buddha probably spoke something very
similar—there’s controversy about that. Pali is a mid-Indian, ancient
dialect that the scriptures were written down in. So even if it wasn’t
exactly what the Buddha spoke at that time, it’s very, very close. And
its only function has been to pass on the words of the Buddha, the
scriptures, and the teachings. It’s a technical language geared to
spiritual themes.

As I’ve been mentioning over the past couple of days, “kusala” and
“akusala” are technical terms that point to a particular state of mind
that is wholesome, skillful, and beneficial and the opposite,
respectively. They have particular parameters. Taṇhā is a specific
desire rooted in greed, hatred, delusion, and non-recognition of the way
things truly are. When the Buddha says taṇhā is the source of suffering,
he’s not saying desire, as a generic term, is the source of suffering.
Rather, it’s a specific form of desire. So, it’s helpful sometimes, as
an exercise, to familiarize yourself with some of the basic terms. Then
things start to leap out.

As you consciously cultivate the quality of loving-kindness, bring
attention to “May I abide in well-being,” trying to establish that wish
firmly and clearly towards yourself within the heart. But it doesn’t
just affect you; it transforms you and, by transforming you, it has an
effect on the world and people around you. After a week of
loving-kindness practice and training, go back home or to your workplace
and take note of anything you find different there. Or does someone say
in disbelief, “You’ve come back from a week of loving-kindness?”

There is a story in the \emph{Visuddhimagga}, a commentarial treatise,
about a man from India who hears about Sri Lanka and how there are lots
of people practicing meditation there. This is almost 1,000 years after
the Buddha passed away. Big changes happened in India at that time, and
Buddhism was flourishing in Sri Lanka. He has the intention to take
ordination, so he gives up his business, takes leave of his family,
travels to Sri Lanka, and takes ordination. He is interested in
loving-kindness meditation, so he gets instruction and goes to live in a
place where many meditators are practicing loving-kindness meditation.

He lives in a particular hut in the forest, surrounded by a nice grove
of trees. He is very assiduous in his cultivation of loving-kindness
meditation and practices diligently for four or five months. Then a wish
comes into his mind, “Maybe I’ll go off to another monastery and
practice with such and such a teacher.”

With the formulation of that wish, he hears sobbing outside his dwelling
place. So, he goes out and there is this deity who lives in a big tree
nearby. He asks what is wrong, and the deity says, “You just had the
thought that you want to leave, and it makes me sad. While you’ve been
here practicing loving-kindness, I’ve been happy and all of the beings
around here have been happy. If you leave, all the other devas and
non-human beings are going to go back to arguing and quarreling, just
like they used to do. While you’ve been here practicing, everything has
been peaceful.”

So he stays longer and continues with his practice. In good Buddhist
fashion, this happens three times. Finally, he gives up his
determination to go anywhere and stays there and attains enlightenment,
which is what always seems to happen in the stories.

One could say it’s just a shaggy-dog story from the scriptures, but I
know of a monk who had something very similar happen. He really embodies
loving-kindness. He’s still living—his name is Ajahn Gunha, and he is a
nephew of Ajahn Chah. Now, he’s well into his sixties. He had some major
realizations when he was a junior monk. When he was a younger monk, he
practiced diligently, and the quality of loving-kindness is a
predominant feature of his being.

He was traveling by himself one time, before he established a regular
monastery. This was along the border of Burma, in an area where there
were big expanses of jungle but also a lot of insurgents. He was walking
through the jungle when he walked into a large encampment of Communist
insurgents. They immediately surrounded him with weapons and tried to
grill him: “How did he get in there?” They thought he was a government
spy because the whole area around their encampment was mined and he had
been able to walk through the minefields. They held him for a long time
and continued to question him. But eventually, when they realized he had
been telling the truth and had indeed walked through their minefields
without having been blown up, they warmed up to him and looked after him
and fed him.

After some time, they were convinced that he was a genuine monk and that
they should let him go. They accompanied him, seeing him out through the
minefields, and set him on his way. Then they came back and there was a
realization: “Why did we let him go? It was so nice having a monk here.
Everybody was living harmoniously and happily. He was such a wonderful
presence for us.” So they tromped out after him and invited him to come
back. They would look after him for the Rains Retreat. He had already
made a commitment to spend the Rains in another place but expressed
appreciation for their offer. So, that is that presence of
loving-kindness. “While he was here, everything felt good.”

So, let’s take some time just to sit and use these phrases, this
particular structure. Establish ease, alertness, and comfort in the
posture. Allow the breathing to be natural, easeful. Relax any kind of
tension. Allow the breath to come in very, very clearly, without any
obstruction. Allow the breath to go out without any obstruction, not
controlling, just feeling very comfortable and at ease.

As you physically feel at ease, attention and mindfulness are present.
Then formulate the wish, “May I abide in well-being.” Allow the breath
to come in and go out. Allow the heart to resonate around that sincere
wish, “May I abide in well-being. May I abide in happiness. May I abide
in freedom from affliction.” Allow that to resonate through the body and
the mind. “May I abide in freedom from suffering and discomfort—mental,
physical, any kind of dukkha.” Establish that wish toward yourself,
cherishing yourself with the sincere wish for freedom from anything that
is afflicting, obstructing.

Allow that wish to be an energy within the body and mind. Allow that
energy to flow through and permeate the whole body, the whole mind: that
sincere wish for happiness, freedom from suffering.

“May I abide in freedom from hostility.” Allow the mind to relinquish
and release any kind of hostile, aversive, or negative intention toward
yourself or others because the first person harmed when you formulate an
intention of hostility or aversion is you. Release and relinquish any
negativity with the clear intention or wish: “May I abide in freedom
from hostility. May I abide in freedom from ill will.” Again, not
setting up any kind of contention or conflict, put it all down and
direct that wish, feeling, and aspiration for well-being toward
yourself.

“May I abide in freedom from anxiety.” Recognize the inevitable tendency
toward fear, worry, and anxious wondering, but set the intention: “May I
abide in freedom from any kind of anxiety.” Recognize that it would be a
very different world for you and others around you if there was freedom
from such anxiety and fear.

“May I maintain well-being in myself.” Return to the wish, the
aspiration for stability of well-being, a steady sense of happiness that
comes when the mind is not overwhelmed by negative habits of mind.
Recognize that this is actually possible. It’s a noble aspiration to set
for yourself. It’s holding yourself dear, cherishing yourself in an
appropriate way.

Allow these aspirations and recollections, this loving-kindness, to
permeate and pervade the body and mind. As that is increasingly
well-established and feels stable, allow the focus to be broader. Expand
the focus so that it encompasses everything around you, inclusive of all
others. “May all beings abide in well-being. May everyone abide in
well-being. May everyone abide in freedom from hostility.”

We can’t force the world and all beings to live that way, but we can
have that aspiration. We can perceive the benefit. Hold that within the
heart. “May everyone abide in freedom from ill will.” We can see the
suffering and pain that arise from ill will, so develop that sincere
wish of compassion, friendliness, and care. Wish that all beings abide
in freedom from ill will.

Allow the heart to be open and expansive, wishing safety and security.
“May everyone abide in freedom from anxiety.”

“May all beings maintain well-being in themselves.” As we’re able to
establish that intention without division within the mind, we stop
separating anybody out into “us” and “them,” my group and that group.
Develop that expansive wish and aspiration, that everyone be able to
maintain well-being in him or herself. This allows the mind to break
down the barriers of how we habitually compare and divide, and suffer
because of it.

Cultivation of these phrases and tuning in to the feelings they generate
are helpful for the cultivation of a mood of loving-kindness, and they
also give us an insight into the way the mind works, how we cherish or
overlook ourselves, how we look at others, how we compare, how we judge.
In the loving-kindness meditation, when we allow the heart to abide in
kindness and well-being there is no judging or comparing.
