\chapter{Closing Ceremony}

First of all, I’d like to express my appreciation to Susan, Karen, and
Cassidy for pulling this retreat together and being willing to step into
the unknown: to see how to do it and take a chance on whether it
actually worked or not. I’ve enjoyed my time here. I express my
appreciation to all of you, the retreatants who came together and who
have been interested and willing to pursue this exploration of
loving-kindness. As I have tried to emphasize through this retreat,
loving-kindness doesn’t stand on its own. It is a part of a spectrum of
the Buddha’s teachings, both in terms of support of those teachings and
also as a result of those teachings.

I would encourage you all to take what you have learned during this time
back into the world—into the family, society, and workplace—and
experiment on how to apply these teachings, how to sustain them. Having
finished the retreat, whatever peacefulness, loving-kindness, and
clarity you have cultivated, please take it with you. You don’t need to
leave it all here. Take it, share it, and recognize that it is a gift to
other people as well. Sharing loving-kindness, virtue, and goodness:
these are suitable things to share with others, and the world is
desperately in need of them.

This simple ceremony that we have just done, the refuges and the five
precepts, is something that is quite ubiquitous in Buddhist tradition.
But I would encourage all of you to take it on board, investigate, and
see how these refuges and precepts can both be an anchor for day-to-day
life as well as the fruit of a skillful life well lived. Having a
refuge, having an anchor in clarity of conduct, allows that virtue and
holding to truth to be the basis of one’s life.

I mentioned in one of the Dhamma talks that one of the most common
formulations that the Buddha gives for the description of the
stream-enterer, the first fruit of realization and liberation, was the
stable confidence and faith in Buddha, Dhamma, and Sangha, and the
stable commitment to the precepts. It is easy to overlook that as being
ordinary. That’s one of the problems we have as humans beings, we
overlook the ordinary, being busy looking for something else. Of course,
our whole life tends to be looking for something else to satisfy us, and
we overlook the opportunities of satisfaction and contentment in the
present moment.

We look for refuges, look for things to solve all our problems, for
things to shore up our sense of well-being—just over there. And it is
usually just out of reach as well. In contrast, we can return to the
things that we normally might conceive of as ordinary and realize the
power of transformation that they hold: the power of the refuges, the
power of the precepts, and the power of loving-kindness.

On a certain level it is hackneyed and old: “May all beings be well and
happy.” You see it written everywhere in Buddhist texts, and you might
even sign your emails with something like that. But recognize that there
is a transformative power in it: an actual commitment and understanding
of how to apply it, how to work with it, how to rely on it, and how to
return to it allows it to be something that nourishes the heart and
nurtures your day-to-day life.

The whole week that we’ve been here, what I’ve been teaching and
emphasizing is pretty ordinary stuff. Certainly most people have heard
it or read it at some time or another. But I think it’s when we make a
commitment within ourselves to put it in to practice, give it the space
to see if it works, and have the confidence to stay with it for a while:
that’s when the transformation takes place. It isn’t as if somebody is
going to get a new bit of information that is going to transform them.
It is rather the old and the ordinary that is probably most useful.

Recognize, reflect, investigate, and allow those things to reverberate
in consciousness. Be willing to keep experimenting, applying, and seeing
what are the new ways of holding a particular perspective, practice,
mode of conduct, or way of training oneself.

As for myself, this is my thirty-fifth year of being a monk, and there
are not a whole lot of new things that I learn. But what I \emph{have}
learned is how to use the ordinary: the foundations that are so
important, the things that are the ground of the teachings and our
training. As we develop a deepening appreciation and a deepening skill
in how to use these, then we see some radical letting go take place,
some radical willingness to put a lot of the things down that we didn’t
want to be carrying around anyway.

So again, I would just like to express my appreciation to all of the
people who have helped out. Everybody helped out in this retreat in the
sense that, while you have been here, you pitched in; everybody made it
happen. That is worthy of appreciation. My own perception of things, and
what the organizers have told me, is that everything has gone very
smoothly. That is really appreciated when we come together and practice:
having a tangible example that when everybody pitches in, helps out, and
harmonizes, then life goes pretty smoothly. In the human realm, that’s
about as good as it gets.
