\chapter{Introduction to Mettā
Meditation}

When we reflect on loving-kindness and use it as a meditation, it’s good
to reflect on its foundation. One of the ways that Ajahn Sumedho
describes this is “not dwelling in aversion.” That’s a helpful way to
look at loving-kindness.

In the Noble Eightfold Path, there is \emph{sammā-saṅkappa}, right
intention or right thought. There are three aspects of this:
\emph{nekkhamma-saṅkappa}, the aspect of renunciation or not drifting
into sensuality; \emph{abyāpāda-saṅkappa}, not thinking with ill will or
aversion; and \emph{avihiṃsa-saṅkappa}, thoughts of non-harming. The
last two are often equated with loving-kindness and compassion.

For loving-kindness to arise, there have to be thoughts of non-ill will,
non-aversion. These thoughts are fundamentally simple. We often think to
ourselves, “I should be thinking these sublime thoughts of love for all
beings, everywhere.” But it’s a great start if you can just not get
averse to other beings. It’s easier to begin this way as well. You can
then encourage yourself along the way.

Encouraging ourselves is very important. That in itself is an act of
loving-kindness, encouraging ourselves in something skillful. Bhante
Guṇaratana tells a lovely story. He was going to teach in Europe, and
when it was getting close to the time that he was to leave, the person
who invited him called and asked him, “What do you teach?” He said, “I
teach mindfulness. I teach \emph{vipassanā}.” And she said, “You don’t
teach loving-kindness do you? I hate loving-kindness!”

In actual fact, loving-kindness is a major foundation for Bhante G’s
teachings. The reason why the sponsor hated loving-kindness was she felt
that she was supposed to have it for everybody. She had been through the
Second World War and, being Jewish, had lost many family members and her
culture. So she had the hope, “You aren’t going to get me to do
loving-kindness.”

However, when you reflect on loving-kindness, you realize that not to
dwell in aversion is an act of great kindness to yourself. The reality
is that the very first person to receive any of the loving-kindness you
are able to conjure up in your heart is you, whether it is directed to
yourself or not. Classically, the way the instructions were set up from
the time of the earliest commentaries on the Buddha’s teachings, when
loving-kindness meditation was structured and systematized, was to
direct loving-kindness towards yourself. You establish loving-kindness
towards yourself before going on to generate loving-kindness for
somebody else.

This has a very strong psychological foundation. You can’t really share
anything until your cup is full, or at least until there is something in
your cup. So you should direct attention to the cultivation and bring
into being thoughts of loving-kindness towards yourself. We can
complicate this with feelings of uncertainty, doubt, or guilt: “Maybe I
shouldn’t be doing it towards myself, because I’m not worthy.”

One of the monks in Australia, Ajahn Brahmavaṃso, teaches people that if
you can’t start with yourself, start with anything that evokes a feeling
of loving-kindness—a little puppy, a kitten, anything like that—because
in reality, the cultivation of loving-kindness is not the repeating of
the words and phrases. It’s about the actual experience of
loving-kindness, warmth, acceptance, openness—the feeling tone of the
heart, including concern for the happiness of yourself and others. It’s
about generating that feeling.

In terms of meditation, it’s directing attention to the feeling or
emotion of kindness and well-wishing and then finding ways to support
and shore that up, to allow it to become stable, suffuse your own being
with it, and spread it out. That requires mindfulness and attention,
which is one of the reasons why I’m only introducing loving-kindness on
the afternoon of the second day of teaching.

It’s important to see that loving-kindness has a context of the refuges,
precepts, mindfulness, and attention to the body, as well as faith,
confidence, and effort. It is necessary to cultivate and develop all the
spiritual faculties. A particular skillful and wholesome quality then
arises, which you can then draw your attention to, while shining it
forth.

When loving-kindness has a stable base, you can allow it to shine more
brightly. Then if, for some reason, the feeling fades, you won’t say to
yourself, “Oh, I’ve completely blown it. I came on a mettā retreat and I
can’t keep the mettā going at all.” That’s not really the point. The
point is that there is a comprehensive spiritual path to be cultivated.
Loving-kindness is an opportunity or an option that we can direct
attention to and see where it goes. We bring attention to the different
faculties of our spiritual cultivation in meditation and, having built
that base, we can then recognize that there is an opportunity to home in
on a particular mature emotion and allow it to come forward.

This is very powerful, useful, and healing, but it’s also only one of
many aspects of the meditation. It’s not a matter of succeeding or
failing: it’s seeing the opportunity that we have of cultivating the
path and then seeing what comes up when we direct attention to
loving-kindness. We can set our intention to the cultivation of
loving-kindness and then start to be flooded by memories of something
very painful. Perhaps there can be a lot of fear or aversion mixed in,
but we can then direct attention to non-aversion: “Can I not get hooked
into aversion and ill will?” Any practice that opens doors into the
heart might result in not being sure what you find there, but they’re
your own doors. It’s learning what is behind these doors that is
important.

Again, this is one of the reasons why it’s so important to have the
foundation of precepts, generosity, restraint, reflection, and
investigation. This gives you the solidity and stability to be able to
not be shaken by what appears out of the doorways of the mind. One of
the popular images from Ajahn Chah’s teaching is the pond in the forest
and how many wonderful creatures come to drink at that pond. It’s quite
a benign image, with nice, little bunny rabbits and furry creatures, but
that’s not all that comes to the pond. Aggressive ones with big teeth
might like to drink there, too. This is nature as it is, and the mind is
nature. Recognizing that it’s all a part of nature, nothing is a
problem.

This is also a wholesome result of loving-kindness practice. As we
cultivate loving-kindness more and become grounded in it, the perception
that nothing is a problem becomes clearer and clearer, and we are always
established in well-wishing.

It’s not that we \emph{need} to wish for others’ happiness and
well-being. This is just the way it is at a cellular level. This is the
way the heart responds when it’s not being self-protective, when it’s
not buying into its complications. We gain confidence in this. Doing
loving-kindness as a practice makes this very conscious. As it is
cultivated, it becomes second nature.

I think of Ajahn Chah as a stellar example of somebody with
loving-kindness. People wanted to draw close to him because of that
kindness. I remember once we were walking around Ajahn Chah’s monastery,
Wat Pah Pong, with Ajahn Liem, after Ajahn Chah had passed away or maybe
around the end of his life. Ajahn Liem, an excellent monk and teacher,
said, “Ajahn Chah had such loving-kindness. That’s why so many people
wanted to be with him, come close to him. That’s why I’ve opted for
equanimity.” This loving-kindness creates a lot of work! People who have
met Ajahn Liem know he has that air of equanimity. He’s in the midst of
everything and his attitude is, “Oh, it’s just the world.”

In consciously cultivating loving-kindness, we use phrases. One of the
ways of doing the meditation is to use a phrase and then let it resonate
for a bit—such as the phrase of sharing loving-kindness, “May I be well,
happy, and peaceful.” Set an intention of well-wishing and then allow it
to resonate for a bit and settle in the body. Loving-kindness is not
separated from what you are feeling in the body. Relax and settle. You
are using the phrases as a meditation tool, a \emph{parikamma}, a
repetition that helps the mind to focus.

The articulation of loving-kindness in thought or verbalizing it helps
you to see: this is the present moment. If this is the thought, then
what’s the feeling? You can then draw attention into the heart. What’s
the feeling? Relax, paying attention to the body and using the breathing
to soften.

I’ll do this as a guided meditation. Then it will be something you can
use on your own as a practice, at your own speed. I will do it at the
speed that I feel comfortable with when teaching a group, which doesn’t
necessarily mean that is how I do it when I’m on my own. To do the
meditation, tune in to the body, paying attention to and relaxing your
posture, allowing the breathing to settle deeply—tapping in to the
rhythm of the breathing and planting a seed within that sphere of
relaxation. Then pay attention to the feeling.

I can’t emphasize enough that \emph{mettā-bhāvanā}, the development of
loving-kindness, is not about getting proficient at memorizing the lines
or coming up with really neat new lines: “the revolutionary art of
cultivating loving-kindness, new and improved!” Pay attention to the
feeling and then allow that feeling to permeate, suffuse, and spread
through the body, mind, and heart.

At the beginning there will just be little flickers of loving-kindness
and the feeling that we associate with a sense of warmth, kindness, or
loving attention. That’s fine. Recognize how to nurture that. The way of
nurturing it is using phrases that seem meaningful. When Jayantā asked
me if there was anything such as copies of different loving-kindness
phrases she could bring, I said yes. Some of the phrases might resonate
with you, and some might not. There is a whole array of different
formulations. Become familiar with what actually resonates within the
heart.

It’s the nature of the mind that once you have something that works
quite nicely and you use it for a while, the mind gets inured to it.
It’s then helpful to come at it from another angle and use other
phrases, keeping it fresh.

The bringing up of the phrases is an act of mindfulness; being attentive
and following a sequential pattern requires a certain mindfulness,
attention, consistency, and application of mind. That is part of the
cultivation. Bringing up and sustaining loving-kindness takes patience
in using and repeating the phrases and techniques.

When loving-kindness appears, it is of course important not to grab on
to it. Don’t take hold of this feeling of loving-kindness and
desperately try to make it stable and steady. That is like taking a tiny
little bird in your hand and squashing it. As with any other meditation
object, you have to hold it very lightly, and if it stays, fine. It’s
like shoring or propping it up and allowing it to grow. It’s very
difficult to force that feeling.

As you direct attention to positive thoughts of well-being and
well-wishing, it’s important not to try to fend off any negative
thoughts. If negative thoughts of irritation and aversion come in,
establish the mind in non-ill will towards them. There is a whole series
of discourses in which the Buddha is sitting in the forest and Māra, the
evil one (Buddhism’s Lucifer and the embodiment of the forces of
darkness) comes and whispers in the Buddha’s ear: “You’re not really
enlightened. It’s just a cop out,” or “You really should be taking more
responsibility.”

The Buddha’s response is, “I know you, Māra.” That’s all it takes for
Māra to pack up his gear; he’s done for. It’s the same for our own
minds. The forces of Māra come whispering, “This is unbearable \ldots{}” The
response is, “I know you Māra.” We don’t have to destroy, annihilate, or
get rid of the ill will. As soon as it’s seen clearly, the forces of
Māra have no ammunition or traction. There’s nothing they can do.

This is a personification of what goes on in our own minds. If things
come up in the mind, whether it’s something we’re really attracted to,
want to distract ourselves with, or doubtful about, it’s really helpful
to say, “I know you Māra.” And then bring constancy to the cultivation
of loving-kindness or whatever meditation we are doing. Today we are
using loving-kindness, but this really applies to our whole practice.

As you use the phrases, experiment with them. As you continue your
practice, experiment and see what is helpful for sustaining the feeling
of well-wishing. It’s like lighting a fire. You have a lighter or a
match, and then you have a small flame. You light some paper, get some
kindling and some bigger wood, and you can build a bigger and bigger
fire.

You do the same thing within the heart. You have a small flame of “May I
be well.” Just attend to that, nurture it, and protect it from the wind.
Slowly get more fuel for that positive feeling. Start with a small spark
and then allow it to gain momentum.

That’s the quality we bring into the practice, that intention of
nurturing, of protecting, of bringing into being something that is very
wholesome, positive, and precious. Take care to allow yourself to do
that. Sometimes we don’t give ourselves the time or space simply to
allow a feeling of warmth, vulnerability, care, or security to establish
itself. It’s a curious aspect of the human condition, but there it is.
It’s a great gift to be able to give ourselves that time and space.

I’m going to do a guided meditation using particular phrases for the
sharing of loving-kindness. This is quite a classical way of doing it.
The phraseology might not be classical, but the approach is classical in
the sense of directing loving-kindness towards yourself, your parents,
and your teachers—the people who are most \emph{kammically} important
and binding. Then start to spread it out to your family, friends, and
then even those who are unfriendly because loving-kindness starts to be
non-discriminative. Even those who are unfriendly—they’re suffering, and
we wish for them to be happy. Then extend loving-kindness out to all
living beings without distinction.

\section{Guided Meditation}

Again, begin by settling into the posture, relaxing any tension you are
feeling in the shoulders, around the eyes, the jaw. Allow the breath to
settle into the abdomen, letting yourself be very comfortable. Allow a
spaciousness in the mind. Then think the thought, “May I be well, happy,
peaceful and prosperous.” A sense of well-wishing toward yourself,
emotionally and materially: “May no harm come to me. May no difficulties
come to me. May no problems come to me.”

On a certain level, we know that there is always going to be some
difficulty, some problems in life. That’s the way things are. But just
set the wish, “May no harm come to me, no difficulties, no problems.”
Allow that sense of ease and well-being that comes when you do not have
to deal with anything like that in the moment.

“May I always meet with success.” I think I’d like to add a word to
that: “May I always meet with \emph{spiritual} success.” The sense is,
“I have the wish to cultivate that which is truly good.” That’s a
spiritual aspiration. “May I have success in my spiritual endeavors.”
That’s a true aspiration.

“May I have the patience, courage, understanding, and determination to
meet and overcome inevitable difficulties, problems, and failures in
life.” That well-wishing encompasses the reality that there are
difficulties and problems. There is always going to be some sort of
failure or another, but may I have the qualities that will allow me to
see those through. May I have patience.

Ajahn Chah once said, “I don’t actually teach very much at all. I just
teach people to be patient.” That patience is courage, a willingness to
be present with things, whether they are pleasant or unpleasant. There
is a curiosity to want to understand and see things through, and a
determination and willingness to stick with things.

I think all of us, if we’ve taken on spiritual practice and stuck with
it for a long enough time—certainly I can vouch for the monastics—are
all pretty stubborn, actually. We need to allow that stubbornness to
become discerning determination. We have the opportunity to do something
really skillful and wholesome and to stick with it. These are the
qualities that allow us to pass through things. That’s an act of
kindness. Allow it to be grounded in loving-kindness.

I’d like to repeat the phrases again. It’s important that we don’t just
skip to “May everybody else receive loving-kindness.” It’s really
important that we’re willing to take the time to direct it towards
ourselves and allow it to become established within our own hearts. “May
I be well, happy, peaceful, and prosperous. May no harm come to me. May
no difficulties come to me. May no problems come to me. May I always
meet with spiritual success. May I have the patience, courage,
understanding, and determination to meet and overcome inevitable
difficulties, problems, and failures in life.”

Then direct these thoughts of well-wishing, that sense of gratitude, to
one’s parents, particularly if they are still alive, but even if they’ve
passed away. Even if you have had a difficult relationship with your
parents, at least you’ve made it to this point. That’s a big deal. “May
my parents be well, happy, peaceful, and prosperous. May no harm come to
them. May no difficulties come to them. May no problems come to them.
May they always meet with spiritual success. May they always have the
patience, courage, understanding, and determination to meet and overcome
inevitable difficulties, problems, and failures in life.”

Now direct attention of loving-kindness towards your teachers, whether
it’s teachers in worldly skills and knowledge or your spiritual
teachers. We are able to cultivate a path of spiritual development
because of our relationship to teachers that we have had and still have.
It’s an important connection to be grateful for and to honor and delight
in. “May my teachers be well, happy, peaceful, and prosperous. May no
harm come to them. May no difficulties come to them. May no problems
come to them. May they always meet with spiritual success. May they also
have the patience, courage, understanding, and determination to meet and
overcome inevitable difficulties, problems, and failures in life.”

Allow attention to be directed towards family, whether it’s spouses,
partners, children, aunts, uncles, cousins, nieces, nephews, or anyone
we’re related to by reason of \emph{kamma}. “May my family be well,
happy, peaceful, and prosperous. May no harm come to them. May no
difficulties come to them. May no problems come to them. May they always
meet with spiritual success. May they also have the patience, courage,
understanding, and determination to meet and overcome inevitable
difficulties, problems, and failures in life.”

Allow the attention to broaden, encompassing your friends and friendly
acquaintances: “May my friends be well, happy, peaceful, and prosperous.
May no harm come to them. May no difficulties come to them. May no
problems come to them. May they always meet with spiritual success. May
they have the patience, courage, understanding, and determination to
meet and overcome inevitable difficulties, problems, and failures in
life.”

Then we can bring to mind those whom we are neutral towards. We may have
some acquaintance with them, but there is no strong feeling either way.
“May those who are neutral to us be well, happy, peaceful, and
prosperous. May no harm come to them. May no difficulties come to them.
May no problems come to them. May they always meet with spiritual
success. May they have the patience, courage, understanding, and
determination to meet and overcome inevitable difficulties, problems,
and failures in life.”

Now allow your attention to include even those who we feel are
unfriendly towards us, people we’ve had difficulty with or whom, in
normal circumstances, we wouldn’t think of in a kindly way. Recognize
that we’re all in the same boat of birth, aging, sickness, and death,
and that they’re worthy of our well-wishing, our kindness. “May those
unfriendly to me be well, happy, peaceful, and prosperous. May no harm
come to them. May no difficulties come to them. May no problems come to
them. May they always meet with spiritual success. May they have the
patience, courage, understanding, and determination to meet and overcome
inevitable difficulties, problems, and failures in life.”

Allow the heart to be non-discriminative, expansive, and unlimited in
its wish towards all living beings. “May all living beings be well,
happy, peaceful, and prosperous. May no harm come to them. May no
difficulties come to them. May no problems come to them. May they always
meet with spiritual success. May they have the patience, courage,
understanding, and determination to meet and overcome inevitable
difficulties, problems, and failures in life.”

You can take the next period of sitting to work with the practice on
your own, as feels comfortable, meaningful, or useful to you.
