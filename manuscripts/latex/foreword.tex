\chapter{Foreword}

{\center
\emph{Even as a mother protects with her life\\her child, her only child
\ldots{}}

}

\vspace{1em}\noindent
That boundless, unconditional love of a mother: an image that I will
forever carry in my heart is of my own mother as she waited for death to
relieve her of a tired and malfunctioning body. Only four days before
her passing, she sat on the edge of the bed in the evening in an attempt
to find a moment of comfort or ease after a long day. As I sat next to
her, the tears I had been hoping to hold spilled forth. I blurted out
the sense of impending loss I was feeling, how much I would miss her. In
that instant her own discomfort and suffering were forgotten. Some
hidden energy source lifted her body from its slight slouch. She put her
arm around me to comfort me and reminded me: “We carry each other in our
hearts.”

When Luang Por Pasanno completed the teaching of the retreat that is the
source of this book in September of 2008, a handful of us declared
enthusiastically that we would produce a book so others could also
benefit from this retreat. Time passed. My mother passed. The idea to
put these teachings into a book resurfaced, but this time with a deeper
intention. It felt like a wonderful tribute to my mother, to all
mothers—whether from a biological mother or someone who has fit the
ideal, perhaps a father, aunt, uncle, older sibling, friend, or
teacher—as an expression of the gratitude for the example and the love.

Many have contributed generously to the development of the book: the
transcribers, various editors, layout team, and proofreaders—in
particular, I'd like to thank Tom Lane for his monumental effort in
editing, Hisayo Suzuki for her sharp eye and mind with the line editing,
and Sumi Shin for the creative and expressive cover design. I am
grateful to all who participated and contributed in this wonderful
community effort. This offering is in honor of mothers, past, present,
and to come. It represents a beautiful transference of the energy of
love and gratitude into something produced for the benefit of others,
seen and unseen.

We are all very grateful for the limitless generosity that Luang Por
exemplifies when teaching the Dhamma.
