\chapter{The Five Aggregates Affected by
Clinging}

Last night during the question and answer session, there was a question
asking about a reference for a sutta that talked about “path and fruit”
and the stages of liberation: the path of liberation and the fruit of
liberation. So I went back last night and scoured some books. There was
also a question the other night about what is fun about being a monk,
and I had fun last night.

This sutta is something people may have heard referenced in different
places. The Buddha is giving a discourse about the eight wonderful and
marvelous qualities of the ocean. One that is well known is that the
ocean has but one taste, the taste of salt. And just as the ocean has
one taste, the taste of salt, even so this Dhamma and discipline, which
is what the Buddha called his dispensation, has but one taste: the taste
of liberation.

The eighth quality is that: “Just as the great ocean is the abode of
great beings, even so this Dhamma and discipline is the abode of great
beings: the stream-enterer and the one practicing for the realization of
the fruit of stream-entry; the once-returner and the one practicing for
the realization of the fruit of once-returning; the non-returner and the
one practicing for the fruit of non-returning; the arahant and the one
practicing for arahantship. This is the eighth wonderful and marvelous
quality of this Dhamma and discipline, which the monks perceive again
and again, and by reason of which they take delight.” (A 8.19)

As I was poking around, I came across a delightful sutta that I thought
I would read this morning, which also concerns the fruits of realization
and, in particular, the first access into the realization of
stream-entry (A 6.10).

\begin{quotation}
On one occasion the Blessed One was dwelling among the Sakyans at
Kapilavattu in the Banyan Tree Park. {[}That was his hometown.{]} Then
Mahānāma the Sakyan approached the Blessed One, paid homage to him, sat
down to one side, and said to the Blessed One:

“Bhante, how does a noble disciple, who has arrived at the fruit and
understood the teaching, often dwell?” {[}That is, the realization of
stream-entry.{]}

“Mahānāma, a noble disciple, who has arrived at the fruit and understood
the teaching, often dwells in this way:

“Here, Mahānāma, a noble disciple recollects the Tathāgata thus: ‘The
Blessed One is an arahant, perfectly enlightened, accomplished in true
knowledge and conduct, fortunate, knower of the world, unsurpassed
trainer of persons to be tamed, teacher of devas and humans, the
Enlightened One, the Blessed One.’ When a noble disciple recognizes the
Tathāgata thus, on that occasion his mind is not obsessed with lust,
hatred or delusion; on that occasion his mind is simply straight, based
on the Tathāgata. A noble disciple whose mind is straight gains
inspiration in the meaning, gains inspiration in the Dhamma, gains joy
connected with the Dhamma. When he is joyful, rapture arises. For one
with a rapturous mind, the body becomes tranquil. One tranquil in body
feels pleasure. For one feeling pleasure, the mind becomes concentrated.
This is called a noble disciple who dwells in balance amid an unbalanced
population, who dwells unafflicted amid an afflicted population. As one
who has entered upon the stream of the Dhamma, he develops recollection
of the Buddha.

“Again, Mahānāma, a noble disciple recollects the Dhamma thus: ‘The
Dhamma is well expounded by the Blessed One, directly visible,
immediate, inviting one to come and see, applicable, to be personally
experienced by the wise.’ When a noble disciple recollects the Dhamma,
on that occasion his mind is not obsessed by lust, hatred, and delusion;
on that occasion his mind is simply straight, based on the Dhamma \ldots{}
This is called a noble disciple who dwells in balance amid an unbalanced
population, who dwells unafflicted amid an afflicted population. As one
who has entered upon the stream of the Dhamma, he develops recollection
of the Dhamma.

“Again, Mahānāma, a noble disciple recollects the Sangha thus: ‘The
Sangha of the Blessed One’s disciples is practicing the good way,
practicing the straight way, practicing the true way, practicing the
proper way; that is, the four pairs of persons, the eight types of
individuals—this Sangha of the Blessed One’s disciples is worthy of
gifts, worthy of hospitality, worthy of offerings, worthy of reverential
salutation, the unsurpassed field of merit for the world.’ When a noble
disciple recollects the Sangha, on that occasion his mind is not
obsessed by lust, hatred, and delusion; on that occasion his mind is
simply straight, based on the Sangha \ldots{} This is called a noble disciple
who dwells in balance amid an unbalanced population, who dwells
unafflicted amid an afflicted population. As one who has entered upon
the stream of the Dhamma, he develops recollection of the Sangha.

“Again, Mahānāma, a noble disciple recollects his own virtuous behavior
as unbroken, flawless, unblemished, unblotched, freeing, praised by the
wise, ungrasped, leading to concentration. When a noble disciple
recollects his virtuous behavior, on that occasion his mind is not
obsessed by lust, hatred, and delusion; on that occasion his mind is
simply straight, based on virtuous behavior \ldots{} This is called a noble
disciple who dwells in balance amid an unbalanced population, who dwells
unafflicted amid an afflicted population. As one who has entered of the
stream of the Dhamma, he develops recollection of virtuous behavior.

“Again, Mahānāma, a noble recollects his own generosity thus: ‘It is
truly my good fortune and gain that in a population obsessed by the
stain of miserliness, I dwell at home with a mind devoid of the stain of
miserliness, freely generous, openhanded, delighting in relinquishment,
devoted to charity, delighting in giving and sharing.’ When a noble
disciple recollects his own generosity, on that occasion his mind is not
obsessed by lust, hatred, and delusion; on that occasion his mind is
simply straight, based on generosity \ldots{} This is called a noble disciple
who dwells in balance amid an unbalanced population, who dwells
unafflicted amid an afflicted population. As one who has entered of the
stream of the Dhamma, he develops recollection of generosity.

“Again, Mahānāma, a noble disciple recollects the deities thus: ‘There
are devas {[}ruled by{]} the four great kings, Tāvatiṃsa devas, Yāma
devas, Tusita devas, devas who delight in creation, devas who control
what is created by others, devas of Brahmā’s company, and devas still
higher than these. There exists in me too such faith as those deities
possessed because of which, when they passed away here, they were born
there. There is found in me such virtuous behavior \ldots{} such learning \ldots{}
such generosity \ldots{} such wisdom as those deities possessed because of
which, when they passed away from here, they were reborn there.’ When a
noble disciple recollects the faith, virtuous behavior, learning,
generosity, and wisdom in himself and in those deities, on that occasion
his mind is not obsessed by lust, hatred, and delusion; on that occasion
his mind is simply straight, based on the deities. A noble disciple
whose mind is straight gains inspiration in the meaning, gains
inspiration in the Dhamma, gains joy connected with the Dhamma. When he
is joyful, rapture arises. For one with a rapturous mind, the body
becomes tranquil. One tranquil in body feels pleasure. For one feeling
pleasure, the mind becomes concentrated. This is called a noble disciple
who dwells in balance amid an unbalanced population, who dwells
unafflicted amid an afflicted population. As one who has entered the
stream of the Dhamma, he develops recollection of the deities.

“Mahānāma, a noble disciple, who has arrived at the fruit and understood
the teaching, often dwells in just this way.” (A 6.10)
\end{quotation}

One of the things that I like to point out is that, particularly in
terms of meditation, there are different recollections such as these to
which we can direct our attention and then there is gladness connected
with the Dhamma, arising on account of that. When there is gladness,
then rapture arises. When one is uplifted in rapture, the body becomes
calm. One who is calm in body feels happy. For one who is happy, the
mind becomes concentrated.

That is usually backwards from how we approach our meditation. “When I
get my \emph{samādhi}, then I’m going to be happy.” All the effort is
put into trying to get the mind to comply with the wish to be still and
concentrated. Rather, we should pay attention to those qualities that
bring a sense of well-being, gladness, calming, and happiness.
Concentration arises out of that. That is a way of establishing
well-being, allowing the mind to settle from there.

On a certain level, the gladness connected to the Dhamma is also from
seeing more clearly how these different things fit together. What is the
obstruction to cultivating loving-kindness? Almost invariably, it will
be the sense of self that impinges on consciousness, and not just
impinges on consciousness, but looms. Reflect that the nature of how we
construct a sense of self is a burden and reflect on that in terms of
Dhamma, with respect to the five khandhas.

When we do the morning chanting, we go through all of the different
aspects of suffering. Then the chant ties it all up, saying, “In brief,
the five focuses of the grasping mind are dukkha.” This is a recognition
that these \emph{pañcupādāna-khandhā} (the aggregates of being) are
affected by grasping, clinging, and attachment. The identification with
body, feeling, perception, mental formations, and consciousness—the
habit of attachment and clinging—is what is intrinsically bound up with
suffering, dissatisfaction, and discontent, and then spills out into all
of the various forms of afflicted emotional states. So investigate those
five aggregates of being or khandhas.

There is a discourse (S 22.100) in which the Buddha gives an image of a
dog being tied to a post. Whether that dog lies down, stands up, sits,
or walks, it is tied to that post. And the Buddha says as long as we are
bound up and tied to those five khandhas—identification with the body,
feelings, perceptions, mental formations, and consciousness—then in the
same way, whether we walk, stand, sit, or lie down, we’re bound with
suffering, discontent, and dissatisfaction: dukkha.

That whole process of identification with body, feelings, perceptions,
mental formations, and consciousness—“this is mine, this is what I am,
this is myself”—is inevitably productive of suffering. It’s inherent.
It’s intrinsic. It can’t be any other way.

So recognize how clinging, attachment, and identification keep
coalescing and forming, establishing themselves. And then recognize that
there is an opportunity of non-clinging, non-attachment.

“How do I do that? How do I not cling to everything I conceive of as
myself?”

If you have ever gone on a long hike with too much stuff, when you put
that pack on, you think, “It’s a bit heavy, but it’s okay.” But you keep
walking and the pack gets heavier and heavier. And you say, “Well, I
shouldn’t have taken this much stuff, but maybe I’ll need it.” You keep
carrying it and carrying it, and you think, “I’ve got to get rid of some
of this stuff.” You let it go because you can’t carry it any longer.

It’s the same thing with attachment and clinging when you realize that
this is a burden and you don’t really want to carry it any longer.
You’re willing to put it down, get rid of it, drop it. That’s the good
news. The bad news is we usually have to wait until we are really
willing to put it down.

The Buddha had us reflect on and investigate these aspects of the
khandhas and their emptiness, their ephemeral nature. The Buddha once
offered a discourse (S 22.95) when he was by a river. As often happens,
there was some foam on the river, and the Buddha used that as an
example.

“See that foam on the river? That foam is just like our bodies. That
foam is insubstantial, empty, and hollow. In the same way our bodies are
just like that.” If we reflect on the nature of the body, it seems
substantial. As we age and get sick, we see that there is really not
much here. All of the different cells and other bodily constituents are
really insubstantial.

The same is true with feeling. In Asia, the monsoon season brings big,
fat, heavy raindrops. When the rain comes down, little bubbles come up.
Feeling is like those bubbles. A little bubble forms and it bursts.
Another little bubble forms and it bursts. That’s exactly like feeling.

It’s good to clarify that “feeling” refers specifically to pleasant,
painful, and neutral feeling. It’s a basic quality of experience, of
sensation—not feelings in terms of emotions. It’s the basic tone of any
kind of contact: pleasant, painful, or neutral. Sights, sounds, smells,
taste, touch, and mind objects: there’s always an associated feeling
tone. Because those experiences are arising and ceasing, the feeling
tone is arising and ceasing. They are little bubbles arising and
popping, arising and popping. We tend to react and respond to them with
our preferences, so we end up layering that very simple feeling tone
with our likes and dislikes, wanting and not wanting. But if we reflect,
there is just a basic feeling, arising and ceasing.

The Buddha compares perception to a mirage, like in the hot season in
India or in America out on the plains. I guess the Central Valley would
get enough heat to form a mirage, images that shimmer in the distance.
It looks like something, and then when you get closer, it’s not that.

Perception is also bound up with memory, giving meaning to things, which
we build through memory: how we value and give importance to things, how
we hold them through the nature of perception. There’s not really a
thinking process going on with that. We perceive something in a certain
way, affected by memory and how we have perceived it over time as well.

Ajahn Amaro recently gave a talk in which he related an experience he
had when he was in England as a young monk. He did a long walk from our
monastery, in the south of England, to a branch monastery up near the
Scottish border. He had completed his fifth rains retreat, which is a
time when monks have the opportunity to spread their wings a bit. He
wanted to try a walk, as a monk, in England and went with a layman who
would be able to help out if necessary. This was the first time in our
community that anyone had tried a long walk as an alms mendicant, and it
took them several months.

This last spring the two of them got together and revisited the walk.
They didn’t do the whole thing. They did parts of it for the
25\textsuperscript{th} anniversary of their quite historic walk.

The really interesting thing was that virtually every place they went
back to, something had been moved. A hill had moved; a house had moved.
It had been so vivid in his memory and he had told the story so many
times, but when he actually went back, it wasn’t how he had remembered
it or had related it to people. It was different.

\emph{Saññā aniccā} (perception is impermanent). Perception is not a
mechanical function that is solid. It’s the \emph{upādāna-khandha} ( the
aggregate of being affected by attachment and clinging). Clinging arises
out of like or dislike, or some kind of delusion. It doesn’t have to be
gross. We pick up on gross things, but it is the subtle things that are
really believable. Then we build on them and they become our reality.

That is our unfortunate circumstance as human beings. We live in a world
that we have created and that is a function of our wishes. The
unacknowledged agenda is, “Boy, if I could just build the world the way
I wanted it! Wow, that would be great!” But the reality of the noble
truth of suffering is that we then have to live in that world.

Perception is one of the fundamental things we build the world out of:
our identification with self—\emph{my} position, \emph{my} preference,
\emph{my} idea, \emph{my} ideal, \emph{my} fear. It’s not really thought
through. It’s a much deeper stratum of the mind than thinking and
planning, and it overlays everything. Reflect on the nature of these
aggregates of being. Reflect on the nature of perception. Perception is
like a mirage, although that may not be the way we naturally conceive
it.

The Buddha compared the mental formations to a banana tree. A person
goes into the forest looking for hardwood and cuts down a plantain tree,
a banana tree. If anyone has seen a banana tree, it’s layer after layer
of leaves. There is no core to it. It seems solid: six inches, eight
inches, twelve inches around. But as we take each layer off, we can peel
it all back and there is nothing there. There’s just another leaf
inside.

So the nature of mental formations, thought processes, is that there are
just these layers but no solid core. It is absurd to consider, “What is
my one true thought?” But this doesn’t mean that we can’t use thoughts,
emotions, and planning, all of which takes place within mental
formations. Mental formations can be related to ill will, aversion,
desire, attachment, fear, or anxiety. Mindfulness is also a mental
formation. Wisdom, loving-kindness, compassion, and equanimity are all
mental formations, but they have a usefulness and skill to them, a
beneficial result.

So it’s not, “Oh, mental formations are a banana tree. I’m done with
that.” No. We really have to know how to use, understand, and skillfully
apply the thought processes and not be trapped or caught by them. It’s
when we see this more clearly that we develop a certain wariness toward
the thought processes.

We have been using thoughts of mettā, loving-kindness, as a practice. We
can see these wholesome mental formations starting to gain momentum over
irritation, aversion, or a negative viewpoint that has arisen. We see
that if we let that go—“I know where this is going, I know where this
will end”—we’ll turn our attention to loving-kindness. We’ll turn our
attention to well-wishing. We can counter the negative impulses. By
seeing mental formations as empty and hollow, we won’t just push them
away. Because they are empty and hollow, we need to be very careful and
attentive around them.

The Buddha has us reflect on consciousness. He compares it to a magician
or conjurer’s trick. In India, in the marketplace or on the street
corner, there can be a magician or conjurer, a person doing magic
tricks. There is a small crowd around him and he is making a little
money with it. Today, the same scene is still going on, 2,500 years
later. If we understand the nature of the conjurer’s trick, the
mechanics behind it, there’s not the same excitement or fascination.
We’re not as willing to lay money down to see the trick again.

The nature of consciousness is very similar. We take these disparate
experiences—contact with the senses—and form them into an experience.
“Yeah, he really said that.” We have formed the consciousness of that
experience really happening, and we remember it a certain way, but
actually, there is an agenda behind consciousness as well. We have to
recognize the hook buried in it.

We will never be free of the five khandhas, the five aggregates of
being. But we can be free from being affected by clinging to the
aggregates, and that is the most important part because we do need to
rely on those five aggregates of being. That’s how we live; that’s how
we accomplish things.

That’s how the Buddha was able to teach—with the five khandhas. If he
had said, “Well, that’s it for the five khandas,” and had done nothing,
then we would be in a bind. After the Buddha’s enlightenment, he spent
forty-five years carrying the khandas around and teaching, establishing
a dispensation that has lasted to this day.

This reflection on the khandhas is a really important part of
understanding the nature of self and attachment to self, the nature of
clinging, and then the nature of relinquishment, letting go. As we are
able to attend to letting go, part of the response is the sense of
loving-kindness. It is a natural response of seeing Dhamma, seeing
things the way they are, that kindness and compassion arise, the sense
of not being trapped by the particular reactions that come up.

I remember Ajahn Jumnien saying that whenever anybody comes to him
saying, “Well, I’ve attained it. I’ve had stream-entry now. No more
rebirth in the lower realms for me. I’m really pleased with my
practice,” he will usually find the occasion to say, “Well, you know
so-and-so has been saying this about you. They have been saying you’re
like this and like that.” Then he would see what sort of response there
was because if there is anger, thinking, “How could they say that?” then
the person is not a stream-enterer.

Here, once again, that holding on to identity is one of the hallmarks of
worldly existence. The relinquishment of that identification process is
one of the fetters that are dropped as one enters the stream of Dhamma.

Ajahn Chah did something similar to someone who came to him. He was
saying how well his practice was growing, that he was practicing really
diligently and had had a breakthrough and was quite confident now that
it was clear sailing to arahantship. Ajahn Chah just grunted and said,
“That’s better than being a dog.” Especially in Thai, when you compare
somebody to being a dog, it’s \emph{bad}. That person got extremely
upset. Ajahn couldn’t help but relate this by saying, “So much for the
stream-enterer.”

So loving-kindness, that inclination to well-wishing, is an integral
part of the Dhamma. As we cultivate loving-kindness, it is not something
that is separate from practicing Dhamma, reflecting on Dhamma. They are
woven together and support each other. We cultivate and bring
loving-kindness into being not only to help establish wholesome states
of consciousness and calm states of abiding. It also lays a foundation
for insight because it is a helpful tool for creating a stability and
brightness of mind that allows us to see the Dhamma. We reflect on and
investigate Dhamma but also see that this is a result. They work
together and are interwoven. So giving attention to these qualities is
something that allows us to gain inspiration and confidence in the
Dhamma.
