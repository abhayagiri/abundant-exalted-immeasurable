\chapter{Mindfulness Immersed in the
Body}

I thought I would begin the morning teaching with a reflection from the
Buddha concerning mindfulness of the body.

\begin{quotation}
“Bhikkhus, even as one who encompasses with his mind the great ocean
includes thereby all the streams that run into the ocean, just so,
whoever develops and cultivates mindfulness directed to the body
includes all wholesome qualities that pertain to true knowledge.

“Bhikkhus, one thing, when developed and cultivated, leads to a strong
sense of urgency \ldots{} leads to great good \ldots{} leads to great security
from bondage \ldots{} leads to mindfulness and clear comprehension \ldots{} leads
to the attainment of knowledge and vision \ldots{} leads to a pleasant
dwelling in this very life \ldots{} leads to the realization of the fruit of
knowledge and liberation. What is that one thing? Mindfulness directed
to the body. This is the one thing that, when developed and cultivated,
leads to the realization of the fruit of knowledge and liberation.

“Bhikkhus, when one thing is developed and cultivated, the body becomes
tranquil, the mind becomes tranquil, thought and examination subside,
and all wholesome qualities that pertain to true knowledge reach
fulfillment by development. What is that one thing? Mindfulness directed
to the body. When this one thing is developed and cultivated, the body
becomes tranquil \ldots{} and all wholesome qualities that pertain to true
knowledge reach fulfillment by development.

“Bhikkhus, when one thing is developed and cultivated, ignorance is
abandoned \ldots{} true knowledge arises \ldots{} the conceit ‘I am’ is abandoned
\ldots{} the underlying tendencies are uprooted \ldots{} the fetters are
abandoned. What is that one thing? Mindfulness directed to the body.
When this one thing is developed and cultivated, the fetters are
abandoned.”

\hspace*{\fill}(A 1:575, 1:576–582, 1:583, 1:586–590)
\end{quotation}

These are very useful reflections because although the theme of this
retreat is mettā, loving-kindness, it’s essential to have an anchor, a
stable base or foundation to work from. Sometimes, trying to direct
attention to a mental state or an emotion is like trying to grab the
wind—it slips right through your fingers. But having a good grounding in
the quality of mindfulness that is focused on the body, you have a firm
basis to build from. As the Buddha said, all of these wholesome
qualities arise out of that mindfulness directed to the body. Once you
have a stable base, then you have a foundation for building wholesome
mental states.

Even if you are doing mettā as a meditation, you still need to have an
anchor. You still need to be grounded in something and have someplace to
keep returning to, which is mindfulness of the body. Become very
familiar with returning to the body.

There are many ways of using mindfulness of the body, but in terms of
the formal practice of meditation, become familiar with mindfulness of
breathing and bring attention to the breathing process. Sit and pay
attention to the breath coming into the body: the sensation of the
breath as it touches the tip of the nose, as it passes through the nasal
passages, the back of the throat, down into the chest, the abdomen
rising. Pay attention to breathing out: the abdomen falling, the
sensation in the body, the chest area, throat, tip of the nose, the
breath going out. Tune in to that rhythm of the breath. Not forcing the
breath, not trying to regulate the breath, just attend to the breath as
you are experiencing it.

There is a tendency for us to try to regulate, oversee, or maintain the
breath in some way, but that’s adding an extra layer of complication.
Unless you’re dead, you are going to be breathing, so it happens anyway.
Just tune in to the breath: allow that sense of relaxing into the
breathing. Also recognize that the breathing takes place within the
whole sphere of the body. You are sitting here and breathing, but it
isn’t as if there’s just a nose suspended in space somewhere: you have
the rest of your body sitting here. How are you holding it? What does it
feel like? Is it comfortable? Is it uncomfortable? Does it feel
spacious? Does it feel contracted? What does the body feel like? Is
there some tension in the face, in the shoulders? Is it relaxed? Can you
soften that? What does it feel like in the abdomen?

Relax the abdomen. It’s quite interesting; the more comfortable we are
with breathing, relaxing, and settling, our posture gets better over
time. It feels natural and balanced. You don’t need to strain to hold
the posture upright and make yourself sit up straight. It’s obviously
necessary, to some degree, to try to keep an upright posture, but if
there is too much force or straining, it gets uncomfortable to sit
straight. But as you keep relaxing, softening, and releasing tension,
you find that the body naturally lines up quite nicely because that’s
the most comfortable way to sit for three-quarters of an hour, an hour,
or however long one sits in meditation. That’s the most natural way for
the body to sit.

One of the images helpful in sitting is imagining that there is a string
or something similar at the top of your head. If you were dangling from
that string, how would the spine and the shoulders line up? It would all
be arranged quite neatly, nice and straight. There would be a balance
there. That sense of an upright posture is one of comfort and ease as
you continue to relax.

The core is the breathing. As you breathe in, allow the breath energy to
suffuse the body. The breath comes in, suffusing the body with the
feeling of warmth, of energy. As the breath goes out, there is a sense
of releasing, relaxing, settling, non-clinging, not holding. You can’t
hold on to the breath. You can hold on to it for maybe a minute tops,
but you start turning blue, and that’s not very comfortable. That
natural rhythm of the breath suffuses the body with a sense of warmth
and energy. And then, as you breathe out, allow the body to feel
suffused with a sense of relaxing, of letting go, releasing.

As we breathe in and we breathe out, the air comes into the body, the
air is expelled out of the body. It’s released into the atmosphere,
whirled around. In the same way, thoughts, impressions, and feelings
come into the mind. Allow them to go back out again; release them, let
them go. The nature of the breath is to come in and go out. It’s the
same thing with thoughts, feelings, memories, and perceptions—it’s the
nature of the mind. We don’t have to hold on, cling, or make anything of
great importance out of it. We recognize: “That’s just a thought, just a
memory.” It comes into being, and we can let it go out again.

Of course, thoughts and feelings are much stickier than the breath, so
it’s helpful to have the image of the breath, that feeling of the breath
coming in and going out, and then extrapolate that to the feelings,
perceptions, memories, and thoughts. They can come in and go out as
well.

Then there is a releasing, a spaciousness that is very relaxing, very
settling. The mind starts to become clearer, more composed. It’s quite
natural. When we allow the breath to suffuse the body, allowing the
breath to energize and warm it, there is a relaxation that comes into
the body. As we allow the mind to have its thoughts and feelings arising
and ceasing—coming in and going out—the mind and the heart relax as
well.

You don’t force the mind to be still or be clear. It’s through allowing
the impressions in the mind to come in and go out without reacting,
clinging, or adding a commentary that a natural clarity and spaciousness
starts to arise. Allow that to happen. It’s an exercise that we need to
remind ourselves to do.

Sometimes, when we have meditated for years, it’s very easy to get into
a rut or a habit. We perceive ourselves in a certain way. We perceive
our thoughts and feelings in certain ways, and then we take the bait. I
remember Ajahn Chah saying, “Human beings are like fish. A fisherman
puts a hook out; it’s got a little bit of bait on it, and they go for it
every time. Well, actually, they’re not like fish at all. When fish take
the bait, it only gets caught in their mouth. Humans are more like
frogs. Frogs swallow that bait and the hook right down into the gut.”

Remember: “I don’t need to take the bait.” The bait is the feeling of
liking and disliking, the feeling of “that’s fearful, that’s
intimidating,” “I really like that, I really want that,” or “that’s
important, I can’t let that one go.” We take the bait.

With that simple process of attending impartially to something as
uncomplicated as the breath coming in and the breath going out, you
might think, “That’s pretty basic. When am I going to get on to the
\emph{real} meditation?” But that is what we have to learn how to do: we
need to develop the ability to relate to the mind, the world, and the
body in a way that we don’t get caught in reactions. That is how the
mind becomes settled and clear.

That exercise of coming back to the breath, the posture, and the body:
how does it feel? Starting from the top of the head and feeling, ask:
“Is there a sense of tension at the top of the head? The forehead, how
can I relax that? What does it feel like around the eyes? Can I relax
that? Can I soften that? Around the jaw, around the throat \ldots{}” You are
paying attention and recognizing there is breathing—breathing in,
breathing out.

What does it feel like in the shoulders, in the arms? What’s a
comfortable position for the arms? How can I hold them so that they hang
comfortably and the hands can be in an easeful position, relaxed and
settled?

Feel the breath. Allow the breath to suffuse and soften the body. Allow
the chest to be open and spacious, rather than contracting in. You can
see that contraction when you are pondering or recollecting something
that has a weight to it, trying to figure it all out. The head comes
down and shoulders come in. Bring some space to it, soften, relax. As
the breath comes in and the breath goes out, allow it to be open, soft,
spacious.

Don’t tighten the abdomen. Allow the breath to come down into the
abdomen. Allow the breath to affect the whole body, the whole trunk.
Breathing in, the abdomen expands, breathing out, it relaxes. If there
is tension in the lower back, soften it. Find a balance where holding
the posture doesn’t take any muscle—you can do this by allowing the
trunk of the body to relax and then letting it balance itself on its
own.

Oftentimes, people will complain, “My back hurts around the shoulder
blades from all this sitting.” A lot of that pain is caused by trying to
hold the posture using the chest, upper back, or shoulders—trying to
\emph{hold} the posture. It gets painful. So relax and allow the spine
to rise up out of the pelvis. That is what we have those bones for.
They’re much easier to look after than the muscles. Let the muscles do
what they’re going to do and just relax them.

We have all the props: cushions, zafus, and whatever else we need to get
a balanced sense of posture and allow the muscles to relax. We can relax
into the posture, and then the breathing becomes much more natural.
Allow the rhythm to be very natural, not forcing it. Then simply pay
attention. Is this a long breath? Is this a short breath? Is it
comfortable? Is it uncomfortable?

You don’t have to go into any more of an analysis than that. Attend to
that: that’s soft, that’s a bit hard, that’s short, that’s long. Just
familiarize yourself with the breath: that’s the in-breath, that’s the
out-breath. Attend to that, mindfully aware of where you are in the
present moment with the breathing process, mindfully aware of the
general feeling of the body, again, without going into an analysis and a
commentary. This is what it feels like. Does it need to be adjusted?
Does it need to be adapted in any way so that it can be a bit more
comfortable?

That is enough—sustaining the attention on the breath, the in-breath and
the out-breath. It’s okay to adjust the posture a bit if necessary,
balancing so that you’re feeling comfortable. It’s important not to
strain or to strive too hard because the mind will only truly settle
once there’s relaxation. You can’t bludgeon your mind into submission;
it’s not going to work. You need to allow it to settle and then attend
very mindfully, sustaining the quality of awareness.

It’s the quality and continuity of awareness that allows the mind to
settle and become peaceful. It’s not imposing a concept, your idea of
peacefulness, on the mind—that’s not going to work.

Instead, relax and learn how to pay attention. Experience the whole
body. The breath is what we attend to in a focused way, but in general,
there is also the sense of what the whole body is feeling like. Is it
nice and straight? Is it balanced? What are the different feelings that
come up?

Experience those feelings in the body. Where in the body do we
experience anxiety and worry? Where do we feel it? What does it make the
body feel like? When a particular desire comes up, where do we feel
that? The experience of ill will and aversion in the body: where do we
sense that? What does it feel like? The body is a very clear measure of
our mental states. If we become more and more attentive, we realize that
the different mental states and moods we have are experienced in the
body and that we can relax around that.

As you familiarize yourself, you understand: “When my body is tensing
like this, when my abdomen gets tight, that’s anxiety, that’s fear” or
“That’s worry, that feeling in the chest.” The movement towards getting
or consuming something creates agitation in the body. In truth, we’re
not very discerning; we’re quite okay with whatever the flavor of the
moment is.

However, we can see a repeated pattern of how desire keeps playing
itself out by the way that the body is experienced. We can recognize how
it works. Tune in to and understand it: “There’s that feeling in the
body.” It gives us a clue. Usually when there’s a particular fear or
desire or aversion, the mind is so quick that it goes right into a
story. It has its justifications: “I wouldn’t think that if it wasn’t
\emph{right}.” Of course, we have all gotten ourselves into trouble with
those kinds of attitudes, and we believe it over and over again.

So tune in to how it is felt in the body. You can experience it without
the story, and that is where you can clarify it or allow it to pass
through. Again, with the breath, say, “I recognize that feeling.” The
mind is going to be whispering—probably screaming—its story, but keep
coming back to the body and then relaxing, attending, letting it go.

It’s very, very simple but very freeing. It’s liberating not to be
trapped in the stories. The feelings and stories start to cascade.
Simplify, coming back to the bodily sensation, the feeling in the body.
The breath is that anchor. The in-breath: soothing, nurturing, warming,
energizing. The out-breath: releasing, relaxing, expanding. We can tap
into that feeling and allow the moods and impressions to dissipate,
grounded in awareness in the present moment.

Sometimes we have to deal with physical pain and mental distress. Those
are very real experiences. But we still have the breath, which is always
neutral. The breath is a neutral base to return to. Can we relax around
the physical pain we’re experiencing? Are we exacerbating it? Maybe
there’s illness. Can we actually do anything about it?

Within that sphere of awareness and the neutral quality of the breath,
we can be present with that pain. Then, if something needs to be done,
we can do it. If not, then we can establish the attitude of not
complicating the pain. Sometimes the physical pain can be quite
bearable, more so than the mental pain of: “I don’t like it. I don’t
want to be here. It shouldn’t be this way. Why me?” We spin ourselves
out like that and that’s where a lot of pain really is. The actual
physical pain is then much more bearable.

Separate these different experiences out. This separation isn’t just an
intellectual analysis. On an experiential level, by having the anchor of
the breath coming in and going out, being grounded, relaxed, and settled
in the body, we are able to use discernment. The word that is usually
translated as “wisdom” in Buddhism, \emph{paññā}, is probably translated
better as “discernment.” Use that ability to discern and ask: What’s
useful and what’s not useful right now? What is going to lead to peace?
What is going to lead to agitation? We need to be able to discern.
Looking and seeing, we are able to recognize that yes, this is the
course of action to follow.

Mindfulness of breathing and of the body are anchors for our practice
and for the cultivation of loving-kindness. The ability to be present
with one in-breath and one out-breath is a tremendous kindness to
oneself. The ability to be with the body, to be non-reactive,
non-judgmental: that’s an act of kindness in the present moment. Being
very attentive lays the foundations for the quality of loving-kindness
that plays itself out and is manifested in our actions.

There is a very lovely discourse, \emph{Lion's Roar} (A 9:11), where
Sāriputta goes to pay respects to the Buddha. He asks to take leave to
go wandering on a journey, and the Buddha gives his permission. He goes
to prepare himself to leave. Then another monk, who has a grudge against
Sāriputta, goes to the Buddha. He says, “Sāriputta hit me and left the
monastery.” The Buddha sends word to one of the other monks asking him
to bring Sāriputta, as there is an accusation against him. The way the
Buddha dealt with such issues is interesting in itself. He didn’t just
dismiss the monk saying, “Sāriputta wouldn’t do that.”

Sāriputta is brought back. The Buddha explains the accusation.
Sāriputta’s reply is very moving. He says, “Having developed mindfulness
of the body for so long, it is impossible to engage in unskillful
action. The mind is well established in loving-kindness, abundant,
exalted, measureless, without enmity and ill will.”

Sāriputta then gives all sorts of illustrations of the different ways
that he has contemplated and developed mindfulness of the body, with the
effect that the mind is always in a place of awareness and well-wishing,
of compassion towards all beings, making it impossible to have harmed
anybody. Sāriputta was extraordinarily skilled in giving illustrations
and teachings. Finally, the monk who had made the accusation gets up and
says, “I’m sorry. I’ve been foolish and have made a false accusation
against Sāriputta. May he forgive me.”

The Buddha’s response is, “Yes, you have been foolish and you have made
a false accusation but, having recognized your faults and wanting to
make amends, you can grow in the Dhamma and discipline.” Then he turns
to Sāriputta and says, “Sāriputta, forgive this foolish man before his
head splits in seven pieces.” Sāriputta says, “I certainly forgive him
and may he forgive me for anything I have done.”

I thought it was interesting how Sāriputta’s immediate response was:
“Somebody who has cultivated mindfulness of the body for as long as I
have, and is so grounded in it, couldn’t possibly give rise to
unwholesome states and is well established in the quality of
loving-kindness and compassion.”

These are some reflections on mindfulness of the body and establishing a
base. Perhaps we could go and do some walking meditation and then come
back for a sitting. Of course, walking meditation is not any different
than sitting, in the sense that when you’re sitting, you’re watching
your breath and, when you’re doing walking meditation, you’re still
paying attention to the whole body, but the primary focus is on the
rhythm of the steps. You’re pacing, walking a bit more slowly than
normal or using whatever pace feels comfortable for establishing
mindfulness.

Pay attention to that sensation of the right foot touching, rising,
moving, and of setting that foot down. Left foot coming up, moving,
setting down. Simply pace, paying attention to the rhythm of the
walking, the sensation of touching the ground, the movement of the foot,
and then the whole body. What’s your posture like?

Again, it’s not merely disembodied feet touching the pavement. It’s the
whole body connecting with the rhythm of the walking and paying
attention to the posture. Holding your hands clasped, usually in
front—it could be in back, but usually in front is a bit more
comfortable—so that there is a composure to the posture. Keep your eyes
downcast so that your gaze is about six feet in front. About twenty,
twenty-five, thirty paces is appropriate, whatever is comfortable.

When you get to the end of the path and turn around, it’s helpful to
stop for a minute, to check and see: Am I here? Am I still present? Am I
still with the walking? Then continue the walking. At the end of the
path, stop for a minute because it’s easy when the body is walking for
the mind to wander off as well. So connect with present reality, make
sure you are sustaining that connection with the posture and the
continuity of mindfulness.
