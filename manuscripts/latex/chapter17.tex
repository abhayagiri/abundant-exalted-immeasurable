\chapter{Questions and Answers}

\qaspace
Question: In the palm-reader story, you mentioned that Ajahn Chah still
had a lot of anger, but that he chose not to act from it. Does this mean
that, for example, if there was a troublesome monk, Ajahn Chah would
still experience a flare of anger but still have the wisdom to set it
aside and consider what to do with a cool head? This sounds similar to
what Ram Dass has said about his defilements: after so many years of
practice, they are more like little neurotic shmoos that he can relate
to in a relaxed way.

\qaspace
Answer: What the palm reader was looking at was fundamental temperament,
karmic tendencies, the patterns that are there. Anger was a particular
tendency that Ajahn Chah had, but through his ability to see clearly, he
relinquished the tendency. The awareness had such clarity that the mind
was not drawn to the tendency at all.

There was a very interesting circumstance that one of the Western monks
had with Ajahn Chah. Ajahn Jayasaro, who is an English monk, was
attending to Ajahn Chah and sitting with him, massaging his feet. A
particular monk came in who had done something wrong, had been
admonished for it and was undergoing a period of penance.

Ajahn Chah barked at him and scolded him, in what seemed to be a
flare-up of anger. That was what was manifested. Ajahn Jayasaro, who was
there massaging his feet, said that the peculiar thing was that there
was absolutely no tension in his body. There was no manifesting of a
mood or any kind of excitement. Ajahn Chah probably said to himself,
“What this monk needs is to be ripped down one side and up another.”

There was a similar circumstance that I had with Ajahn Maha Boowa, who
is well known for a couple of things. First, he is considered the
arahant of the age in Thailand. He is in his mid-nineties now. He has
been considered to be highly attained for a long time, and he is also
well known for his ferocious tendencies.

I remember one time going to visit and spend some time with him. I had
done a lot of practice at the time, and my mind was very still and
clear. I was watching Ajahn Maha Boowa speaking in a ferocious way,
hearing the words and seeing his actions. But I was flooded with these
feelings of joy and loving-kindness that I felt he was actually
emanating.

You may be able to control anger, but it is different from when anger is
not there at all. Certainly you were always quite careful around Ajahn
Chah because you never knew which side was going to come out. It wasn’t
as if he was just playing with you, but it is as if he always responded
to the situation or the person and there was no fixed way he was going
to be boxed in: “Oh, I’m supposed to be a nice loving-kindness monk.”
You didn’t want to get on the wrong side of him because he could be very
direct and blunt.

\qaspace
Q: Would you explain the duties adult children have towards aging
parents? And the duties of parents towards their children, as taught by
the Buddha.

\qaspace
A: The Buddha taught that one should look after one’s parents and try to
care for their needs. In Asia, in general, that is very much a part of
the cultural norm. One looks after one’s parents. Of course, the parents
have a duty toward their children of both helping and being role models
for their children and grandchildren. In an ideal world, that is how it
works.

The Buddha also said that the greatest gift that one could give to their
parents, and I think it would work the same for parents towards
children, is encouraging one’s parents in virtue, generosity, and right
view. There are material things that one can give to one’s parents, and
that is an excellent thing to be doing. However, it is even more
beneficial and useful if one can encourage one’s parents to establish
themselves in a higher degree of virtue and generosity, as well as in
aspects of right view and discernment.

\qaspace
Q: In a Dhamma talk several years ago, you mentioned in an interview a
Christian monk. Reflecting on his forty years of monastic life, the monk
said something about “I wish I’d been more kind.” Reflecting on your
thirty-five years in robes, do you have any lessons that stand out? I’m
not interested in regrets, but the kind of wisdom-knowledge that the
accumulation of years of practice illuminates.

\qaspace
A: It would depend on whatever is the thought of the moment as it comes
out. It is also very encouraging that people actually remember Dhamma
talks. I had forgotten I said that.

A strong lesson that pops out, picking at random out of the hat right
now, is the aspect of patience. It is certainly something that Ajahn
Chah emphasized over and over again. For myself that is one lesson that
comes to mind: just be willing to be patient with things.

Sometimes we don’t recognize the goodness that we are engaged in. Often
our standards and expectations are high, and we judge ourselves in ways
that reflect that expectation. We’re not patient enough to recognize the
goodness that we are actually doing. It’s important to be present with
that fundamental goodness that is a motivating factor in all of the
things we do.

In respect to all of you who are on a retreat here, you wouldn’t be here
if you weren’t committed to leading a wholesome, skillful life. And that
commitment is often something not reflected on. Instead we think, “I’ve
got to do this. I’ve got to do that. I’ve got to live up to this
standard. This teacher said this. That teacher said that.” It goes on
and on.

So just be patient enough to set that all aside and say, “Well, what’s
really important to me?” Allow that to come into the heart and recognize
it. And again, patience isn’t just grimly enduring; rather, it is being
able to be present with experience. That is how I relate to patience:
that willingness and ability to be present with experience, so we can
take it in and be present for it.

\qaspace
Q: Is it hopeless to send loving-kindness to Māra?

\qaspace
A: Well, Māra gets a bad rap. It depends on how we relate to it. Mara is
the personification of the tempter. In the midst of being tempted with
the desire of the moment, loving-kindness might not be the right tool.
But in a bigger sense, that is the most important thing to have
loving-kindness for, in the sense that loving-kindness is the wish for
well-being. Māra is the epitome of the tempter, the personification of
evil. That’s a lot of suffering isn’t it? The \emph{kammic}
accumulations don’t bode well for him.

When we follow our desires and get caught in our views and opinions, how
we suffer! That the whole construct keeps repeating itself: these are
the forces of Māra. That’s a worthy object of loving-kindness because
there is so much suffering there. This helps turn our attention to the
way out of suffering.

\qaspace
Q: When you have guided meditation in the past two afternoons, I’ve
enjoyed slipping past the rapture and joy. I found the meditations
incredibly grounding and at the same time streaming with energy. I’m
finding it difficult to get past joy without your vocal guidance. Can
you offer suggestions?

\qaspace
A: It’s an experiential memory. It’s good to remember that it’s possible
to go past that excitement of rapture and joy and be grounded in
something that is very stilling, but still clear and spacious. So, tune
in to the body because the body remembers these things. Come into the
body, into the rhythm of the breath, the feelings and perceptions of
joy, but allow oneself to move into steadiness and clarity.

Try to connect with it on a physical level, that is, with the tendency
for energy, rapture, and joy. There can be a level of excitement that
bubbles up. When we give an image to what joy does, it bubbles up. So
move the energy down and through the body, allowing it to settle. That’s
a skillful way to work with it because once it comes up, it can feed
into the proliferating mind.

\qaspace
Q: What are the characteristics of our personalities? Are personalities
conditioned by \emph{kamma} and our family, culture, and nationality?
How do I learn not to take mine as true and real?

\qaspace
A: It’s the suffering: that’s how we learn to drop it and recognize the
limitations, drawbacks, and pain involved. Certainly, personalities are
conditioned by all that: past \emph{kamma}, present \emph{kamma}, family
\emph{kamma}, cultural \emph{kamma}, nationality, gender, everything.
It’s all part of the conditioning process.

But then recognize its impermanence, instability, and changing nature.
Because what we call our personality sometimes manifests like this,
sometimes like that, depending on the people we are with and
circumstance we are in. It isn’t as if it’s fixed and immutable. It
picks and chooses. So learn how to recognize that, by seeing that
changing, impermanent, uncertain nature. What we take as mine and
me—it’s not fixed at all.

\emph{Anicca} is often translated as “impermanent or changing.” Ajahn
Chah would tend to translate the word \emph{anicca} as “uncertain or
“not sure.” I think that is an important shift. What is impermanent or
changing tends to be objectified, whereas “uncertain” is much more
subjectified: how I feel; how I experience things. It’s an important
shift. Bring it back into the subjective field of “how do I experience
what I think of as moods, feelings, impressions, thoughts, or
personality?” So one is seeing the suffering involved that inevitably
comes up with those perceptions—the burden that I was talking about this
morning, the limitation of being a dog tied to a post.

Also, it is important that we don’t take it too seriously, that we are
able to laugh at ourselves and the human condition, because taking it
all very seriously weighs us down. In the end it is absurd. Life is
sometimes “How did I walk into this Samuel Beckett play?”

\qaspace
Q: Just to clarify for my loving-kindness anxiety, are any phrases okay?
And can they be said as a chant or at any speed? Is one more effective
for attaining deep concentration?

\qaspace
A: The phrases have to have some sort of feeling meaningful to you. Is
it meaningful to you? Does it resonate somehow? Because what is really
important is not so much the phrases, but the feeling that arises, the
feeling that is generated and established within the heart,
loving-kindness.

As you cultivate meditation as a practice, it’s helpful to have a few
techniques, methods, phrases, and chants that feel comfortable to you
because the nature of the mind is to get inured to whatever you are
doing. No matter how good it is, you are going to get familiarized, and
it is going to lose its potency, whether quickly or slowly. It’s the
nature of the mind.

Have a few tools. It’s like when you have a tradesman show up at your
house. You want an electrician to do something and he walks in with no
tools or just one screwdriver. You are not going to feel confident that
the job is going to get done. It’s the same here. As we are practicing
and cultivating, have some tools on your belt.

\qaspace
Q: Sometimes when I am concentrated, I have spontaneous body or facial
movements. They may be large or just small twitches. Can you comment on
this?

\qaspace
A: It’s reasonably common. This is when an attunement with the body is
quite helpful, and loving-kindness is very good with balancing that kind
of energy within the body-mind complex, allowing that to unravel and
untie. Whereas trying to concentrate too strongly or forcing the mind to
have mindfulness tends to exacerbate things. So mindfulness practice
that is broad and body-based lets that kind of energy flow through the
body-mind. Then loving-kindness has a subsidiary effect of softening,
replacing, and settling. These are good practices to work with.

\qaspace
Q: Is gladness the same as thankfulness?

\qaspace
A: In the chant in which the word is translated as gladness, the Pali
word is \emph{muditā}, which is appreciative joy, sympathetic joy,
delighting in the goodness and success of others. So that’s not quite
the same as thankfulness. Another Pali word that is sometimes translated
as gladness is \emph{pāmojja}. That’s a feeling of well-being, a kind of
delight, a sense of feeling good about something.

Often the causal conditions in these sequences are a bit different:
reflecting on virtue and the sense of non-remorse, a feeling of gladness
arises. Reflecting on the truth of the teachings, gladness arises, or
faith or confidence. Thankfulness is more gratitude, which in Pali is
\emph{kataññutā}. That feeling of thankfulness, appreciation, and
gratitude has a different quality to it. So the word gladness is used in
a couple of different ways.

\qaspace
Q: By the way, what does “Pasanno” mean?

\qaspace
A: It’s primary meaning is one having faith, but also one having joy,
sort of joyful faith.

\qaspace
Q: The near enemy of equanimity is aloofness. I’ve tasted both. One is
much more familiar, and the difference is quite apparent. Can you
provide any clues to identify subtle aloofness, which may not be so
obvious, especially to the habituated mind?

\qaspace
A: One of the things that immediately comes up—I’ve mentioned it several
times and then it also came up in the reading of the \emph{Kālāma
Sutta}—is how the Buddha asks if this is wholesome or unwholesome? Is it
beneficial or not beneficial? So there is an immediate turning to what
is \emph{kusala}.

The near enemy of equanimity is aloofness or indifference. Indifference
will always have a certain negativity, dullness, or slight aversion.
It’s sort of: “I don’t want to deal with this, get out of my face.”

Whereas equanimity, as a brahmavihāra, will be \emph{kusala}: wholesome
and associated with a sense of well-being, clarity, and presence. There
is equanimity in the sense of being balanced, but it isn’t a shutting
out or holding at bay. There doesn’t need to be. With equanimity we can
be completely present with whatever is there because it isn’t
intimidating. It’s not something that is fearful. It’s not even strange.
It’s just the way it is.

We can hold something in a very wholesome space because it’s not
intimidating or fearful, and it doesn’t stir up any aversion or
negativity. It’s just the way it is. So there is a clarity there. Tuning
into kusala and akusala sorts things out very quickly.

\qaspace
Q: In your opinion, why do you think that Venerable Moggallāna is most
often portrayed in suttas as a bit of an idiot and highly impractical
while Venerable Sariputta is portrayed as wise and skillful? Both were
the Buddha’s leading disciples. Is this possibly a result of translation
or translators’ view, or is this so in Pali also?

\qaspace
A: Actually, I think if anybody gets the short end of the stick, it is
Ānanda, because he is a bit of a fall guy sometimes. But with all of his
psychic powers, there is a certain spectacular quality to Moggallāna. He
is also sometimes involved in sorting out certain problems within the
Sangha. He is trusted, as well as proactive in dealing in Sangha
matters.

I’m not so familiar with the Mahāyāna sutras. Certainly Sāriputta is
often a fall guy in the Mahāyāna sutras. I’m not sure how Moggallāna
comes across in the Mahāyāna sutras.

In regards to Anuruddha, there is a passing comment about when Ānanda
had asked Anuruddha to sort something out and it didn’t get done (A
4:243). The Buddha says, “Why did you even ask Anuruddha? You know what
he’s like. He’s just not interested in practical matters.” It wasn’t a
criticism. He was just the wrong person for the job.

I think one of the things is that in the Pali Canon, these great
disciples come across as very human. They have personalities. They are
distinct individuals.

There is a great sutta (Ud 4.4) in which Sāriputta is sitting and
meditating in the forest and Moggallāna, who is adept in psychic powers,
sees this scene unfolding where there are two \emph{yakkhas}, which is a
type of celestial demon. They are flying along overhead and see
Sariputta meditating in the forest. It is the day before the Full Moon
Day, and monks have nicely shaved heads.

One of the yakkhas says, “Look at that nicely shaved head. I’d just love
to take my club and wallop it.” And the other yakkha says, “I wouldn’t
do that if I were you. These sons of the Sakyan, they are pretty special
sometimes. Be careful.” The other says, “I don’t know—that’s a great
target. I don’t think I can resist.” So he goes over, takes his club,
and wallops Sariputta over the head. The description is that he gives
him a blow that could split a mountain.

Sāriputta is unmoved. Moggallāna has seen this scene unfold. Sāriputta
comes out of his meditation, and Moggallāna asks, “How are you feeling
today?” Sāriputta says, “Okay. But I’ve got a bit of a headache today.”
Moggallāna says, “It’s no wonder, with these yakkhas.” He relates the
scene. “This is wonderful. These mighty yakkhas can give a blow to
Sāriputta, and all he says is, ‘I’ve got a bit of a headache.’”

Then Sāriputta says, “What’s really amazing to me is that you can see
these things happening. I didn’t see so much as a mud sprite, let alone
a yakkha, a deva, or anything like that.” The sutta ends with these two
great disciples admiring each other’s qualities.

\qaspace
Q: What are the primary distinctions between Theravāda, Mahāyāna, and
Vajrayāna?

\qaspace
A: Without confusing people, I think Theravāda, just by the very literal
translation of its name, means the teaching of the elders. Its primary
focus is in trying to maintain the tradition as it was passed down as
closely as they could assume were the original teachings of the Buddha.
There are always some accretions and additions, but that has been its
primary flavor.

It’s also interesting that when the Dalai Lama gives teachings, one of
the things he will do is invite Mahāyāna, Theravāda, and Vajrayāna monks
to give blessings before the teachings begin. He will always invite the
Theravadans to give a Pali chant because he will say that the oldest
scriptures are in Pali, and these are the elder brothers within the
Buddhist fold. He gives deference. It is very beautiful.

In Buddhist history, there was a range of schools. About 250 years after
the Buddha, by the time of King Asoka, there were about eighteen
distinct schools of Buddhism. Subsequent to that, there began to be a
Mahāyāna movement, reinvigorating and redefining some of the principles.
One of the main principles is putting emphasis on the Bodhisattva ideal.

The Vajrayāna grew out of the Mahāyāna. There are several Vajrayāna
schools. Usually when someone says Vajrayāna, they mean Tibetan. The
Vajrayāna schools in Tibet actually took many of the different schools
and brought them to Tibet. One of the distinctive qualities of the
Vajrayana is the emphasis on faith, ceremony, ritual, and the efficacy
and use of that.

These are just snapshots. Of course, within these there are a huge
number of schools as well. When you look at Buddhism, there is not a
Theravāda tradition, a Mahāyāna tradition, and a Vajrayāna tradition
that are each homogenous and monolithic. A country like Thailand is
considered very strictly Theravāda. But if you start looking and digging
into the history of doctrinal and popular Buddhism, one thing looks like
a Mahāyāna teaching and then another one looks like a tantric ritual. So
there has been mixing and cross-pollination going on for millennia.

\qaspace
Q: What could American culture learn from Thai culture?

\qaspace
A: What immediately comes to mind is one of the first things you learn
when you go to Thailand. There is a phrase, \emph{mai pen rai}. That
would be a really good thing for American culture to learn. It means
“never mind,” as in, “let go, it’s not a big deal.”

\emph{Mai pen rai}: literally it means, “It is not anything.” It allows
people to navigate the thorny aspects of human relationships. \emph{Mai
pen rai.} That’s the way it is. Never mind. There is no point in getting
upset. \emph{Mai pen rai.}

Of course, Thailand has been steeped in Buddhist culture for a long,
long time. There is a book that was written by Phra Payutto; I don’t
think it has ever been translated into English. He had spent a few years
in America as a visiting scholar at Harvard University and Swarthmore
College and his English is very good. The book he wrote is about
Thailand viewing America and America viewing Thailand—the interchange,
the things each could learn, just seeing the possibilities.

That’s one of the things for myself, having grown up in North American
culture and having lived in Thailand for twenty-three years in a rural
environment without speaking English and so being able to be steeped in
a different culture. It was very helpful in being able to step back from
my cultural conditioning. It isn’t as if there is a perfect culture
anywhere. The human condition is just not that way. You think, “Wow, if
there was a culture that was steeped in Buddhism that would solve all
the problems of the world.” No, it wouldn’t. There are human beings
there. They’ll create suffering wherever they go.
