\chapter{Refuges and Precepts}

I’d like to welcome everyone who has gathered for this occasion and to
express appreciation to Susan, Karen, and Cassidy for pulling it
together and prodding me into giving a retreat. This is the second
retreat I’ve led in California in twelve years. The theme is
\emph{mettā}, loving-kindness, and the context is a retreat, a practice
situation. We are taking the opportunity to explore the different
expressions, development, and implications of loving-kindness.

As I understand it, we can’t extract loving-kindness out of the Buddha’s
teachings as an orphan that stands alone. It doesn’t work that way. We
pick up a corner of the Buddha’s teachings and everything comes along
with it, including mettā. It’s like I pick up this little tassel here
and, of course, everything connected to it starts coming up as
well—everything’s attached. Pretty soon, the cushion, the bell,
everything comes along with it.

The teachings of loving-kindness are within a whole context and setting.
Of course, one of the most important foundations for any aspect of the
teaching to arise out of is what we’ve just done: determining the
refuges and precepts. It’s not just a curious little ceremony that we
begin a retreat with and that’s the end of it. The refuges of Buddha,
Dhamma, and Sangha provide the foundation for the wholesome qualities of
the heart to arise. The precepts establish a strong foundation of
integrity and virtue, as well as an ability to cultivate, both
individually and in a group setting, a sense of trust: trust in oneself
and the whole environment that one is in.

Reflecting on and recollecting the Buddha, Dhamma, and Sangha as refuges
is always an important reminder for us. The reality is that we are
always taking refuge in something, whether it is in some distraction or
particular worry or fear. The mind goes there, and that’s where we
create our refuge. We rely on it and build our lives around it.

As we find out, that’s not a very satisfying refuge, but it’s what we do
as human beings. It’s like Ajahn Chah’s definition of what a human being
is: “A human being is a being with issues.” It’s always having an issue
with something or other. It keeps us busy and gives us something to live
for. And then we die—it goes on and on.

So, we redirect our attention to something that is a worthy refuge: the
Buddha as a historical figure and symbol of fundamental qualities, such
as wisdom, compassion, and purity. The first line of one of the morning
chants we do at the monastery is \emph{Buddho, susuddho
karuṇāmahaṇṇavo.} \emph{Buddho} is one who knows, one with wisdom.
\emph{Susuddho} is purity, and \emph{karuṇāmahaṇṇavo}, great compassion.

Even the recitation \emph{namo tassa bhagavato arahato
sammāsambuddhassa} refers to those qualities. \emph{Bhagavato} is the
Blessed One, who radiates the blessing and heart quality of compassion.
\emph{Arahato} is one who is far from defiling tendencies and thus,
pure. \emph{Sammāsambuddhassa} refers to the self-enlightened, wisdom
quality.

The traditional recollections of the Buddha always refer to these
qualities of wisdom, compassion, and purity. Of course, these are not
qualities belonging solely to the Buddha, but the Buddha is the
archetype of what is most fruitful for us to cultivate, pay attention
to, and use to create a balance within the heart.

Wisdom, compassion, and purity are the qualities that we cultivate and
pay attention to if we reflect on how our practice is doing and how our
life is going. Are these qualities being attended to? Am I out of
balance or missing something? If so, how can I reconfigure that, so I
pay attention to those aspects of the Buddha?

The Dhamma is the teaching in a conventional sense and also the
underlying truth of existence. There are fundamental truths that we are
able to realize, understand, and penetrate, and when those truths are
seen and understood, they have a transformative quality to them. The
Buddha himself said that whether a Buddha arises in the world or not,
there is still this underlying nature of things. All compounded things
are impermanent and unsatisfactory. All \emph{dhammas}, all things, are
not self.

It’s helpful when a Buddha comes along and points it out, but this is
also the underlying truth, the way things are. We can direct our
attention to and realize these fundamental truths.

The Sangha as a refuge, on a conventional level, consists of ordained
people who have stepped out of the world and made a commitment to
spiritual practice and training. Having these examples is a helpful
tool. This is a monastic retreat in the sense of having the monastic
sangha here. We have our own particular perspectives. Some people find
it useful, but it’s not everybody’s cup of tea. This is an opportunity
to get a bit of the flavor of having ordained sangha nearby.

However, when the Buddha refers to sangha outside of the conventional
level, he’s again pointing to qualities of the heart. In our morning
chant, when we recollect Sangha, we say: \emph{Supaṭipanno bhagavato
sāvakasaṅgho}, the Sangha of the Blessed One’s disciples are those who
practice well; \emph{ujupaṭipanno}, practice directly;
\emph{ñāyapaṭipanno}, practice for the overcoming of suffering, for
understanding and knowledge; \emph{sāmīcipaṭipanno}, practice
appropriately, with integrity. These are also the qualities of those who
have entered into the fruition of the path: stream-enterers,
once-returners, non-returners, and arahants.

When we come together to practice in a retreat situation, we take on the
eight precepts for the duration of the retreat. There’s a simplification
that takes place, having the eight precepts as a foundation. There’s a
very interesting discourse that the Buddha gave to Visākhā, one of his
foremost female disciples. He was praising the lay community’s keeping
of the lunar observance days once a week. In Pali, it’s called the
\emph{Uposatha}. Traditionally, laypeople would take the eight precepts
in the morning and keep them for a day and a night.

The way the Buddha described it is that the arahants, who are fully
enlightened, live their whole lives not harming, not taking the life of
any living creature. Laypeople are able to live just like arahants for a
day and a night, not taking the life of any living creature. Arahants
live their whole lives not taking what isn’t given, not stealing, and
committed to honesty. For a day and a night, you are able to live just
like an enlightened being, refraining from taking that which is not
given—and so on, through all the eight precepts.

This is basically how enlightened beings live; it’s their hardwired
structure. For us to gather for a week of practicing together is a very
special opportunity. Sometimes there is not much reflection when we ask
someone how a retreat went. They’ll say, “Oh, my mind was all over the
place; there was this ache and pain; there was this, and there was
that.” Nobody says, “Well, I lived for a week in simplicity, living like
the noble ones. It was great.” It ends up being hopelessly enmeshed in
“me” and “my mind.”

It’s no small thing to be able to say, “I got this opportunity to live
simply for this week. It was great.” It has a wonderful effect. We are
also doing it in an ideal, heavenly world here, surrounded by a lovely
forest.

We have this opportunity to come and practice walking and sitting
meditation together, having taken on the refuges and precepts. The theme
is loving-kindness, and I would encourage everybody to approach this
whole time of practice together within this sphere of loving-kindness.
Use walking and sitting meditation to bring up the quality of kindness.

Traditionally, in the classical modes of cultivating loving-kindness,
the first person you generate loving-kindness towards is yourself. So
just be kind to yourself during this period of retreat. It’s a great
opportunity to have time for solitude and quiet like this, to cultivate
mindfulness and continuity of awareness. Enjoy it. Don’t tie yourself up
in knots.

I think loving-kindness can’t be separated from the Dhamma or anything
else. Those qualities of letting go, non-contention: that’s where
loving-kindness arises. When we let go of our moods, obsessions,
worries, and fears—what’s left? Loving-kindness. We aren’t contending,
struggling, and fighting—feeling in opposition to the weather, the
circumstances, the people we are with, or ourselves, internally. There’s
just that non-contention, that non-struggle, that non-irritation. The
quality of loving-kindness is able to arise quite naturally as a result.

It’s a good idea to review how we’re holding the meditation as we begin
the practice. As we take time to establish and recollect the context of
sitting here, just relaxing the body and awareness, ask, “Where am I
holding, where am I tightening?” Say, “Okay, let’s just relax and
settle.” Take a couple of minutes sweeping through the body, paying
attention: “Let’s just relax here, let’s get settled.” Breath coming in
and going out; tune in to the breath and allow that feeling of settling
to establish itself. Then establish continuity of awareness with the
object of meditation.

I imagine that everybody here has experience in meditation. People will
have their own particular methods they are comfortable or familiar with.
My own inclination and ease of familiarity are with mindfulness of
breathing, so if I am giving instruction, I will almost invariably use
that as a framework in formal meditation.

If you are using a different object for some reason, it’s easy to
extrapolate the instruction for mindfulness of breathing and turn it to
whatever method or technique that you are using. It’s very similar with
walking meditation. Find a comfortable spot. If you like being in the
shade, be in the shade. If you like being in the sun, be in the sun.
Find a path that’s nice and level and long enough that you feel
comfortable—twenty, twenty-five, thirty paces, or whatever. If it’s too
long, it’s easy for the mind to start wandering. If it’s too short, it’s
difficult to get settled, going back and forth all the time. Find
something that’s comfortable for you.

The underlying theme is how to make it comfortable, bringing ease into
the practice, whether it’s walking or sitting. Or, if you like standing
meditation, you can do standing meditation. Again, it’s not so much the
posture or the method: it’s how you approach establishing the mind,
bringing attention to the object of meditation, sustaining and working
with continuity. Where does the mind go? Does it drift? Does it get
loose and nebulous? Or does it home in on the minutiae of the mind and
get tied up and tight?

Pay attention to what the mind is doing. Balance is an important quality
to bring to the practice. How can you best sustain that continuity of
awareness? That’s where the mind becomes peaceful. It doesn’t become
peaceful just by forcing the mind onto an object and holding it there.
Even if you succeed in doing that, it doesn’t actually make the mind
that peaceful.

For the mind to settle, there needs to be a balance of interested
application of energy and ease. If there’s too much forcing, it creates
tension; if it’s too loose, the mind gets nebulous and cloudy and
drifts. We need to balance the mind by working with what is.

Whether it’s sitting, walking, standing, or lying down, continuity
starts to kick in and allows the mind to settle. The Buddha talks about
cultivation of mindfulness in all four postures. It’s essential to
develop continuity in the different postures.

I remember the first year I was a monk: the evening meeting at the
monastery where I was would be an hour sitting, an hour standing, and an
hour walking, with some chanting at the beginning or end. That would be
the evening meeting—an hour each. I remember when I first started, I
thought, “Wow, great. They do standing meditation. Man, that’s going to
be good. My knees kill me when I’m sitting.”

Then I realized that the pain just goes somewhere else. What bodies do
is get uncomfortable. What is needed is to relax around the pain or the
discomfort and cultivate kindness.

There was one Rains Retreat I did when I had about fifteen or twenty
years as a monk, during which I had several months of lying-down
meditation. I had fractured my pelvis, so I wasn’t doing anything else.
But I was in very quiet, peaceful surroundings. I had lobbied the
doctor, who was a supporter of the monastery, and he had sprung me out
of the hospital and got me back to the monastery. I had said, “Whether
I’m in the hospital or at the monastery, I’m going to be lying down
anyway. I’ll be a lot happier out at the monastery.”

We realize that what’s important is the continuity of the awareness and
the quality of the attention that we pay to things. It’s not getting a
particular posture or technique down. The techniques and postures are
essential, but they’re tools, and it’s how you use a tool that’s
important. During the week that we have here, pay attention. “How am I
holding the meditation? How am I holding the perception? How am I
holding my body? How am I holding the things that are coming up in my
own mind? How am I holding the perceptions that I have of the people
around me?” That’s what is really important.

The mind in a mettā retreat can think, “All those other people are
peaceful. How come I’m not peaceful?” Of course, everybody else is
thinking the same thing: “All these other people look like they’re happy
and kind; they’re really getting loving-kindness. I’m obviously not
getting it.”

We can take loving-kindness as a theme, and then sometimes, what comes
up is how irritating the world is. But it doesn’t really matter. How are
we holding it? What is actually coming up in the mind, and how are we
relating to it? How are we layering our experience? That’s where the
kindness is, having that sense of non-contention, letting go: “This is
what’s coming up? Oh, that can be let go of. I don’t have to contend
with that. Back to the breath, back to the walking, back to the
continuity of awareness.”

These are a few reflections to begin the retreat with.
