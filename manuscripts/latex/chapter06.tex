\chapter{The Five Hindrances}

As we develop meditation, whether it’s mettā bhāvanā (the development of
loving-kindness) or other kinds of meditation, we always have to develop
skill in dealing with what are called the hindrances, \emph{nīvaraṇa} in
Pāli. It’s something that we need to be attentive to, not just in
meditation, but in daily life and our interactions with people. Whether
we’re walking, sitting, standing, or lying down, we need to be aware of
how these hindrances overwhelm or obscure the mind.

The hindrances that the Buddha pointed to are sensual desire, ill will,
sloth and torpor, restlessness and worry, and skeptical doubt. These
undermine the goodness, steadiness, and clarity of the mind, so it’s
important that we become aware of something as a hindrance, particularly
in the context of cultivation of the meditations on loving-kindness.
When ill will, anger, aversion, irritation, and negativity arise, there
should be an immediate flag going up in the mind that says, “Aha, this
is a hindrance.” Not, “That schmuck; how could they do that?” That’s the
immediate belief in a construct of negativity, rather than a sense of,
“Oh, this is a hindrance; now how do I deal with it?”

As with other aspects of meditation, one of the important foundations of
dealing with the hindrances is tuning in to the body. Recognizing, what
does the body feel like when there is sensual desire? Where do you feel
it? How does it make the body feel: the excitement, the anticipation,
the leaning into the hope of gratification?

What does the body feel like with ill will, negativity, and aversion:
the feeling of being on edge and tensing that energizes us? Of course
it’s not a very wonderful kind of energy, but sometimes it temporarily
feels better than nothing if you’re looking for something to get
energized with. Sometimes it seems that there’s nothing better than a
good rant to clear out the pipes, but that leaves debris everywhere.

Sloth and torpor: what does it feel like in the body? What’s the general
tone—listlessness, dullness? Restlessness and worry: the agitation you
feel isn’t just a mental event; the body is involved as well. Doubt:
that uncertainty, hesitation, inability to move forward, that feeling of
pulling back.

These are things that we feel in the body. We learn how to be attentive
to the experience of the hindrances as a whole body/mind process. As we
cultivate mindfulness of breathing and mindfulness of the body, things
start to become a bit clearer. You sense that this is a hindrance rather
than feeling out of sorts and uncomfortable and then looking for
something to blame, either externally or by blaming yourself. You can
reflect, “Oh, this is what I’m feeling.” Work with that; breathe through
it, bringing attention to the feeling with clear awareness.

Bring up something skillful and wholesome to work with. One of the
discourses of the Buddha begins with Ananda visiting a group of nuns.
There is an exchange concerning the nuns’ practice, and the nuns are
making particularly good progress in their cultivation of meditation and
training. Ananda tells the Buddha, and the Buddha gives a discourse,
approving the practice and adding that in the cultivation of mindfulness
and meditation, it is necessary to direct the mind to something
uplifting.

When the mind is scattered, diffused, or overwhelmed by dullness or
torpor, direct the attention to something that is uplifting and
pleasurable. The word in Pāli is \emph{pasādaniya-nimitta}: a sign of
that which is pleasurable or satisfying.

We are directing attention to mettā bhāvanā. That’s a particularly
pleasurable sign and object of attention to direct the mind to. This
isn’t done only to overcome ill will and aversion. It also can be very
helpful with sloth and torpor, when the mind is dull. When we direct the
mind to something pleasurable, the mind can take interest in it. Usually
we get trapped in sloth and torpor because we are not sufficiently
interested in something as neutral as the breath, but the object of
loving-kindness is very pleasurable. Or with restlessness and worry, we
can have a mind that is fidgety, worried, or anxious about this or that.
The mind will latch on to anything because it’s quite happy to worry
about anything. But give the mind something pleasurable, and it can take
some satisfaction in hanging out with that. Hang out with
loving-kindness, and let it churn. If the mind is going to churn, let it
churn with something skillful.

Sometimes people pick up a strongly adversarial approach to dealing with
unwholesome, unwanted states of mind. In Buddhist teachings, the
language can often take on a warrior-like tone. The Buddha was from the
warrior caste and did use those images, but they aren’t what he used all
the time.

This is a very good example of the different ways that the Buddha
approached the practice. Sometimes it’s solely to be mindful. It’s all
going to work out; we just have to be mindful. All we need to do is
gently bring the mind back to the object as an impartial observer. It
should be noted, though, that the impartial observer is often a fallacy.
We have our issues and agendas going all the time, whether we’re aware
of them or not. Part of the function of our practice is to understand
those agendas.

Another approach is to bring attention consciously to something
wholesome, skillful, pleasurable, and satisfying to the mind. On a
certain level you know that doubt, wavering, and uncertainty are
unskillful, but it’s hard to convince the doubtful and wavering mind
that loving-kindness is a good thing. You’ve got to work at it.

However, we need to be a bit circumspect. Feeding sensual desire with
pleasurable objects is not a good idea; we are pretty good at that
already. The appropriate way to address the tendency to sensual desire
is to take it to another level. Instead of fixating on that which seems
gratifying and pleasurable, realize, “Oh, there’s so much suffering in
this.” Focus on loving-kindness and realize the kindest thing is not
being trapped in sensual desire’s illusion of pleasure and
gratification.

The hindrances are a very fruitful realm for investigation, and
overcoming them is a necessary foundation to lay for our practice. As we
continue to cultivate, we need to be more and more skilled and quick at
noticing: Is that a hindrance? Is that an obstruction to that which is
truly skillful?

In another discourse, the Buddha points to the hindrances as being the
fuel or nourishment for ignorance, \emph{avijjā}. Ignorance, or not
knowing the true nature of things, is the ground that all suffering
arises out of. But that ignorance is not something that is immutable and
indivisible, or a fixed aspect of the mind. It comes into being fed and
nourished by certain conditions; the conditions that feed and nurture it
are the five hindrances. If we are working with the five hindrances, we
are very directly undermining that fundamental tendency towards avijjā,
the lack of true knowledge and awareness.

As we practice through the day, doing sitting and walking meditation, be
aware if a hindrance is present. This particular thought, mood,
perception, or feeling—could it be a hindrance?

If it is, work with it. Ground it in the awareness of the body. When the
Buddha describes the hindrances, he points out that when we’re able to
relinquish and let go of them, the mind is able to become more peaceful
and settled. When we’re able to relinquish the hindrance, joy, a sense
of well-being, and satisfaction come up in the mind.

The Buddha gives images to illustrate this. He compares one who is
overwhelmed by the hindrance of sensual desire to someone in debt. With
sensual desire, there’s always something hanging over you. When one is
in debt there’s always a sense of concern, so the mind isn’t able to
come to a place of ease. But when one has paid off one’s debts, there’s
a sense of happiness, well-being, and joy that arises quite naturally. I
think anybody who has finally paid off a mortgage thinks, “Wow, that
feels so good.” There is a sense of well-being and happiness that arises
when sensual desire is abandoned and let go of.

Ill will is compared to somebody who is sick and has a fever. Food
doesn’t agree with them; nothing tastes good. As long as there is fever
or illness, one feels out of sorts and uncomfortable, but when the
sickness goes away, food tastes good and the world looks better again. A
sense of happiness and well-being arises. The relinquishing and
abandoning of ill will and aversion feels good.

Sloth and torpor, the dullness and drowsiness that overtake the mind,
the Buddha compares to somebody who is in prison. When someone is put in
prison, he or she doesn’t have access to family or resources. There is a
great sense of loss and suffering, but if the person is released from
prison, property is restored, and connections with family
re-established, then he or she would feel extremely happy. There would
be a sense of ease and well-being. Similarly, a sense of ease and
well-being comes to one who is able to relinquish and abandon sloth and
torpor.

The Buddha compares restlessness and worry to slavery. Slaves aren’t
able to go where they want or do what they want; they’re not their own
masters. If they were granted the freedom to go where they want and do
what they want, they would feel happiness and well-being. It’s the same
way with the relinquishing of restlessness and worry. There is a
tremendous sense of well-being if we realize that, having been a slave
to restlessness, agitation, and worry, we are no longer worrying about
something and in particular, we are not worrying about what the next
thing to worry about is. To be able to put all that down and say,
“There’s just this body and this mind. There’s this peace and
awareness”—that’s a different world.

Skeptical doubt, the Buddha compares to somebody who has property and
possessions and who must travel through a desert or wilderness fraught
with danger and robbers. If that person makes a safe passage through
that desert or wilderness and gets to the destination with all property
intact, he or she would experience joy and well-being. In the same way,
when we are trapped in doubt, everything is fraught with danger,
difficulty, and uncertainty; we are always wondering what is going to
happen. If we relinquish that doubt, a sense of safety, refuge, and
certainty arises.

There are many tools to use in working with these hindrances, but
particularly during this retreat, please experiment with the application
of loving-kindness. See how it’s able to work as a means for undermining
the hindrances. A sense of wholesomeness is intrinsic to
loving-kindness. These are very beautiful states of mind. We can direct
attention to loving-kindness and have confidence that this is a
wholesome state of mind.

In the cultivation of more refined states of mind, as we relinquish the
hindrances, the mind has a sense of brightness and stability. In the
classical description of the first \emph{jhāna}, the mind is withdrawn
from sensual desires and unwholesome states of mind. Through the
strength of that withdrawal, that delighting in seclusion from the
agitation of sensual desire and from unwholesome states of aversion, ill
will, doubt, restlessness, sloth, and torpor, the mind is able to
settle.

We need to work at it to a certain extent. One of the descriptions that
the Buddha gives is a bath man or a bath man’s apprentice. In the
Buddha’s time they used a kind of clay for soap. You would take that
ball of clay and wet it, knead it, and permeate and pervade, suffuse and
fill that ball of clay, so that it’s not too wet but the moisture
completely pervades it. It doesn’t drip and the consistency is very
even. That’s what you would use to rub against yourself if you were in a
bath in the Buddha’s time.

The meditator does the same thing with his or her body and mind. It’s
interesting that the Buddha points to the body. The meditator permeates
and pervades, suffuses and fills the body with that delight in seclusion
from sensual desire and unwholesome states. Attend to that feeling of
seclusion, so that the mind is able to be present with the breath as it
comes in and goes out of the body, as the body is sitting and relaxing.
Fully permeating and pervading, suffusing and filling the breath, body,
and mind, with this quality of loving-kindness. Drawing back from and
relinquishing the five hindrances.

This is a present-moment practice. As you breathe in, there is a sense
of permeating and pervading, suffusing and filling. As you breathe out,
allow loving-kindness, the thoughts and feeling of well wishing,
kindness, and warmth, to fill the body and mind.

If any hindrances arise, if there’s any sensual desire, ill will, sloth
and torpor, restlessness, or doubt, rather than seeing it as an enemy
coming to attack and then gearing up to annihilate and destroy it, come
from a base of loving-kindness: “How do I work with this? This is not to
my benefit or well-being. What’s a skillful way of relinquishing this?
What’s a skillful way of letting this go?” From that place, you can be
as innovative as you want.

The classic advice for relinquishing sensual desire is the contemplation
of the unattractive nature of the body. Oftentimes this contemplation is
picked up with aversion, attending to the repugnant nature of the body.
Why does it have to be like that? With loving-kindness you can say,
“What’s the point in this constant attraction to something that is
always breaking down? It’s never comfortable. Why do I keep trying to
make it happy? Give it up.” There is a kindness there. “Oh, okay, I can
let that go.” Then what is left in the mind is very bright.

With aversion and ill will, rather than being fearful, approach it with
a sense of kindness: “This aversion, this ill will, why do I set myself
up in opposition to everything? Why do I keep getting into arguments
with it?” Even if you actually win the argument, there is still a
resonance of having got into an argument with yourself, a petty squabble
that you have maintained in your mind. “I can let that go.”

Bring a sense of kindness to working with sloth and torpor. Sometimes we
can get frustrated with sloth and torpor when the mind is dull, or we
get very idealistic, thinking we should be able to sit and not have
these intrusions of dullness. I think one of the things that is
important to understand is that loving-kindness is not acquiescing to
everything and saying everything is fine: “This sloth and torpor, isn’t
it lovely, wonderful?” or “That aggressive, obnoxious person, isn’t he
nice?” No, we don’t have to acquiesce. But we don’t have to get caught
up in aversion, don’t have to get trapped by the negativity.

So, we conserve a lot of energy by not fighting with sloth and torpor
and instead recognizing, “What’s a skillful way of working with this?”
Sometimes it’s a real kindness, rather than idealistically sitting and
struggling, just to get up and stand. Go and do some walking meditation.
Sit with your eyes open. “Sit with our eyes open? Do real meditators do
that?” Well, why not? A very simple thing can give energy. Sitting with
our eyes open can energize. We can pass through the period of sloth and
torpor; it’s actually fairly simple. But we do have to have kindness to
ourselves to allow that, and we tend not to.

With restlessness and worry, it’s helpful to tune in to the body. When
the mind is restless, it’s moving around, looking for something,
anything. So permeate and pervade, suffuse and fill the body with the
quality of loving-kindness. Breathe in and out, tuning in to the rhythm
and the feeling of the energy through the body. Even if it’s a restless
energy, allow and be very conscious of permeating and pervading,
suffusing and filling. Turn your attention to non-desire and step away
from any unwholesome states, negativity, and aversion.

Find a place in the body that does feel comfortable. Then find an object
that is pleasurable or satisfying, pasādaniya nimitta. Maybe your chest
or stomach is restless. Well, what about your hands? Can you just relax
your hands? Can you make that a pleasurable sensation? Pick a spot and
see that you can suffuse that particular area with a feeling of
well-being and seclusion from the unwholesome. From that you have a
base, and you can work on that to pervade the other areas of the body.

We are working in accord with that image of the bath man. It’s a lovely
image and very much body-based. And of course we do have a mind with
this body as well. That is, we can affect the mind directly without
getting entangled in it. Sometimes, seclusion is from the mind itself.

The mind goes off on a particular tangent, say, doubt. It’s doubting
about this, uncertain about that. Well, why do we have to make a
decision? Why do we even have to have a particular opinion or a view
about whether something is right or wrong? There is just this breath;
there’s just this sensation of the body, breathing in, breathing out.
This itself is a great act of kindness.

Use the theme of the five hindrances and the application of
loving-kindness, not simply as a theory or a mental state. Work it in
and see how it pervades the rest of your practice. It’s not just about
meditation. It isn’t as if you come into the meditation hall and the
five hindrances appear only then. This is a life skill; get familiar
with this. Also, the more skillful you are at becoming aware of how the
five hindrances work and how you can relinquish them, then when you do
come to sit, the more the mind is able to settle more quickly. So I
offer that for reflection today.
