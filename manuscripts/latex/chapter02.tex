\chapter{Mettā: A Mature Emotion}

\qaspace
Question: My activist neighbor is going to Nevada to register voters.
It’s difficult to convince myself that I’m sitting in meditation for the
benefit of all beings (not just giving myself a “gift”), let alone
explain a week of sitting to her. Could you comment?

\qaspace
Answer: There are a couple of different layers there. The main one that
leaps out is that I don’t think you need to explain yourself to
everybody, especially in the light of loving-kindness. It is usually not
an act of loving-kindness to try to compare and then explain yourself to
others. That’s usually an exercise in judgment of yourself in which you,
invariably, come up short. The very impulse to have to explain yourself
to others is almost always generated from the sense, “I must be doing
something wrong, and therefore, I have to explain myself.” You’re
inevitably going to come up short.

A good question is: “Is this an act of kindness towards myself?” And
then, “Is it a real act of kindness to explain myself to other people?
Do they actually need to know?” It’s something of an American compulsion
to explain yourself to others.

I remember the very first year that I came to America to help found
Abhayagiri. We were invited on almsround to a neighbor’s house, which is
about seven miles from the monastery. You walk down Tomki Road and East
Road on almsround to get to their house. It’s a long walk. A woman
pulled over and asked, “Where are you going? Do you need a ride?” I had
misjudged the time and we were running a bit late, so it was quite good
to have a ride. There was still maybe a mile or a mile and a half left.
It wasn’t that far, but in the distance of a mile or a mile and a half,
I learned more about that woman’s life than I had wanted to know. I had
never seen her before. There is a compulsion in the American psyche to
try to explain everything to everybody. I don’t know how much of a
kindness that is to other people.

So these are some basic thoughts in terms of the reflection around
loving-kindness: “Is it a kindness to oneself? Is it a kindness to
others?”

In the realm of meditation and activism, we can ask ourselves: “What is
better? What should I be doing?” There’s a tendency in the mind to think
in terms of either/or. Either I should be doing this, or I should be
doing that. Either this is right and that’s wrong or that’s wrong and
this is right.

That approach is very divisive and complicates things. Again, is it a
kindness to think in that way? It’s a real questioning of the underlying
ways that we relate to ourselves and the world around us: what is the
effect of that framework of right and wrong, either/or? It’s important
in terms of spiritual practice, particularly from this perspective of
loving-kindness, to get some space around that way of dividing things
up—separating them out and being in opposition to things.

I don’t think that is a helpful way to reflect on our experience because
how the Buddha structured things was not so much as right and wrong, but
more as skillful and unskillful, wholesome and unwholesome. We realize
that an activist who is registering voters is doing something wholesome
and skillful. That’s a good thing to do.

“Is it what \emph{I} need to be doing? Is it what I want to be doing
right now? Do I feel drawn to that?” Maybe yes, maybe no. It’s also
about being able to see somebody else’s skillfulness or wholesome
activity and delight in that, being able to derive a sense of
encouragement from others’ good actions without intimidating ourselves:
“Other people can and are happy to do that. That’s great. What I’d like
to do for this week, at least, is to have a chance for a period of
retreat—settling, creating an inner anchor for myself. That’s a
wholesome, good, and skillful thing to do.” It doesn’t mean that because
we have made the choice to be on retreat that we’re right and the rest
of those schmucks out there really blew it. These are very separate
realms.

We can encourage ourselves in the choices that we make without
undermining ourselves. That’s an act of kindness. We can also see how
other people choose to use their time and energy and encourage and
support them, or at least delight in the good that they are doing. That
is an aspect of our own wholesomeness as well, and we benefit from that.

There is a word in Buddhist jargon: in Pali, it is \emph{puñña}, in the
Thai language, \emph{boon}. Inevitably, Thai people who have been to the
West or have met a Westerner who is studying Buddhism ask, “How do you
translate \emph{boon}?” It’s one of those very difficult terms to
render. But the Buddha himself said, “\emph{Puñña} is another word for
happiness.” (A 7.62) It is the result of skillful action: good,
wholesome actions or activities that result in happiness and well-being
for you and for others.

Generosity and giving are puñña. Keeping precepts and virtue are puñña.
Meditation is puñña. Listening to teachings is puñña. Teaching is puñña.
Giving the opportunity for other people to access teachings is puñña.
Helping others, acts of service, are puñña. Delighting in the good that
other people do is puñña. Dedicating the blessings that come from your
own good actions is puñña.

There are many levels, but these are avenues for creating happiness. We
realize that we can tie them into the theme of loving-kindness. The
recognition of that which is wholesome and skillful is an act of
loving-kindness, as well as the commitment to doing those acts—also
realizing that the thought, “If I’m not meditating, I’m wasting my
time,” is not a fixed thing.

There are many avenues of wholesome and skillful action, and it’s
important to be able to have a recognition of the spectrum, so that we
can seize the opportunities, such as seeing somebody else doing
something skillful and then delighting in that. It’s expressed in the
Pali word \emph{anumodana}, delighting in the good that is done. You
don’t even have to do anything.

If you see somebody else doing something skillful, it doesn’t have to be
intimidating: “I just don’t measure up.” Nor do you have to be jealous:
“They’re not so good, really.” There is that sort of criticism, which is
a way to pull people down and put them in their place. It’s very
small-minded; it’s not kind to yourself and not kind to others.

Conversely, there is a spaciousness in the heart when you are neither
intimidated by others nor torturing yourself because you are not doing
quite the “right” thing in the “right” way. Feeling guilty about
something that you didn’t do properly just goes on and on. None of that
is a kindness to yourself or to others.

I remember one time traveling as a translator and attendant to a very
senior, well-known, and highly respected Thai monk, Luang Por
Paññānanda. A few people in the room here have met him before. When I
took a group to Thailand, we went to pay respects to him. Now, there is
someone with serious puñña, a whole life of giving. When we went, he was
sick and in the hospital, but it was a hospital that he had built. He
was about ninety-six at the time, but very bright.

I traveled with him in the 1980s and one time we were in New Zealand. It
was the evening session: chanting, meditation, Dhamma talk, and
questions afterwards. One of the questions that somebody asked fairly
early on was: “How do we deal with that feeling of guilt?” Of course, I
was familiar with that feeling, but the interesting thing was that when
I tried to translate it, I realized that I didn’t know what the word for
guilt is in Thai. I had been translating for teachers and studying the
language and the Dhamma for years, but my mind drew a blank, so I burst
out laughing. I explained to him what the question was, why I was
laughing, and how the concept was a bit distant in the Thai language and
culture. I had to explain to him what Westerners do with their minds to
make themselves feel guilty.

He listened and got this very concerned look on his face as I was
explaining how guilt works. When I finished the explanation, he said,
“Oh, that’s really suffering. Tell them not to do that.” It isn’t as if
Thai people don’t have these emotions. There is a very healthy place for
remorse, but not that complication of guilt, which is so easy to carry
around because of the strong sense of self, “me,” and judgment. We
judge, compare, and divide ourselves into “me” and this world that we
are either trying to live up to or being intimidated by. These are very
painful distinctions.

An act of loving-kindness comes with the attitude of “this is the way
things are,” in the sense that there is a recognition: “Well, that’s
just a feeling. I’m going to get some space around that, not contend
with it, and not slip into those thoughts of comparison and suffering
that accompany it.” That is an act of loving-kindness towards oneself.
It’s a very useful skill to be able to see somebody who is doing
something good and then be able not to make it a judgment about
yourself, torturing yourself.

I think that sometimes, when we see loving-kindness as one of the
\emph{brahmavihāras}, the divine abidings, that puts it out \emph{there}
somewhere: “The divine, that’s way off somewhere else.” One of the monks
in England has started translating brahmavihāra as “mature emotion,” and
that’s a skillful way of bringing attention to that aspect. It is a
mature emotion to be able to turn attention to loving-kindness,
compassion, sympathetic joy, and equanimity. The doorway into them all
is the quality of loving-kindness. Bring attention to that, consciously,
and then also bring attention to what obstructs it. We can start to pay
attention to the immature emotions, which are fairly accessible, and
recognize: “There’s that twinge of jealousy; there’s that twinge of
comparing; there’s that irritation.” Rather than letting them gain
momentum, you can ask: “How can I bring a mature emotion into this? How
can I create some space around this?”

The quality of loving-kindness creates space; it’s a very spacious
emotion. As we create that spaciousness, it’s also very solid. It’s not
a fleeting kind of pleasure, delight, or gratification. Irritation,
jealousy, and aversion come up in the mind, but they don’t create a
steadiness. They’re not stable feelings and emotions in the heart.

With that feeling of loving-kindness—as we tap into and direct attention
to it—we realize there is a stability and groundedness that comes from
it. We have a ground beneath that is not shaken by the vagaries of
either the internal world of our emotions and reactions or the external
world of change and praise and blame. When I think of people I’ve met
who embodied loving-kindness, there was a tremendous steadiness or
stability there.

Thinking of something being billed as a “Mettā Retreat,” what do we do?
Do we come and ooze niceness? That turns my stomach, actually. Or do we
put a lot of effort into trying to beam love everywhere? That can get
pretty tiring. Again, the people I’ve met who embodied loving-kindness
possessed tremendous stability.

I think of one particular circumstance in my life that was very helpful
and illuminating for me. I had been the abbot of the International
Forest Monastery in Thailand for a few years, not all that long, and I
was finding it overwhelming. I had a chance to go to England. Ajahn
Sumedho and many of the other senior western monks, old mates, were
there. I was still stuck in Thailand looking after a monastery. I wasn’t
sure whether I was capable of doing it, or if I even wanted to do it.
There were conflicts with the monks and duties with the lay community. I
was feeling overwhelmed.

Toward the end of my stay in England, which was just a few weeks, I was
dreading getting on the plane. It was going to be a miserable plane ride
because I was thinking: “Oh, I’ve got to go back to the monastery, and
there’s this monk and that monk, and blah, blah, blah.” One of the
senior monks in England, who had lived in Thailand for many years, had a
little package. He said, “When you get to Bangkok, please take this and
offer it to this particular monk.” This monk, Phra Payutto, had a good
reputation, but I hadn’t met him yet. He was also not so well known at
that time. Now, he is internationally renowned.

In those days he lived in a temple in Bangkok at the edge of Chinatown,
in the old central part of the city. The area that he was in wasn’t
particularly nice. Oftentimes in Bangkok, all of the monks’ shops are in
one area and then there is another area that is the clothing section.
The place where the temple was looked like they fixed the transmissions
of car engines there. Oil, grease, and car parts were everywhere as you
entered the temple. One of the things the temple also did was cremations
for the poorest of the poor people. There was always something
happening, and it was crowded.

I went to the back, where his dwelling place was. He radiated peace and
kindness in the center of this business, dirtiness, pollution, and
chaos. He was an anchor of peace, clarity, and kindness. Talking to him,
he was extraordinarily kind, and he received me very well. He’s now very
well-known and respected in Thailand. People come from all levels of
society to pay respects to him. He’s one of the very few monks in
Thailand to have completed all of his studies of the scriptures and Pali
when he was still a novice. When that happens, the King is traditionally
the sponsor for the ordination. He has a brilliant mind. Sometimes,
somebody with a brilliant mind is not necessarily attuned to other
people, but he had a real completeness; everything was suffused with
kindness.

It really struck me: “Okay, that’s a way that I can survive. If I could
turn my attention to those qualities of kindness and mettā, that would
hold me in good stead. Also, I’m going back to a quiet forest monastery
in the northeast of Thailand, and if I can’t get some peace out of that,
I’m doing something wrong with my holy life.” He was a wonderful
example.

Some of the people here also met him when I took the group to Thailand.
He received us so graciously. He is in constant pain; he has always been
plagued by illness. But although he is in constant pain, the quality of
kindness that radiates from him is very tangible.

You see, again, the sense of a mature emotion that is very steadying,
stabilizing, and grounding. These are not qualities that people are born
with. It’s not that they get it all; these are universal qualities that
anybody can cultivate and tap into. I think it’s an important reflection
to develop: “Oh, this is a mature emotion to direct attention to.” It
gives us the opportunity, whether one is in a family or social
situation, whether it’s dealing with Buddhists, people in society, or
oneself, to be able to direct attention to this very important quality.
